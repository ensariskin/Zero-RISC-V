% =============================================================================
% MAIN - RV32I 3-Way Superscalar İşlemci Tez Dokümantasyonu
% =============================================================================

% =============================================================================
% PREAMBLE - RV32I 3-Way Superscalar İşlemci Tez Dokümantasyonu
% =============================================================================

\documentclass[12pt,a4paper,oneside]{report}

% -----------------------------------------------------------------------------
% Temel Paketler
% -----------------------------------------------------------------------------
\usepackage[utf8]{inputenc}
\usepackage[T1]{fontenc}
\usepackage[turkish]{babel}
\usepackage{lmodern}

% -----------------------------------------------------------------------------
% Sayfa Düzeni
% -----------------------------------------------------------------------------
\usepackage[
    top=2.5cm,
    bottom=2.5cm,
    left=3cm,
    right=2.5cm
]{geometry}
\usepackage{setspace}
\onehalfspacing

% -----------------------------------------------------------------------------
% Grafik ve Şekiller
% -----------------------------------------------------------------------------
\usepackage{graphicx}
\usepackage{tikz}
\usetikzlibrary{shapes.geometric, arrows.meta, positioning, calc, fit, backgrounds}

% Fix TikZ + Turkish babel conflict
\makeatletter
\AtBeginDocument{%
  \@ifpackageloaded{babel}{%
    \addto\extrasturkish{\shorthandoff{=}<>}%
  }{}%
}
\makeatother

% -----------------------------------------------------------------------------
% Tablolar
% -----------------------------------------------------------------------------
\usepackage{booktabs}
\usepackage{longtable}
\usepackage{multirow}
\usepackage{array}
\usepackage{tabularx}

% -----------------------------------------------------------------------------
% Kod Listeleme
% -----------------------------------------------------------------------------
\usepackage{listings}
\usepackage{xcolor}

\definecolor{codegreen}{rgb}{0,0.6,0}
\definecolor{codegray}{rgb}{0.5,0.5,0.5}
\definecolor{codepurple}{rgb}{0.58,0,0.82}
\definecolor{backcolour}{rgb}{0.95,0.95,0.92}

\lstdefinestyle{verilogstyle}{
    backgroundcolor=\color{backcolour},
    commentstyle=\color{codegreen},
    keywordstyle=\color{blue}\bfseries,
    numberstyle=\tiny\color{codegray},
    stringstyle=\color{codepurple},
    basicstyle=\ttfamily\footnotesize,
    breakatwhitespace=false,
    breaklines=true,
    captionpos=b,
    keepspaces=true,
    numbers=left,
    numbersep=5pt,
    showspaces=false,
    showstringspaces=false,
    showtabs=false,
    tabsize=2,
    language=Verilog,
    morekeywords={logic, always_ff, always_comb, posedge, negedge, module, endmodule,
                  input, output, reg, wire, parameter, localparam, assign, begin, end,
                  if, else, case, endcase, default, for, generate, endgenerate,
                  typedef, enum, struct, packed, unpacked, interface, endinterface,
                  modport, clocking, endclocking, assert, property, sequence},
    literate={ı}{{\i}}1 {İ}{{\.I}}1 {ğ}{{\u{g}}}1 {Ğ}{{\u{G}}}1
             {ş}{{\c{s}}}1 {Ş}{{\c{S}}}1 {ü}{{\"u}}1 {Ü}{{\"U}}1
             {ö}{{\"o}}1 {Ö}{{\"O}}1 {ç}{{\c{c}}}1 {Ç}{{\c{C}}}1
}

\lstset{style=verilogstyle}

% -----------------------------------------------------------------------------
% Matematik
% -----------------------------------------------------------------------------
\usepackage{amsmath}
\usepackage{amssymb}
\usepackage{amsfonts}

% -----------------------------------------------------------------------------
% Referanslar ve Linkler
% -----------------------------------------------------------------------------
\usepackage{hyperref}
\hypersetup{
    colorlinks=true,
    linkcolor=blue,
    filecolor=magenta,
    urlcolor=cyan,
    citecolor=green,
    pdftitle={RV32I 3-Way Superscalar İşlemci},
    pdfauthor={},
    bookmarks=true
}
\usepackage{cleveref}

% -----------------------------------------------------------------------------
% Başlık ve Alt Başlık Formatları
% -----------------------------------------------------------------------------
\usepackage{titlesec}

\titleformat{\chapter}[display]
{\normalfont\huge\bfseries}{\chaptertitlename\ \thechapter}{20pt}{\Huge}
\titlespacing*{\chapter}{0pt}{-20pt}{40pt}

% -----------------------------------------------------------------------------
% Üst Bilgi ve Alt Bilgi
% -----------------------------------------------------------------------------
\usepackage{fancyhdr}
\pagestyle{fancy}
\fancyhf{}
\fancyhead[L]{\leftmark}
\fancyhead[R]{\thepage}
\renewcommand{\headrulewidth}{0.4pt}

% -----------------------------------------------------------------------------
% Özel Komutlar
% -----------------------------------------------------------------------------
% Sinyal isimleri için
\newcommand{\sig}[1]{\texttt{#1}}

% Modül isimleri için
\newcommand{\module}[1]{\textbf{\texttt{#1}}}

% Kayıt isimleri için
\newcommand{\reg}[1]{\texttt{#1}}

% Bit aralığı için
\newcommand{\bits}[2]{[#1:#2]}

% Önemli kavramlar için
\newcommand{\concept}[1]{\textit{#1}}

% Neden kutusu
\usepackage{tcolorbox}
\tcbuselibrary{skins,breakable}

\newtcolorbox{nedenbox}[1][]{
    colback=blue!5!white,
    colframe=blue!75!black,
    fonttitle=\bfseries,
    title=Neden Bu Tasarım?,
    #1
}

\newtcolorbox{dikkatbox}[1][]{
    colback=orange!5!white,
    colframe=orange!75!black,
    fonttitle=\bfseries,
    title=Dikkat,
    #1
}

\newtcolorbox{ornekbox}[1][]{
    colback=green!5!white,
    colframe=green!75!black,
    fonttitle=\bfseries,
    title=Örnek,
    #1
}

% -----------------------------------------------------------------------------
% Kısaltmalar
% -----------------------------------------------------------------------------
\usepackage{acronym}

% -----------------------------------------------------------------------------
% Float Kontrolü
% -----------------------------------------------------------------------------
\usepackage{float}
\usepackage{placeins}

% -----------------------------------------------------------------------------
% Dipnot
% -----------------------------------------------------------------------------
\usepackage{footnote}

% -----------------------------------------------------------------------------
% Algoritma
% -----------------------------------------------------------------------------
\usepackage{algorithm}
\usepackage{algpseudocode}
\floatname{algorithm}{Algoritma}
\renewcommand{\algorithmicrequire}{\textbf{Girdi:}}
\renewcommand{\algorithmicensure}{\textbf{Çıktı:}}


\begin{document}

% -----------------------------------------------------------------------------
% Başlık Sayfası
% -----------------------------------------------------------------------------
\begin{titlepage}
    \centering
    \vspace*{2cm}

    {\scshape\LARGE Üniversite Adı \par}
    \vspace{1cm}
    {\scshape\Large Elektrik-Elektronik Mühendisliği Bölümü \par}
    \vspace{2cm}

    {\huge\bfseries RV32I 3-Way Superscalar İşlemci Tasarımı ve Gerçeklemesi \par}
    \vspace{2cm}

    {\Large Yüksek Lisans Tezi \par}
    \vspace{2cm}

    {\large Hazırlayan: \par}
    {\Large\itshape Öğrenci Adı \par}
    \vspace{1cm}

    {\large Danışman: \par}
    {\Large\itshape Prof. Dr. Danışman Adı \par}

    \vfill

    {\large \today \par}
\end{titlepage}

% -----------------------------------------------------------------------------
% İçindekiler
% -----------------------------------------------------------------------------
\tableofcontents
\newpage

% -----------------------------------------------------------------------------
% Şekiller Listesi
% -----------------------------------------------------------------------------
\listoffigures
\newpage

% -----------------------------------------------------------------------------
% Tablolar Listesi
% -----------------------------------------------------------------------------
\listoftables
\newpage

% -----------------------------------------------------------------------------
% Kısaltmalar
% -----------------------------------------------------------------------------
\chapter*{Kısaltmalar}
\addcontentsline{toc}{chapter}{Kısaltmalar}

\begin{acronym}[BRAT]
    \acro{ALU}{Arithmetic Logic Unit -- Aritmetik Mantık Birimi}
    \acro{BRAT}{Branch Resolution Alias Table -- Dal Çözümleme Takma Ad Tablosu}
    \acro{BTB}{Branch Target Buffer -- Dal Hedef Tamponu}
    \acro{CDB}{Common Data Bus -- Ortak Veri Yolu}
    \acro{FIFO}{First In First Out -- İlk Giren İlk Çıkar}
    \acro{FSM}{Finite State Machine -- Sonlu Durum Makinesi}
    \acro{GHR}{Global History Register -- Küresel Geçmiş Kaydı}
    \acro{IPC}{Instructions Per Cycle -- Çevrim Başına Komut}
    \acro{ISA}{Instruction Set Architecture -- Komut Seti Mimarisi}
    \acro{LSQ}{Load Store Queue -- Yükleme/Saklama Kuyruğu}
    \acro{PC}{Program Counter -- Program Sayacı}
    \acro{PHT}{Pattern History Table -- Örüntü Geçmiş Tablosu}
    \acro{RAT}{Register Alias Table -- Kayıt Takma Ad Tablosu}
    \acro{RAS}{Return Address Stack -- Dönüş Adresi Yığını}
    \acro{RF}{Register File -- Kayıt Dosyası}
    \acro{ROB}{Reorder Buffer -- Yeniden Sıralama Tamponu}
    \acro{RS}{Reservation Station -- Rezervasyon İstasyonu}
    \acro{WAR}{Write After Read -- Okumadan Sonra Yazma}
    \acro{WAW}{Write After Write -- Yazmadan Sonra Yazma}
    \acro{RAW}{Read After Write -- Yazmadan Sonra Okuma}
\end{acronym}
\newpage

% =============================================================================
% BÖLÜMLER
% =============================================================================

% Bölüm 1: Giriş
% =============================================================================
% BÖLÜM 1: GİRİŞ
% =============================================================================

\chapter{Giriş}
\label{chap:giris}

\section{Motivasyon}
\label{sec:motivasyon}

Modern bilgisayar sistemlerinin performans gereksinimleri sürekli artmaktadır. Tek çevrimde
tek komut işleyen geleneksel skalar işlemciler, bu artan talepleri karşılamakta yetersiz
kalmaktadır. İşlemci frekansını artırmanın fiziksel sınırlarına (güç tüketimi, ısı dağılımı)
ulaşılmasıyla birlikte, performans artışı için farklı yaklaşımlar gerekli hale gelmiştir.

\concept{Superscalar} işlemciler, tek bir çevrimde birden fazla komutun paralel olarak
işlenmesine olanak tanıyarak bu soruna çözüm sunar. Bu yaklaşım, \concept{Komut Düzeyinde
Paralellik} (Instruction-Level Parallelism - ILP) kavramını kullanarak, programdaki
bağımsız komutları eşzamanlı yürütür.

Bu tez çalışmasında, RISC-V RV32I komut seti mimarisini destekleyen, 3 yollu (3-way)
superscalar bir işlemci tasarımı ve gerçeklemesi sunulmaktadır. Tasarım, Tomasulo
algoritmasını temel alarak sıra dışı (out-of-order) yürütme yeteneği sunarken,
spekülatif dal tahmini ile performansı maksimize etmeyi hedeflemektedir.

\section{Tezin Kapsamı}
\label{sec:kapsam}

Bu çalışma aşağıdaki konuları kapsamaktadır:

\begin{itemize}
    \item \textbf{RV32I Komut Seti Desteği:} RISC-V temel tamsayı komut setinin tam
          gerçeklemesi (37 komut)

    \item \textbf{3-Way Superscalar Mimari:} Her çevrimde en fazla 3 komutun paralel
          olarak decode, issue, dispatch ve commit edilmesi

    \item \textbf{Sıra Dışı Yürütme:} Tomasulo algoritması tabanlı dinamik zamanlama
          ile veri bağımlılıklarının donanım seviyesinde çözümlenmesi

    \item \textbf{Spekülatif Dal Tahmini:} Tournament, GShare ve 2-bit sayaç tabanlı
          dal tahmin mekanizmaları ile JALR hedef tahmini

    \item \textbf{Eager Misprediction Recovery:} Branch Resolution Alias Table (BRAT)
          ile sıfır gecikmeli yanlış tahmin düzeltmesi

    \item \textbf{Load/Store Queue:} Bellek operasyonlarının sıralı commit ile
          sıra dışı yürütülmesi
\end{itemize}

\section{Tasarım Hedefleri}
\label{sec:hedefler}

İşlemci tasarımında aşağıdaki hedefler gözetilmiştir:

\begin{enumerate}
    \item \textbf{Yüksek IPC (Instructions Per Cycle):} Teorik maksimum 3.0 IPC hedefine
          yaklaşmak için pipeline verimliliğinin maksimize edilmesi

    \item \textbf{Düşük Dal Cezası:} Yanlış tahmin durumunda minimum çevrim kaybı
          için eager recovery mekanizması

    \item \textbf{Ölçeklenebilirlik:} Parametrik tasarım ile farklı konfigürasyonlara
          kolayca uyarlanabilirlik

    \item \textbf{Doğrulanabilirlik:} Modüler yapı ve kapsamlı assertion'lar ile
          fonksiyonel doğrulama kolaylığı

    \item \textbf{Sentezlenebilirlik:} FPGA ve ASIC hedefleri için uygun RTL tasarımı
\end{enumerate}

\section{Tez Organizasyonu}
\label{sec:organizasyon}

Bu tez aşağıdaki şekilde organize edilmiştir:

\begin{description}
    \item[Bölüm 2 - Mimari Genel Bakış:] İşlemcinin genel mimarisi, pipeline yapısı
         ve temel tasarım prensipleri

    \item[Bölüm 3 - Fetch Stage:] Komut getirme aşaması, çoklu komut fetch, dal
         tahmini ve program sayacı yönetimi

    \item[Bölüm 4 - Issue Stage:] Komut decode, register renaming, kaynak tahsisi
         ve BRAT mekanizması

    \item[Bölüm 5 - Dispatch Stage:] Reorder Buffer, Reservation Station ve
         Register File yapıları

    \item[Bölüm 6 - Execute Stage:] ALU, shifter ve dal çözümleme birimleri

    \item[Bölüm 7 - Memory Stage:] Load/Store Queue ve bellek erişim yönetimi

    \item[Bölüm 8 - Performans Analizi:] Benchmark sonuçları ve karşılaştırmalı
         analiz

    \item[Bölüm 9 - Doğrulama Stratejisi:] Test metodolojisi ve doğrulama
         yaklaşımları

    \item[Bölüm 10 - Sonuç:] Çalışmanın özeti ve gelecek çalışmalar
\end{description}

\section{Katkılar}
\label{sec:katkilar}

Bu tez çalışmasının başlıca katkıları şunlardır:

\begin{enumerate}
    \item \textbf{BRAT Mekanizması:} Spekülatif dallanma için düşük gecikmeli
          recovery sağlayan yeni bir RAT snapshot yönetim yapısı

    \item \textbf{3-Port Circular Buffer:} ROB ve LSQ kaynak tahsisi için verimli
          bir paralel allocation/deallocation mekanizması

    \item \textbf{Combinational Bypass:} Dal çözümleme sonuçlarının aynı çevrimde
          tüm pipeline aşamalarına iletilmesi

    \item \textbf{Tournament Predictor:} Yerel ve global dal geçmişini birleştiren
          hibrit tahmin mekanizması

    \item \textbf{Açık Kaynak Gerçekleme:} Eğitim ve araştırma amaçlı kullanılabilecek
          tam dokümante edilmiş SystemVerilog kodu
\end{enumerate}


% Bölüm 2: Mimari Genel Bakış
% =============================================================================
% BÖLÜM 2: MİMARİ GENEL BAKIŞ
% =============================================================================

\chapter{Mimari Genel Bakış}
\label{chap:mimari}

Bu bölümde, RV32I 3-way superscalar işlemcinin genel mimarisi, pipeline yapısı ve
temel tasarım prensipleri açıklanmaktadır.

\section{Superscalar İşlemci Kavramı}
\label{sec:superscalar_kavram}

\subsection{Skalar ve Superscalar İşlemciler}

Geleneksel \concept{skalar} işlemciler, her çevrimde en fazla bir komut işler. Bu
yaklaşım basit ve öngörülebilir olmasına rağmen, modern uygulamaların performans
gereksinimlerini karşılamakta yetersiz kalır.

\concept{Superscalar} işlemciler, tek bir çevrimde birden fazla komutun paralel
olarak işlenmesine olanak tanır. N-way superscalar bir işlemci, teorik olarak
çevrim başına N komut (N IPC) işleyebilir.

\begin{nedenbox}
3-way superscalar tasarım seçilmesinin nedenleri:
\begin{itemize}
    \item \textbf{Donanım Karmaşıklığı:} 2-way'den belirgin performans artışı sağlarken,
          4-way'e göre daha düşük alan ve güç tüketimi
    \item \textbf{ILP Sınırları:} Tipik programlarda ortalama ILP 2-3 civarındadır;
          daha geniş issue width marjinal kazanç sağlar
    \item \textbf{Critical Path:} 3 paralel yol, kabul edilebilir clock period'u
          korurken yeterli paralellik sunar
\end{itemize}
\end{nedenbox}

\subsection{Komut Düzeyinde Paralellik (ILP)}

Superscalar işlemcilerin performansı, programdaki \concept{Komut Düzeyinde Paralellik}
(ILP) miktarına bağlıdır. ILP, aynı anda yürütülebilecek bağımsız komut sayısını ifade
eder.

ILP'yi sınırlayan faktörler:
\begin{itemize}
    \item \textbf{Veri Bağımlılıkları:} RAW, WAR, WAW hazard'ları
    \item \textbf{Kontrol Bağımlılıkları:} Dal komutları ve belirsiz akış
    \item \textbf{Kaynak Çekişmesi:} Sınırlı fonksiyonel birim sayısı
\end{itemize}

\section{Pipeline Yapısı}
\label{sec:pipeline}

İşlemci, 5 ana aşamadan oluşan bir pipeline yapısına sahiptir. Her aşama,
belirli görevleri yerine getirir ve bir sonraki aşamaya veri aktarır.

\subsection{Pipeline Aşamaları}

\begin{figure}[H]
\centering
\fbox{
\begin{minipage}{0.9\textwidth}
\centering
\textbf{Fetch} $\rightarrow$ \textbf{Issue} $\rightarrow$ \textbf{Dispatch} $\rightarrow$ \textbf{Execute} $\rightarrow$ \textbf{Commit}\\[0.5em]
(5-wide) \hspace{1cm} (3-wide) \hspace{1cm} (3-wide) \hspace{1.2cm} (3 FU) \hspace{1.2cm} (3-wide)
\end{minipage}
}
\caption{İşlemci pipeline aşamaları}
\label{fig:pipeline}
\end{figure}

\begin{table}[H]
\centering
\caption{Pipeline aşamaları ve genişlikleri}
\label{tab:pipeline_stages}
\begin{tabular}{llcp{6cm}}
\toprule
\textbf{Aşama} & \textbf{Görev} & \textbf{Genişlik} & \textbf{Açıklama} \\
\midrule
Fetch & Komut getirme & 5-wide & Bellekten 5 komut okuma, dal tahmini \\
Issue & Decode + Rename & 3-wide & Komut çözümleme, register renaming \\
Dispatch & ROB/RS yazma & 3-wide & Kaynak tahsisi, operand okuma \\
Execute & Yürütme & 3 FU & ALU, shifter, branch işlemleri \\
Commit & Retire & 3-wide & Sonuçların RF'e yazılması \\
\bottomrule
\end{tabular}
\end{table}

\begin{nedenbox}
\textbf{Neden 5-wide fetch, 3-wide issue?}

Fetch aşaması 5 komut getirirken, issue aşaması 3 komut işler. Bu asimetri,
instruction buffer'ın (IB) doldurulmasını sağlar:
\begin{itemize}
    \item Dal tahmini yanlışlığında IB boşalır
    \item 5-wide fetch, IB'yi hızlıca yeniden doldurur
    \item Issue aşaması, IB'den komut çekme hızından bağımsız çalışır
    \item Fetch stall durumlarında (cache miss) IB tampon görevi görür
\end{itemize}
\end{nedenbox}

\subsection{In-Order vs Out-of-Order Aşamalar}

\begin{table}[H]
\centering
\caption{Aşamaların sıralama özellikleri}
\label{tab:ordering}
\begin{tabular}{lll}
\toprule
\textbf{Aşama} & \textbf{Sıralama} & \textbf{Açıklama} \\
\midrule
Fetch & In-Order & Program sırasında komut getirme \\
Issue & In-Order & Program sırasında decode ve rename \\
Dispatch & Out-of-Order & Hazır komutlar önce yürütülür \\
Execute & Out-of-Order & Bağımsız yürütme \\
Commit & In-Order & Program sırasında sonuç yazma \\
\bottomrule
\end{tabular}
\end{table}

\section{Tomasulo Algoritması}
\label{sec:tomasulo}

Bu işlemci, sıra dışı yürütme için \concept{Tomasulo algoritmasını} kullanmaktadır.
1967'de Robert Tomasulo tarafından IBM System/360 Model 91 için geliştirilen bu
algoritma, veri bağımlılıklarını donanım seviyesinde dinamik olarak çözümler.

\subsection{Temel Kavramlar}

\subsubsection{Register Renaming}

\concept{Register renaming}, WAW (Write-After-Write) ve WAR (Write-After-Read)
hazard'larını ortadan kaldırır. Mimari register'lar (x0-x31) fiziksel register'lara
(bu tasarımda 64 adet) eşlenir.

\begin{lstlisting}[caption={Register renaming örneği}]
// Orijinal kod (WAW hazard)
ADD x1, x2, x3    // x1'e yaz
SUB x4, x1, x5    // x1'i oku
MUL x1, x6, x7    // x1'e yeniden yaz (WAW!)

// Renaming sonrası
ADD p32, p2, p3   // p32 <- x1'in yeni mapping'i
SUB p33, p32, p5  // p32'yi oku (RAW - gerçek bağımlılık)
MUL p34, p6, p7   // p34 <- x1'in en yeni mapping'i (WAW yok!)
\end{lstlisting}

\subsubsection{Reservation Station}

\concept{Reservation Station} (RS), yürütülmeyi bekleyen komutları tutar. Her RS
entry'si:
\begin{itemize}
    \item Operasyon türünü
    \item Kaynak operand değerlerini veya üretici komut tag'lerini
    \item Hedef register bilgisini
    \item Hazır durumunu
\end{itemize}
saklar.

\subsubsection{Common Data Bus (CDB)}

Yürütme sonuçları, \concept{Common Data Bus} üzerinden yayınlanır. Bekleyen tüm
RS entry'leri bu yayını izler ve eşleşen tag'leri yakaladığında operand değerini
günceller.

\subsection{Tomasulo Akışı}

\begin{enumerate}
    \item \textbf{Issue:} Komut decode edilir, register'lar rename edilir, RS'e yazılır
    \item \textbf{Dispatch:} Operandlar hazır olduğunda komut yürütmeye gönderilir
    \item \textbf{Execute:} Fonksiyonel birimde işlem yapılır
    \item \textbf{Write Result:} Sonuç CDB üzerinden yayınlanır
    \item \textbf{Commit:} ROB başındaki komut retire edilir, RF güncellenir
\end{enumerate}

\section{Spekülatif Yürütme ve BRAT}
\label{sec:speculative}

Dal komutları, program akışında belirsizlik yaratır. \concept{Spekülatif yürütme},
dal sonucu belirlenmeden önce tahmine dayalı olarak komut yürütmeye devam eder.

\subsection{Dal Tahmini}

İşlemci, üç farklı dal tahmin mekanizması içerir:
\begin{itemize}
    \item \textbf{2-Bit Sayaç:} Basit, düşük maliyetli yerel tahmin
    \item \textbf{GShare:} Global history ile PC XOR'lanarak indeksleme
    \item \textbf{Tournament:} Yerel ve global tahminlerin kombinasyonu
\end{itemize}

\subsection{Branch Resolution Alias Table (BRAT)}

Yanlış tahmin durumunda işlemci durumunun geri yüklenmesi gerekir. Geleneksel
yaklaşımda, yanlış tahmin yapan dal ROB başına ulaşana kadar beklenir. Bu,
onlarca çevrim gecikmeye neden olabilir.

\concept{BRAT}, her dal komutu için RAT'ın anlık görüntüsünü (snapshot) saklar.
Yanlış tahmin tespit edildiğinde, ROB başı beklenmeden anında geri yükleme yapılır.

\begin{nedenbox}
BRAT'ın temel avantajları:
\begin{itemize}
    \item \textbf{Sıfır Gecikmeli Recovery:} Dal yürütüldüğü çevrimde geri yükleme
    \item \textbf{Combinational Bypass:} Aynı çevrimde tüm modüllere sinyal iletimi
    \item \textbf{In-Order Resolution:} Program sırasında dal çözümleme garantisi
\end{itemize}
BRAT detayları \autoref{sec:brat}'te açıklanmaktadır.
\end{nedenbox}

\section{Kaynak Yönetimi}
\label{sec:kaynak_yonetimi}

3-way superscalar mimari, her çevrimde 3 komut için kaynak tahsisi gerektirir:
\begin{itemize}
    \item \textbf{ROB Entry:} Her komut için benzersiz bir ROB indeksi
    \item \textbf{Physical Register:} Hedef register yazan komutlar için
    \item \textbf{LSQ Entry:} Load/store komutları için
\end{itemize}

Bu tahsisler, \concept{3-Port Circular Buffer} yapısı ile yönetilir.
Detaylar \autoref{sec:circular_buffer}'da açıklanmaktadır.

\section{Modül Hiyerarşisi}
\label{sec:hiyerarsi}

\begin{figure}[H]
\centering
\fbox{
\begin{minipage}{0.95\textwidth}
\centering
\textbf{rv32i\_superscalar\_core}\\[1em]
\begin{tabular}{cccc}
\fbox{fetch\_stage} & \fbox{issue\_stage} & \fbox{dispatch\_stage} & \fbox{execute\_stage} \\[0.5em]
\scriptsize multi\_fetch & \scriptsize decoder x3 & \scriptsize ROB & \scriptsize ALU x3 \\
\scriptsize instr\_buffer & \scriptsize RAT & \scriptsize RS x3 & \scriptsize LSQ \\
\end{tabular}
\end{minipage}
}
\caption{Üst düzey modül hiyerarşisi}
\label{fig:hierarchy}
\end{figure}

\section{Veri Akışı}
\label{sec:veri_akisi}

\begin{enumerate}
    \item \textbf{Fetch $\rightarrow$ Issue:}
          5 komut + PC değerleri + dal tahmin bilgileri

    \item \textbf{Issue $\rightarrow$ Dispatch:}
          3 decoded komut + renamed operandlar + ROB/LSQ indeksleri

    \item \textbf{Dispatch $\rightarrow$ Execute:}
          Hazır komutlar + operand değerleri + kontrol sinyalleri

    \item \textbf{Execute $\rightarrow$ Commit:}
          Sonuç değerleri + dal çözümleme bilgileri

    \item \textbf{Commit $\rightarrow$ RF:}
          Retire edilen komutların sonuçları
\end{enumerate}

\section{Kontrol Akışı}
\label{sec:kontrol_akisi}

\subsection{Normal Operasyon}

Normal çalışmada, her aşama bağımsız olarak ilerler. Stall sinyalleri, kaynak
yetersizliğinde pipeline'ı durdurur:

\begin{itemize}
    \item \sig{rob\_full}: ROB dolu, issue durur
    \item \sig{rs\_full}: RS dolu, dispatch durur
    \item \sig{lsq\_full}: LSQ dolu, load/store issue durur
\end{itemize}

\subsection{Misprediction Recovery}

Yanlış tahmin tespit edildiğinde:
\begin{enumerate}
    \item BRAT, RAT snapshot'ını geri yükler
    \item Fetch, doğru PC'ye yönlendirilir
    \item Pipeline flush sinyali gönderilir
    \item ROB, spekülatif entry'leri temizler
    \item RS, spekülatif komutları iptal eder
\end{enumerate}

Tüm bu işlemler tek çevrimde gerçekleşir (combinational bypass sayesinde).


% Bölüm 3: Fetch Stage
% =============================================================================
% BÖLÜM 3: FETCH STAGE
% =============================================================================

\chapter{Fetch Stage}
\label{chap:fetch}

Fetch stage, işlemcinin ilk pipeline aşamasıdır ve bellekten komutların getirilmesinden
sorumludur. Bu bölümde, 5-wide fetch mimarisi, komut tamponu, dal tahmini mekanizmaları
ve program sayacı yönetimi detaylı olarak açıklanmaktadır.

\section{Genel Bakış}
\label{sec:fetch_genel}

Fetch stage, her çevrimde en fazla 5 komut getirerek instruction buffer'ı doldurur.
Issue stage ise bu tampondan 3 komut çeker. Bu asimetrik tasarım, pipeline
verimliliğini artırır.

\begin{figure}[H]
\centering
\fbox{
\begin{minipage}{0.9\textwidth}
\centering
\textbf{Fetch Stage Veri Akışı}\\[1em]
\begin{tabular}{ccccc}
\fbox{Inst Memory} & $\xrightarrow{\text{5 inst}}$ & \fbox{Multi Fetch} & $\xrightarrow{\text{5 inst}}$ & \fbox{Inst Buffer} \\[0.5em]
 & & $\downarrow$ & & $\downarrow$ \\[0.5em]
 & & \scriptsize PC Controller & & $\xrightarrow{\text{3 inst}}$ \fbox{Issue Stage} \\
 & & \scriptsize Branch Predictor & & \\
\end{tabular}
\\[1em]
\textit{Misprediction feedback: Issue Stage $\rightarrow$ Multi Fetch}
\end{minipage}
}
\caption{Fetch stage blok diyagramı}
\label{fig:fetch_block}
\end{figure}

\subsection{Fetch Stage Bileşenleri}

\begin{table}[H]
\centering
\caption{Fetch stage modülleri}
\label{tab:fetch_modules}
\begin{tabular}{llp{7cm}}
\toprule
\textbf{Modül} & \textbf{Dosya} & \textbf{Görev} \\
\midrule
Multi Fetch & \texttt{multi\_fetch.sv} & 5 komut paralel getirme koordinasyonu \\
Instruction Buffer & \texttt{instruction\_buffer\_new.sv} & Fetch-decode ayrıştırma tamponu \\
PC Controller & \texttt{pc\_ctrl\_super.sv} & Program sayacı yönetimi \\
Jump Controller & \texttt{jump\_controller\_super.sv} & Dal/atlama komut tespiti \\
Tournament Predictor & \texttt{tournament\_predictor.sv} & Hibrit dal tahmini \\
GShare Predictor & \texttt{gshare\_predictor\_super.sv} & Global history tabanlı tahmin \\
JALR Predictor & \texttt{jalr\_predictor.sv} & Dolaylı atlama hedef tahmini \\
\bottomrule
\end{tabular}
\end{table}

\section{Multi Fetch Modülü}
\label{sec:multi_fetch}

\module{multi\_fetch} modülü, fetch stage'in ana koordinasyon birimidir. Bellekten
5 komut okur, dal tahminlerini yapar ve instruction buffer'a iletir.

\subsection{Tasarım Amacı}

\begin{nedenbox}
\textbf{Neden 5-wide fetch?}

3-way issue için 5-wide fetch seçilmesinin nedenleri:
\begin{itemize}
    \item \textbf{Buffer Doldurma:} Misprediction sonrası instruction buffer hızla
          yeniden doldurulmalı. 5 > 3 olduğu için buffer birikir.
    \item \textbf{Dal Kesilmesi:} Bir dal ``taken'' tahmin edilirse, sonraki komutlar
          geçersiz olur. 5 komut getirerek, en az 1 geçerli komut garantilenir.
    \item \textbf{Fetch Bandwidth:} Modern bellek sistemleri geniş bant genişliği sunar.
          Bu kapasiteyi kullanmamak israftır.
\end{itemize}
\end{nedenbox}

\subsection{Komut Geçerlilik Mantığı}

Fetch edilen 5 komuttan bazıları geçersiz olabilir. Bir dal komutu ``taken'' tahmin
edilirse, ondan sonraki komutlar program akışında yer almaz:

\begin{lstlisting}[caption={Fetch geçerlilik sinyalleri}]
// Branch prediction invalidation logic
assign block_0 = jump_0 | jalr_0;
assign block_1 = jump_1 | jalr_1;
assign block_2 = jump_2 | jalr_2;
assign block_3 = jump_3 | jalr_3;

// Final fetch valid signals
assign fetch_valid_o[0] = base_valid;
assign fetch_valid_o[1] = base_valid & ~block_0;
assign fetch_valid_o[2] = base_valid & ~block_0 & ~block_1;
assign fetch_valid_o[3] = base_valid & ~block_0 & ~block_1 & ~block_2;
assign fetch_valid_o[4] = base_valid & ~block_0 & ~block_1 & ~block_2 & ~block_3;
\end{lstlisting}

\subsection{Misprediction İşleme}

BRAT'tan gelen misprediction sinyalleri, oldest-first öncelikle işlenir:

\begin{lstlisting}[caption={Eager flush mantığı}]
always_comb begin
    if (misprediction_i_0) begin
        eager_flush = 1'b1;
        eager_flush_target_pc = correct_pc_i_0;
    end else if (misprediction_i_1) begin
        eager_flush = 1'b1;
        eager_flush_target_pc = correct_pc_i_1;
    end else if (misprediction_i_2) begin
        eager_flush = 1'b1;
        eager_flush_target_pc = correct_pc_i_2;
    end else begin
        eager_flush = 1'b0;
        eager_flush_target_pc = {size{1'b0}};
    end
end
\end{lstlisting}

\begin{nedenbox}
\textbf{Neden oldest-first öncelik?}

Aynı anda birden fazla dal çözümlenebilir. En yaşlı misprediction önceliklidir çünkü:
\begin{itemize}
    \item Genç dallar, yaşlı dalın spekülatif yolunda olabilir
    \item Yaşlı misprediction düzeltildiğinde, genç dallar flush edilecek
    \item Genç misprediction'ları işlemek gereksiz çalışma olur
\end{itemize}
\end{nedenbox}

\section{Instruction Buffer}
\label{sec:instruction_buffer}

\module{instruction\_buffer\_new} modülü, fetch ve decode aşamalarını ayrıştıran
bir FIFO tampondur.

\subsection{Tasarım Amacı}

\begin{nedenbox}
\textbf{Neden instruction buffer gerekli?}
\begin{itemize}
    \item \textbf{Hız Uyumsuzluğu:} Fetch 5-wide, issue 3-wide. Buffer bu farkı dengeler.
    \item \textbf{Stall İzolasyonu:} Issue stall olduğunda fetch devam edebilir (buffer dolana kadar).
    \item \textbf{Misprediction Toleransı:} Buffer doluyken, recovery süresi kısalır.
    \item \textbf{Latency Gizleme:} Bellek gecikmesi buffer tarafından emilir.
\end{itemize}
\end{nedenbox}

\subsection{Buffer Yapısı}

Buffer, circular buffer olarak implement edilmiştir:

\begin{table}[H]
\centering
\caption{Instruction buffer parametreleri}
\label{tab:ibuf_params}
\begin{tabular}{lcp{6cm}}
\toprule
\textbf{Parametre} & \textbf{Değer} & \textbf{Açıklama} \\
\midrule
BUFFER\_DEPTH & 16 & Maksimum tamponlanabilir komut sayısı \\
Giriş Genişliği & 5-wide & Her çevrimde yazılabilecek komut \\
Çıkış Genişliği & 3-wide & Her çevrimde okunabilecek komut \\
\bottomrule
\end{tabular}
\end{table}

Her buffer entry'si şu alanları içerir:
\begin{itemize}
    \item \sig{instruction}: 32-bit komut kodu
    \item \sig{pc}: Komutun program sayacı değeri
    \item \sig{imm}: Önceden decode edilmiş immediate değeri
    \item \sig{branch\_prediction}: Dal tahmini sonucu
    \item \sig{pc\_at\_prediction}: Tahmin yapıldığındaki PC
    \item \sig{global\_history}: Dal predictor için global history
    \item \sig{ras\_tos\_checkpoint}: RAS checkpoint pointer
\end{itemize}

\subsection{Backpressure Yönetimi}

Buffer dolduğunda, fetch stage'e backpressure uygulanır:

\begin{lstlisting}[caption={Backpressure mantığı}]
// Conservative: leave space for 5 instructions
assign fetch_ready_o = !flush_i && !buffer_full_o && (space_available >= 5);
\end{lstlisting}

\begin{nedenbox}
\textbf{Neden \texttt{space\_available >= 5}?}

Fetch stage, mevcut çevrimde zaten 5 komut göndermiş olabilir. Bu komutlar henüz
buffer'a yazılmamışken (kombinasyonel gecikme), yeni fetch başlatılmamalı.
Konservatif yaklaşım deadlock'u önler.
\end{nedenbox}

\subsection{Forwarding Mantığı}

Buffer boşken ve fetch valid ise, komutlar doğrudan çıkışa forward edilir:

\begin{lstlisting}[caption={Direct forwarding}]
assign use_fwd_0 = (count == 0) & (num_to_write >= 1) & read_en_0;
assign use_fwd_1 = (count <= read_en_0) & (num_to_write >= (1 + use_fwd_0)) & read_en_1;
assign use_fwd_2 = (count <= read_en_0 + read_en_1) &
                   (num_to_write >= (1 + use_fwd_0 + use_fwd_1)) & read_en_2;
\end{lstlisting}

Bu mekanizma, buffer boşken bile zero-cycle forwarding sağlar, pipeline bubble'ları
minimize eder.

\section{PC Controller}
\label{sec:pc_controller}

\module{pc\_ctrl\_super} modülü, program sayacı yönetiminden sorumludur.

\subsection{PC Güncelleme Senaryoları}

\begin{table}[H]
\centering
\caption{PC güncelleme öncelikleri}
\label{tab:pc_priority}
\begin{tabular}{clp{6cm}}
\toprule
\textbf{Öncelik} & \textbf{Senaryo} & \textbf{Yeni PC Değeri} \\
\midrule
1 & Misprediction & \sig{correct\_pc} (BRAT'tan) \\
2 & JALR (tahminli) & \sig{jalr\_prediction\_target} \\
3 & Branch/JAL (taken) & \sig{PC + imm} \\
4 & Normal akış & \sig{PC + 20} (5 komut) \\
\bottomrule
\end{tabular}
\end{table}

\subsection{Paralel PC Hesaplama}

5 komut için PC değerleri paralel olarak hesaplanır:

\begin{lstlisting}[caption={Paralel PC hesaplama}]
assign current_pc_0 = pc_current_val;
assign current_pc_1 = parallel_mode ? pc_current_val + 32'd4 : current_pc_0;
assign current_pc_2 = parallel_mode ? pc_current_val + 32'd8 : current_pc_0;
assign current_pc_3 = parallel_mode ? pc_current_val + 32'd12 : current_pc_0;
assign current_pc_4 = parallel_mode ? pc_current_val + 32'd16 : current_pc_0;
\end{lstlisting}

\subsection{Misprediction Recovery}

Misprediction durumunda PC, BRAT'tan gelen doğru değere ayarlanır:

\begin{lstlisting}[caption={PC misprediction recovery}]
parametric_mux #(.mem_width(size), .mem_depth(2)) correction_mux(
    .addr(misprediction),
    .data_in({correct_pc, pc_plus}),
    .data_out(pc_new_val));
\end{lstlisting}

\section{Dal Tahmini}
\label{sec:branch_prediction}

İşlemci, üç farklı dal tahmin mekanizması içerir. Bu mekanizmalar, farklı dal
davranış paternlerini hedefler.

% =============================================================================
% DAL TAHMİNİ ALT BÖLÜMLERİ
% =============================================================================

\subsection{Dal Tahmini Genel Bakış}

Dal komutları, program akışında belirsizlik yaratır. Dal sonucu belirlenene kadar
(execute aşaması) işlemci, hangi komutları getireceğini bilemez. Dal tahmini,
bu belirsizliği spekülatif olarak çözerek pipeline verimliliğini artırır.

\begin{nedenbox}
\textbf{Neden dal tahmini kritik?}

Dal komutları tipik programlarda \%15-25 oranında görülür. Tahminsiz bir işlemcide
her dal 3+ çevrim gecikmeye neden olur. \%20 dal oranı ve 3 çevrim ceza ile:
\[
\text{Efektif IPC} = \frac{1}{1 + 0.20 \times 3} = 0.625
\]
Bu, teorik IPC'nin \%37.5 altındadır. Doğru tahmin bu kaybı minimize eder.
\end{nedenbox}

\subsection{Tahmin Mekanizmaları}

İşlemci üç farklı dal tahmin mekanizması içerir:

\begin{table}[H]
\centering
\caption{Dal tahmin mekanizmaları karşılaştırması}
\label{tab:predictors}
\begin{tabular}{lp{3cm}p{3cm}p{4cm}}
\toprule
\textbf{Mekanizma} & \textbf{Güçlü Yön} & \textbf{Zayıf Yön} & \textbf{En İyi Senaryo} \\
\midrule
2-Bit Sayaç & Basit, düşük maliyet & Korelasyon yakalayamaz & Tutarlı dallar (döngüler) \\
GShare & Global korelasyon & Aliasing sorunları & Korele dallar \\
Tournament & Adaptif seçim & Yüksek maliyet & Karışık iş yükleri \\
\bottomrule
\end{tabular}
\end{table}

\subsubsection{2-Bit Sayaç (Bimodal Predictor)}

En basit dal tahmin mekanizmasıdır. Her dal adresi için 2-bit doyurulmuş sayaç tutulur:

\begin{figure}[H]
\centering
\fbox{
\begin{minipage}{0.85\textwidth}
\centering
\textbf{2-Bit Sayaç Durum Makinesi}\\[1em]
\begin{tabular}{cccc}
\fbox{SNT (00)} & $\xrightarrow{\text{taken}}$ & \fbox{WNT (01)} & $\xrightarrow{\text{taken}}$ \\[0.3em]
$\circlearrowleft$ NT & & $\updownarrow$ not taken & \\[0.3em]
& & \fbox{WT (10)} & $\xrightarrow{\text{taken}}$ \\[0.3em]
& & $\updownarrow$ not taken & \\[0.3em]
& & \fbox{ST (11)} & $\circlearrowleft$ T \\
\end{tabular}
\\[1em]
\textit{SNT, WNT: Tahmin = Not Taken \quad|\quad WT, ST: Tahmin = Taken}
\end{minipage}
}
\caption{2-bit sayaç durum makinesi}
\label{fig:2bit_fsm}
\end{figure}

\begin{nedenbox}
\textbf{Neden 2-bit (1-bit değil)?}

1-bit sayaç, tek bir yanlış sonuçta hemen fikir değiştirir. Döngü sonlarında
bu sorunlu olur: döngü 100 kez ``taken'' olduktan sonra 1 kez ``not taken''
olur ve 1-bit sayaç hemen ``not taken'' tahmin etmeye başlar.

2-bit sayaç, iki ardışık yanlış sonuç gerektirir. Bu, döngü sonu gibi
``anomali'' durumlarına karşı dayanıklılık sağlar.
\end{nedenbox}

\subsubsection{GShare Predictor}

GShare, \concept{global history} ile PC'yi XOR'layarak indeks oluşturur.
Bu, farklı dallar arasındaki korelasyonu yakalar.

\begin{lstlisting}[caption={GShare indeks hesaplama}]
// GShare index = PC XOR Global History Register
assign predict_index_0 = current_pc_0[INDEX_WIDTH+1:2] ^ ghm0;
assign predict_index_1 = current_pc_1[INDEX_WIDTH+1:2] ^ ghm1;
...
\end{lstlisting}

\paragraph{Global History Register (GHR)}

GHR, son N dalın sonuçlarını (taken/not-taken) bir shift register'da tutar.
Her dal çözümlendiğinde, sonuç GHR'a shift edilir.

\begin{lstlisting}[caption={GHR güncelleme}]
// Per-slot history advance uses predicted bits
assign global_history_1 = slot_branch_0 ?
    {global_history_0[INDEX_WIDTH-2:0], branch_taken_o_0} : global_history_0;
assign global_history_2 = slot_branch_1 ?
    {global_history_1[INDEX_WIDTH-2:0], branch_taken_o_1} : global_history_1;
\end{lstlisting}

\begin{nedenbox}
\textbf{Neden spekülatif GHR güncellemesi?}

Dal sonucu execute aşamasında belli olur, ancak tahmin fetch aşamasında yapılır.
Eğer GHR güncellemesi commit'e kadar bekleseydi, ardışık dallar için yanlış
history kullanılırdı.

Spekülatif güncelleme, tahmin edilen sonucu GHR'a hemen ekler. Misprediction
durumunda GHR, BRAT'tan restore edilir.
\end{nedenbox}

\subsubsection{Tournament Predictor}

Tournament predictor, GShare ve bimodal predictor'ları birleştirir. Bir
\concept{chooser table}, hangi predictor'ın kullanılacağına karar verir.

\begin{lstlisting}[caption={Tournament chooser durumları}]
typedef enum logic [1:0] {
    STRONG_BIMODAL = 2'b00,
    WEAK_BIMODAL   = 2'b01,
    WEAK_GSHARE    = 2'b10,
    STRONG_GSHARE  = 2'b11
} chooser_state_e;
\end{lstlisting}

\paragraph{Chooser Güncelleme Kuralı}

Chooser, yalnızca iki predictor farklı tahmin yaptığında güncellenir:
\begin{itemize}
    \item Her iki predictor aynı tahmini yaparsa: chooser değişmez
    \item Predictor'lar farklı tahmin yaparsa: doğru olanın yönünde güncelle
\end{itemize}

\begin{nedenbox}
\textbf{Neden her iki predictor'ı da eğitiyoruz?}

Alternatif: Sadece seçilen predictor'ı eğitmek. Ancak bu yaklaşımda, chooser
yanlış predictor'a sabitlenirse, diğer predictor güncellenemez ve ``öğrenemez.''

Her iki predictor'ı eğitmek, chooser değiştiğinde diğer predictor'ın hazır
olmasını sağlar.
\end{nedenbox}

\paragraph{History Packing}

Tournament predictor, hem GShare hem de bimodal bilgisini BRAT'a kaydetmelidir.
Bu bilgi, \sig{global\_history\_*\_o} sinyalinde paketlenir:

\begin{lstlisting}[caption={Tournament history packing}]
// Layout of global_history bus (MSB..LSB):
//   [INDEX_WIDTH+2:3] = GHR_before   (INDEX_WIDTH bits, from gshare)
//   [2]               = gshare_pred
//   [1]               = bimodal_pred
//   [0]               = chooser_sel  (1 => gshare, 0 => bimodal)
\end{lstlisting}

\subsection{JALR Predictor}

JALR (Jump And Link Register) komutları, hedef adresi bir register'dan okur.
Bu nedenle, hedef adres fetch aşamasında bilinmez.

\module{jalr\_predictor} modülü, bir \concept{Branch Target Buffer} (BTB) kullanarak
JALR hedeflerini tahmin eder.

\begin{table}[H]
\centering
\caption{JALR predictor özellikleri}
\label{tab:jalr_predictor}
\begin{tabular}{lp{8cm}}
\toprule
\textbf{Özellik} & \textbf{Değer/Açıklama} \\
\midrule
Tablo Boyutu & 32 entry (parametrik) \\
İndeksleme & PC[INDEX\_WIDTH+1:2] \\
Saklanan Veri & Hedef PC adresi \\
Güncelleme & Execute aşamasından (gerçek hedef) \\
\bottomrule
\end{tabular}
\end{table}

\begin{nedenbox}
\textbf{Neden ayrı JALR predictor?}

Dal predictor'ları yön tahmini yapar (taken/not-taken). JALR için bu yeterli
değildir; hedef adres de tahmin edilmelidir.

JALR'lar genellikle fonksiyon dönüşleri (RET = JALR x0, ra, 0) veya dolaylı
çağrılardır. BTB, son kullanılan hedefi saklar ve çoğu durumda doğru tahmin
sağlar.
\end{nedenbox}

\subsection{Return Address Stack (RAS)}

Fonksiyon dönüşleri (RET) için özel bir yapı olan \concept{Return Address Stack},
CALL/RET çiftlerini takip eder.

\paragraph{RAS Operasyonları}
\begin{itemize}
    \item \textbf{CALL (JAL/JALR with rd=ra):} Dönüş adresini (PC+4) stack'e push et
    \item \textbf{RET (JALR x0, ra, 0):} Stack'ten pop et ve hedef olarak kullan
\end{itemize}

\paragraph{Spekülatif RAS ve Checkpoint}

RAS, spekülatif olarak güncellenir. Misprediction durumunda geri alınabilmesi
için, her dal komutu RAS top-of-stack pointer'ını BRAT'a kaydeder:

\begin{lstlisting}[caption={RAS checkpoint}]
output logic [2:0] ras_tos_checkpoint_o, // RAS TOS at fetch time
input  logic ras_restore_en_i,
input  logic [2:0] ras_restore_tos_i
\end{lstlisting}

\subsection{Predictor Güncelleme Akışı}

\begin{enumerate}
    \item \textbf{Fetch:} Tahmin yapılır, GHR spekülatif güncellenir
    \item \textbf{Issue:} Tahmin bilgisi BRAT'a kaydedilir
    \item \textbf{Execute:} Dal çözümlenir, gerçek sonuç belirlenir
    \item \textbf{BRAT Resolution:} Sonuç in-order olarak çıkar
    \item \textbf{Predictor Update:} Tablo güncellenir, GHR düzeltilir
\end{enumerate}

\begin{lstlisting}[caption={Predictor güncelleme ayrımı}]
// Branch predictor: update when update_valid & !is_jalr
assign branch_update_valid_0 = update_valid_i_0 & ~is_jalr_i_0;

// JALR predictor: update when update_valid & is_jalr
assign jalr_update_valid_0 = update_valid_i_0 & is_jalr_i_0;
\end{lstlisting}


\section{Fetch Stage Özeti}

\begin{table}[H]
\centering
\caption{Fetch stage özellikleri}
\label{tab:fetch_summary}
\begin{tabular}{lp{8cm}}
\toprule
\textbf{Özellik} & \textbf{Değer/Açıklama} \\
\midrule
Fetch Genişliği & 5 komut/çevrim \\
Buffer Derinliği & 16 entry \\
Dal Tahmin Yöntemleri & Tournament, GShare, 2-bit sayaç \\
JALR Tahmini & BTB tabanlı hedef tahmini \\
Misprediction Kaynağı & BRAT (eager recovery) \\
Backpressure & fetch\_ready\_o sinyali ile \\
\bottomrule
\end{tabular}
\end{table}


% Bölüm 4: Issue Stage
% =============================================================================
% BÖLÜM 4: ISSUE STAGE
% =============================================================================

\chapter{Issue Stage}
\label{chap:issue}

Issue stage, fetch edilen komutların decode edildiği, register renaming'in yapıldığı
ve kaynak tahsisinin gerçekleştirildiği pipeline aşamasıdır. Bu bölümde, 3-way
paralel decode, Register Alias Table (RAT), BRAT mekanizması ve kaynak yönetimi
detaylı olarak açıklanmaktadır.

\section{Genel Bakış}
\label{sec:issue_genel}

Issue stage, instruction buffer'dan 3 komut alır ve her biri için:
\begin{enumerate}
    \item Komut decode işlemi (RV32I format çözümleme)
    \item Register renaming (RAT lookup ve allocation)
    \item ROB/LSQ kaynak tahsisi
    \item Dal komutu ise BRAT'a snapshot kaydetme
\end{enumerate}
işlemlerini paralel olarak gerçekleştirir.

\begin{figure}[H]
\centering
\fbox{
\begin{minipage}{0.9\textwidth}
\centering
\textbf{Issue Stage Veri Akışı}\\[1em]
\begin{tabular}{ccccc}
 & & \fbox{Decoder 0} & & \\
\fbox{Inst Buffer} & $\rightarrow$ & \fbox{Decoder 1} & $\rightarrow$ & \fbox{RAT + BRAT + Free List} $\rightarrow$ \fbox{Dispatch} \\
 & & \fbox{Decoder 2} & & \\
\end{tabular}
\\[0.5em]
\scriptsize 3 komut paralel decode $\rightarrow$ Register renaming $\rightarrow$ Renamed operands
\end{minipage}
}
\caption{Issue stage blok diyagramı}
\label{fig:issue_block}
\end{figure}

\subsection{Issue Stage Bileşenleri}

\begin{table}[H]
\centering
\caption{Issue stage modülleri}
\label{tab:issue_modules}
\begin{tabular}{llp{6cm}}
\toprule
\textbf{Modül} & \textbf{Dosya} & \textbf{Görev} \\
\midrule
Issue Stage & \texttt{issue\_stage.sv} & Üst seviye koordinasyon \\
RV32I Decoder & \texttt{rv32i\_decoder.sv} & Komut format çözümleme (×3) \\
RAT & \texttt{register\_alias\_table.sv} & Register renaming \\
Circular Buffer & \texttt{circular\_buffer\_3port.sv} & ROB/LSQ allocation \\
BRAT & \texttt{brat\_circular\_buffer.sv} & Dal recovery snapshot \\
\bottomrule
\end{tabular}
\end{table}

\section{RV32I Decoder}
\label{sec:decoder}

Her issue slot için bağımsız bir \module{rv32i\_decoder} instance'ı bulunur.
Decoder, 32-bit komuttan kontrol sinyallerini çıkarır.

\subsection{Decoder Çıkışları}

\begin{table}[H]
\centering
\caption{Decoder çıkış sinyalleri}
\label{tab:decoder_outputs}
\begin{tabular}{lcp{6cm}}
\toprule
\textbf{Sinyal} & \textbf{Boyut} & \textbf{Açıklama} \\
\midrule
\sig{control\_word} & 26 bit & ALU operasyonu, memory erişimi, vb. \\
\sig{branch\_sel} & 3 bit & Dal koşulu seçimi \\
\sig{rs1\_arch} & 5 bit & Kaynak register 1 adresi \\
\sig{rs2\_arch} & 5 bit & Kaynak register 2 adresi \\
\sig{rd\_arch} & 5 bit & Hedef register adresi \\
\sig{rd\_write\_enable} & 1 bit & Hedef register yazma izni \\
\sig{load\_store} & 1 bit & Memory operasyonu mu? \\
\sig{branch} & 1 bit & Dal/atlama komutu mu? \\
\bottomrule
\end{tabular}
\end{table}

\subsection{RV32I Komut Formatları}

RV32I, 6 temel komut formatı tanımlar:

\begin{table}[H]
\centering
\caption{RV32I komut formatları}
\label{tab:rv32i_formats}
\begin{tabular}{lp{8cm}}
\toprule
\textbf{Format} & \textbf{Kullanım} \\
\midrule
R-type & Register-register ALU (ADD, SUB, AND, OR, ...) \\
I-type & Immediate ALU, load, JALR \\
S-type & Store komutları \\
B-type & Conditional branch \\
U-type & LUI, AUIPC \\
J-type & JAL \\
\bottomrule
\end{tabular}
\end{table}

\subsection{Paralel Decode}

3 decoder paralel çalışır, ancak aralarında bağımlılık yoktur:

\begin{lstlisting}[caption={Paralel decoder instantiation}]
// Decoder 0
rv32i_decoder #(.size(DATA_WIDTH)) decoder_0 (
    .instruction(instruction_i_0),
    .control_word(control_signal_internal_0),
    .branch_sel(branch_sel_internal_0)
);

// Decoder 1
rv32i_decoder #(.size(DATA_WIDTH)) decoder_1 (
    .instruction(instruction_i_1),
    .control_word(control_signal_internal_1),
    .branch_sel(branch_sel_internal_1)
);

// Decoder 2
rv32i_decoder #(.size(DATA_WIDTH)) decoder_2 (
    .instruction(instruction_i_2),
    .control_word(control_signal_internal_2),
    .branch_sel(branch_sel_internal_2)
);
\end{lstlisting}

\begin{nedenbox}
\textbf{Neden bağımsız decoder'lar?}

Decode aşaması kombinasyonel lojiktir ve komutlar arası bağımlılık gerektirmez.
Her komut, sadece kendi 32-bit instruction word'üne bakarak decode edilir.
Bu, tam paralel decode sağlar ve critical path'i minimize eder.
\end{nedenbox}

\section{Register Alias Table}

RAT, issue stage'in en kritik bileşenidir. Register renaming, kaynak tahsisi
ve spekülatif recovery mekanizmalarını içerir.

% =============================================================================
% RAT - REGISTER ALIAS TABLE
% =============================================================================

\section{Register Alias Table (RAT)}
\label{sec:rat}

Bu bölümde, Tomasulo algoritmasının temel bileşeni olan Register Alias Table (RAT)
yapısı detaylı olarak açıklanmaktadır.

\subsection{Tasarım Amacı}

\concept{Register renaming}, WAW (Write-After-Write) ve WAR (Write-After-Read)
hazard'larını ortadan kaldırmak için kullanılır. RAT, mimari register'ları
(x0-x31) fiziksel register'lara (0-63) eşler.

\begin{nedenbox}
\textbf{Neden register renaming gerekli?}

Aşağıdaki kod parçasını düşünün:
\begin{lstlisting}
ADD x1, x2, x3    // I1: x1'e yaz
SUB x4, x1, x5    // I2: x1'i oku (RAW - gerçek bağımlılık)
MUL x1, x6, x7    // I3: x1'e yaz (WAW - I1 ile)
AND x8, x1, x9    // I4: x1'i oku (RAW - I3 ile)
\end{lstlisting}

Renaming olmadan:
\begin{itemize}
    \item I3, I1 bitene kadar beklemeli (WAW)
    \item I4, I3 bitene kadar beklemeli
\end{itemize}

Renaming ile:
\begin{itemize}
    \item I1 → p32, I3 → p33 (farklı fiziksel register)
    \item I1 ve I3 paralel yürütülebilir
    \item Sadece gerçek RAW bağımlılıkları kalır
\end{itemize}
\end{nedenbox}

\subsection{RAT Yapısı}

\begin{table}[H]
\centering
\caption{RAT parametreleri}
\label{tab:rat_params}
\begin{tabular}{lcp{6cm}}
\toprule
\textbf{Parametre} & \textbf{Değer} & \textbf{Açıklama} \\
\midrule
ARCH\_REGS & 32 & Mimari register sayısı (x0-x31) \\
PHYS\_REGS & 64 & Fiziksel register sayısı \\
PHYS\_ADDR\_WIDTH & 6 bit & Fiziksel register adresi genişliği \\
\bottomrule
\end{tabular}
\end{table}

\paragraph{Fiziksel Register Alanı}

64 fiziksel register iki bölgeye ayrılır:
\begin{itemize}
    \item \textbf{0-31:} Register File (RF) - commit edilmiş değerler
    \item \textbf{32-63:} Reorder Buffer (ROB) - in-flight değerler
\end{itemize}

\begin{nedenbox}
\textbf{Neden 64 fiziksel register?}

32 mimari register + 32 ROB entry = 64 fiziksel register.
\begin{itemize}
    \item Her in-flight komut için 1 ROB entry gerekli
    \item 32 ROB entry, 32 komut paralel yürütme kapasitesi sağlar
    \item 3-way superscalar için bu yeterli buffer derinliği
\end{itemize}
\end{nedenbox}

\subsection{RAT Operasyonları}

\subsubsection{Kaynak Register Lookup}

Kaynak register'lar (rs1, rs2) için mevcut mapping okunur:

\begin{lstlisting}[caption={Kaynak register lookup}]
// Direct RAT lookup
assign rs1_phys_0 = rat_table[rs1_arch_0];
assign rs2_phys_0 = rat_table[rs2_arch_0];

// Same-cycle forwarding for dependent instructions
assign rs1_phys_1 = rs1_arch_1_equal_rd_arch_0 ? rd_phys_0 : rat_table[rs1_arch_1];
assign rs2_phys_1 = rs2_arch_1_equal_rd_arch_0 ? rd_phys_0 : rat_table[rs2_arch_1];

assign rs1_phys_2 = rs1_arch_2_equal_rd_arch_1 ? rd_phys_1 :
                    rs1_arch_2_equal_rd_arch_0 ? rd_phys_0 : rat_table[rs1_arch_2];
\end{lstlisting}

\begin{nedenbox}
\textbf{Neden same-cycle forwarding?}

Aynı çevrimde issue edilen 3 komut arasında bağımlılık olabilir:
\begin{lstlisting}
ADD x1, x2, x3    // Inst 0: x1'e yaz
SUB x4, x1, x5    // Inst 1: x1'i oku (Inst 0'a bağımlı)
\end{lstlisting}

RAT tablosu henüz güncellenmedi. Forwarding olmadan Inst 1, eski x1 mapping'ini
görür. Same-cycle forwarding, Inst 0'ın yeni rd\_phys değerini Inst 1'e iletir.
\end{nedenbox}

\subsubsection{Hedef Register Allocation}

Hedef register (rd) için yeni fiziksel register allocate edilir:

\begin{lstlisting}[caption={Hedef register allocation}]
always_comb begin
    // Instruction 0
    if (need_alloc_0 && found_first) begin
        allocated_phys_reg[0] = first_free;
        allocation_success[0] = 1'b1;
    end

    // Instruction 1
    if (need_alloc_1 && found_second) begin
        allocated_phys_reg[1] = second_free;
        allocation_success[1] = 1'b1;
    end

    // Instruction 2
    if (need_alloc_2 && found_third) begin
        allocated_phys_reg[2] = third_free;
        allocation_success[2] = 1'b1;
    end
end
\end{lstlisting}

\subsubsection{RAT Güncelleme}

Başarılı allocation sonrası RAT tablosu güncellenir:

\begin{lstlisting}[caption={RAT güncelleme}]
always_ff @(posedge clk) begin
    // Rename: Update RAT for new allocations
    if (need_alloc_0 && rd_arch_0 != 0) begin
        rat_table[rd_arch_0] <= allocated_phys_reg[0];
    end
    if (need_alloc_1 && rd_arch_1 != 0) begin
        rat_table[rd_arch_1] <= allocated_phys_reg[1];
    end
    if (need_alloc_2 && rd_arch_2 != 0) begin
        rat_table[rd_arch_2] <= allocated_phys_reg[2];
    end
end
\end{lstlisting}

\begin{nedenbox}
\textbf{Neden \texttt{rd\_arch != 0} kontrolü?}

RISC-V'de x0 register'ı sabit sıfırdır ve yazılamaz. x0'a yapılan yazmalar
görmezden gelinir. RAT'ta x0 her zaman fiziksel register 0'a map edilir.
\end{nedenbox}

\subsubsection{Commit İşleme}

ROB commit olduğunda, değer RF'e yazılır ve RAT güncellenir:

\begin{lstlisting}[caption={Commit işleme}]
// Commit: Restore architectural register to RF mapping
if (commit_valid[0] && commit_addr_0 != 0) begin
    if (commit_rob_idx_0 == rat_table[commit_addr_0][4:0]) begin
        rat_table[commit_addr_0] <= {1'b0, commit_addr_0};
    end
end
\end{lstlisting}

\begin{nedenbox}
\textbf{Neden ROB indeksi karşılaştırması?}

Aynı mimari register için birden fazla in-flight yazma olabilir:
\begin{lstlisting}
ADD x1, x2, x3    // ROB[5]: x1'e yaz
MUL x1, x4, x5    // ROB[8]: x1'e yaz
\end{lstlisting}

ADD commit olduğunda, x1 hâlâ MUL'un sonucuna (ROB[8]) bağlıdır.
RAT'ı RF'e döndürmek yanlış olur. Karşılaştırma, sadece ``en son yazma''
commit olduğunda RF mapping'e dönmeyi sağlar.
\end{nedenbox}

\subsection{3-Way Paralel Renaming}

3 komut aynı anda rename edilir. Bu, karmaşık bağımlılık kontrolü gerektirir:

\begin{lstlisting}[caption={3-way bağımlılık kontrolü}]
// Instruction 1 depends on Instruction 0?
assign rs1_arch_1_equal_rd_arch_0 = (rs1_arch_1 == rd_arch_0) &&
    (rd_arch_0 != 5'h0) && decode_valid[0] && rd_write_enable_0;

// Instruction 2 depends on Instruction 0 or 1?
assign rs1_arch_2_equal_rd_arch_0 = (rs1_arch_2 == rd_arch_0) &&
    (rd_arch_0 != 5'h0) && decode_valid[0] && rd_write_enable_0;
assign rs1_arch_2_equal_rd_arch_1 = (rs1_arch_2 == rd_arch_1) &&
    (rd_arch_1 != 5'h0) && decode_valid[1] && rd_write_enable_1;
\end{lstlisting}

\subsection{Kaynak Yönetimi}

RAT, ROB ve LSQ kaynak tahsisini de yönetir. Bu mekanizmalar aşağıdaki alt
bölümlerde detaylı olarak açıklanmaktadır.

% =============================================================================
% CIRCULAR BUFFER - ROB VE LSQ ALLOCATION
% =============================================================================

\subsection{3-Port Circular Buffer ile Kaynak Allocation}
\label{sec:circular_buffer}

Bu bölümde, ROB ve LSQ kaynak tahsisi için kullanılan 3-port circular buffer
yapısı detaylı olarak açıklanmaktadır.

\subsubsection{Problem: 3-Way Paralel Allocation}

3-way superscalar mimaride, her çevrimde en fazla 3 komut issue edilir. Her komut
potansiyel olarak şu kaynakları gerektirir:
\begin{itemize}
    \item 1 ROB entry (fiziksel register = ROB indeksi)
    \item 1 LSQ entry (load/store komutları için)
\end{itemize}

Geleneksel \concept{free list} yaklaşımında, 3 bağımsız boş kaynak bulmak karmaşık
priority encoder mantığı gerektirir. N entry'li bir free list için:
\begin{itemize}
    \item İlk boş entry'yi bul: O(N) tarama veya priority encoder
    \item İkinci boş entry'yi bul: O(N) tarama, ilkini hariç tut
    \item Üçüncü boş entry'yi bul: O(N) tarama, ilk ikisini hariç tut
\end{itemize}

Bu yaklaşım hem alan hem de zamanlama açısından maliyetlidir.

\begin{nedenbox}
\textbf{Neden geleneksel free list yetersiz?}
\begin{itemize}
    \item \textbf{Donanım Maliyeti:} 3 bağımsız priority encoder, her biri 32-64 bit
    \item \textbf{Critical Path:} Birinci sonuç ikinciye, ikinci üçüncüye bağımlı
    \item \textbf{Karmaşıklık:} Misprediction recovery için tüm allocation'ları track etmek gerekir
\end{itemize}
\end{nedenbox}

\subsubsection{Çözüm: Index-as-Value Circular Buffer}

Bu tasarımda, circular buffer'ın her entry'sinin değeri kendi indeksine eşittir:

\begin{lstlisting}[caption={Index-as-value circular buffer yapısı}]
// Buffer initialization
for (int i = 0; i < BUFFER_DEPTH; i++) begin
    buffer[i] = i;  // Entry[0]=0, Entry[1]=1, ..., Entry[31]=31
end
\end{lstlisting}

\paragraph{Temel Fikir}

Buffer, gerçek veri depolamaz. Sadece hangi indekslerin ``kullanılabilir'' olduğunu
yönetir:
\begin{itemize}
    \item \sig{read\_ptr}: Bir sonraki allocation'ın yapılacağı pozisyon
    \item \sig{write\_ptr}: Deallocation yapıldığında kullanılacak pozisyon
    \item \sig{count}: Mevcut kullanılabilir entry sayısı
\end{itemize}

\paragraph{Allocation (Okuma)}

Allocation, \sig{read\_ptr}'dan okuma ile yapılır:

\begin{lstlisting}[caption={3-port paralel allocation}]
// 3 parallel reads
assign read_data_0 = read_ptr;           // Allocated ID = read_ptr
assign read_data_1 = read_ptr + 1;       // Next ID
assign read_data_2 = read_ptr + 2;       // Next+1 ID

// Advance pointer by number of successful allocations
always_ff @(posedge clk) begin
    if (read_en_0 || read_en_1 || read_en_2)
        read_ptr <= read_ptr + read_count;
end
\end{lstlisting}

\paragraph{Deallocation (Yazma)}

Deallocation, \sig{write\_ptr}'a yazma ile yapılır (değer zaten sabit):

\begin{lstlisting}[caption={3-port paralel deallocation}]
// 3 parallel writes (from commit)
always_ff @(posedge clk) begin
    if (write_en_0 || write_en_1 || write_en_2)
        write_ptr <= write_ptr + write_count;
end
\end{lstlisting}

\begin{nedenbox}
\textbf{Neden bu tasarım üstün?}
\begin{itemize}
    \item \textbf{O(1) Complexity:} Priority encoder yok, sadece pointer aritmetiği
    \item \textbf{Paralel Erişim:} 3 allocation aynı anda, bağımsız olarak
    \item \textbf{Sıfır Depolama:} Gerçek veri saklanmaz, sadece pointer'lar
    \item \textbf{Basit Recovery:} Pointer reset ile tüm allocation'lar geri alınır
\end{itemize}
\end{nedenbox}

\subsubsection{Free List Yönetimi (ROB Allocation)}

ROB allocation için 32 entry'lik circular buffer kullanılır:

\begin{lstlisting}[caption={Free address buffer instantiation}]
circular_buffer_3port free_address_buffer(
    .clk(clk),
    .rst_n(reset),
    .redo_last_alloc(|branch_mispredicted_o),
    .read_en_0(need_alloc_0),
    .read_en_1(need_alloc_1),
    .read_en_2(need_alloc_2),
    .read_data_0(first_free),
    .read_data_1(second_free),
    .read_data_2(third_free),
    .read_valid_0(found_first),
    .read_valid_1(found_second),
    .read_valid_2(found_third),
    .write_en_0(commit_valid[0]),
    .write_en_1(commit_valid[1]),
    .write_en_2(commit_valid[2]),
    .set_read_ptr_en(free_addr_set_en),
    .set_read_ptr_value(free_addr_set_value)
);
\end{lstlisting}

\paragraph{ROB Allocation neden RAT'ta?}

Bu tasarımda, \concept{fiziksel register = ROB indeksi} eşitliği kullanılır:
\begin{itemize}
    \item Mimari register'lar (x0-x31): Fiziksel register 0-31 (RF'te)
    \item ROB entry'leri (0-31): Fiziksel register 32-63 (ROB'da)
\end{itemize}

RAT zaten register renaming yapıyor. Yeni hedef için ROB ID allocation doğal
olarak renaming sürecinin parçasıdır.

\subsubsection{LSQ Index Allocation}

Load/store komutları için ayrı bir circular buffer kullanılır:

\begin{lstlisting}[caption={LSQ address buffer instantiation}]
circular_buffer_3port #(.BUFFER_DEPTH(32)) lsq_address_buffer(
    .clk(clk),
    .rst_n(reset),
    .redo_last_alloc(|branch_mispredicted_o),
    .read_en_0(need_lsq_alloc_0),
    .read_en_1(need_lsq_alloc_1),
    .read_en_2(need_lsq_alloc_2),
    .write_en_0(lsq_commit_0),
    .write_en_1(lsq_commit_1),
    .write_en_2(lsq_commit_2),
    .set_read_ptr_en(lsq_flush_valid_i),
    .set_read_ptr_value(first_invalid_lsq_idx_i)
);
\end{lstlisting}

\begin{nedenbox}
\textbf{Neden ayrı LSQ buffer?}

LSQ allocation sadece load/store komutları için gerekli. Ayrı buffer tutmak:
\begin{itemize}
    \item ROB ve LSQ yaşam döngülerini bağımsız yönetir
    \item Her yapı kendi hızında dolup boşalabilir
    \item Misprediction recovery ayrı ayrı yapılabilir
\end{itemize}
\end{nedenbox}

\subsubsection{Misprediction Recovery}

Misprediction durumunda, yanlış yolda yapılan allocation'lar geri alınmalıdır.
İki mekanizma kullanılır:

\paragraph{1. Pointer Reset (\sig{set\_read\_ptr\_en})}

Misprediction tespit edildiğinde, \sig{read\_ptr} mispredicting instruction'ın
allocation noktasına reset edilir:

\begin{lstlisting}[caption={Misprediction pointer reset}]
always_comb begin
    if (brat_resolved_0 && brat_mispredicted_0) begin
        free_addr_set_en = 1'b1;
        free_addr_set_value = brat_resolved_phys_0 + 1;
    end else if (brat_resolved_1 && brat_mispredicted_1) begin
        free_addr_set_en = 1'b1;
        free_addr_set_value = brat_resolved_phys_1 + 1;
    end else ...
end
\end{lstlisting}

\begin{nedenbox}
\textbf{Neden \texttt{+1}?}

Mispredicting branch'in kendi allocation'ı geçerlidir. Sadece ondan sonraki
allocation'lar geri alınmalı. Bu yüzden yeni \sig{read\_ptr} = branch'in
fiziksel register'ı + 1.
\end{nedenbox}

\paragraph{2. Redo Last Allocation (\sig{redo\_last\_alloc})}

Misprediction aynı çevrimde tespit edilirse, o çevrimdeki allocation'lar henüz
commit edilmemiştir. Bu sinyal, son allocation'ı geri alır:

\begin{lstlisting}[caption={Redo last allocation}]
always_ff @(posedge clk) begin
    if (redo_last_alloc) begin
        read_ptr <= read_ptr - last_alloc_count;
    end
end
\end{lstlisting}

\subsubsection{Buffer Doluluk Kontrolü}

Allocation'a hazır olup olmadığını belirleyen sinyaller:

\begin{lstlisting}[caption={Rename ready sinyalleri}]
// ROB allocation ready
assign rename_ready = (free_count >= 3) ? 3'b111 :
                      (free_count == 2) ? 3'b011 :
                      (free_count == 1) ? 3'b001 : 3'b000;

// LSQ allocation ready
assign lsq_alloc_ready = (lsq_free_count >= 3) ? 3'b111 :
                         (lsq_free_count == 2) ? 3'b011 :
                         (lsq_free_count == 1) ? 3'b001 : 3'b000;
\end{lstlisting}

Bu sinyaller, issue stage'e kaç komutun kabul edilebileceğini bildirir.

\subsubsection{Circular Buffer Özet Tablosu}

\begin{table}[H]
\centering
\caption{Circular buffer özellikleri}
\label{tab:circular_buffer}
\begin{tabular}{lcc}
\toprule
\textbf{Özellik} & \textbf{Free List (ROB)} & \textbf{LSQ Buffer} \\
\midrule
Derinlik & 32 entry & 32 entry \\
Port Sayısı & 3 read, 3 write & 3 read, 3 write \\
Allocation & Issue aşamasında & Load/store issue'da \\
Deallocation & ROB commit'te & LSQ commit'te \\
Reset Kaynağı & BRAT misprediction & BRAT misprediction \\
\bottomrule
\end{tabular}
\end{table}


% =============================================================================
% BRAT - BRANCH RESOLUTION ALIAS TABLE
% =============================================================================

\subsection{Branch Resolution Alias Table (BRAT)}
\label{sec:brat}

Bu bölümde, spekülatif dal tahmini recovery'si için kullanılan BRAT mekanizması
detaylı olarak açıklanmaktadır.

\subsubsection{Problem: Geleneksel Misprediction Recovery}

Geleneksel Tomasulo tabanlı işlemcilerde, misprediction recovery şu şekilde çalışır:

\begin{enumerate}
    \item Dal komutu execute edilir, misprediction tespit edilir
    \item Dal komutu ROB başına ulaşana kadar beklenir
    \item ROB başında recovery başlar: RAT sıfırlanır, pipeline flush edilir
    \item Doğru yoldan fetch yeniden başlar
\end{enumerate}

\begin{figure}[H]
\centering
\fbox{
\begin{minipage}{0.85\textwidth}
\centering
\textbf{Geleneksel Recovery}\\[0.5em]
ROB: \fbox{0 (Head)} \fbox{1} \fbox{\textcolor{red}{2 (Mispred)}} \fbox{3} \fbox{4} \fbox{5} \fbox{6} \fbox{7}\\[0.5em]
\textit{Bekleme: Mispredicting branch (2) ROB head'e ulaşmalı $\rightarrow$ 2 commit bekle}
\end{minipage}
}
\caption{Geleneksel recovery: Dal ROB head'e ulaşmalı}
\label{fig:traditional_recovery}
\end{figure}

\begin{nedenbox}
\textbf{Neden geleneksel yaklaşım yavaş?}

Mispredicting dal ROB'un ortasındaysa, önündeki tüm komutların commit olması
beklenir. 32 entry'lik ROB'da, dal 16. pozisyondaysa, 15 commit beklenir.
Her commit 1 çevrimde 3 komut işlese bile, bu 5+ çevrim gecikme demektir.
Spekülatif dallar için bu gecikme kabul edilemez.
\end{nedenbox}

\subsubsection{Çözüm: BRAT ile Eager Recovery}

BRAT, her dal komutu için RAT'ın anlık görüntüsünü (snapshot) saklar. Misprediction
tespit edildiğinde, ROB başı beklenmeden anında geri yükleme yapılır.

\begin{figure}[H]
\centering
\fbox{
\begin{minipage}{0.85\textwidth}
\centering
\textbf{BRAT: Snapshot Tabanlı Anında Recovery}\\[1em]
\begin{tabular}{ccc}
\fbox{Branch 0 Snapshot} & \fbox{Branch 1 Snapshot} & \fbox{Branch 2 Snapshot} \\
\scriptsize (Head) & & \scriptsize (Tail) \\
\end{tabular}
\\[0.5em]
$\uparrow$ Execute Result (Match?) \\[0.5em]
$\downarrow$ Mispred $\rightarrow$ RAT Restore \\[0.5em]
\textit{ROB head beklenmeden anında recovery}
\end{minipage}
}
\caption{BRAT: Snapshot tabanlı anında recovery}
\label{fig:brat_recovery}
\end{figure}

\subsubsection{BRAT Entry Yapısı}

Her BRAT entry'si şu alanları saklar:

\begin{table}[H]
\centering
\caption{BRAT entry alanları}
\label{tab:brat_entry}
\begin{tabular}{llp{6cm}}
\toprule
\textbf{Alan} & \textbf{Boyut} & \textbf{Açıklama} \\
\midrule
\sig{branch\_phys} & 6 bit & Branch'in fiziksel register ID'si (ROB ID) \\
\sig{rat\_snapshot} & 32×6 bit & Tüm RAT mapping'inin kopyası \\
\sig{resolved} & 1 bit & Branch execute edildi mi? \\
\sig{mispredicted} & 1 bit & Tahmin yanlış mıydı? \\
\sig{correct\_pc} & 32 bit & Doğru hedef PC \\
\sig{is\_jalr} & 1 bit & Branch mi JALR mı? \\
\sig{pc\_at\_prediction} & 32 bit & Predictor update için orijinal PC \\
\sig{global\_history} & 8+ bit & Branch predictor history \\
\sig{ras\_tos} & 3 bit & RAS checkpoint pointer \\
\bottomrule
\end{tabular}
\end{table}

\begin{nedenbox}
\textbf{Neden bu kadar veri saklanıyor?}

Recovery sadece RAT restore değildir. Misprediction sonrası:
\begin{itemize}
    \item RAT restore edilmeli (snapshot)
    \item Fetch doğru PC'ye yönlendirilmeli (correct\_pc)
    \item Branch predictor güncellenmeli (pc\_at\_prediction, global\_history)
    \item RAS restore edilmeli (ras\_tos)
    \item JALR predictor güncellenmeli (is\_jalr)
\end{itemize}
Tüm bilgileri tek yerde tutmak, tek çevrimde recovery sağlar.
\end{nedenbox}

\subsubsection{BRAT Operasyonları}

\paragraph{1. Push (Dal Issue Edildiğinde)}

Yeni dal komutu issue edildiğinde, BRAT'a entry eklenir:

\begin{lstlisting}[caption={BRAT push mantığı}]
always_comb begin
    brat_push_en[0] = decode_valid[0] && branch_0 && !brat_full && !brat_restore_en;
    brat_push_en[1] = decode_valid[1] && branch_1 && !brat_full && !brat_restore_en;
    brat_push_en[2] = decode_valid[2] && branch_2 && !brat_full && !brat_restore_en;

    // Push snapshots - Store RAT state AFTER the branch instruction
    brat_push_snapshot_0 = rat_after_inst0;
    brat_push_snapshot_1 = rat_after_inst1;
    brat_push_snapshot_2 = rat_after_inst2;
end
\end{lstlisting}

\begin{nedenbox}
\textbf{Neden ``daldan sonraki'' RAT state saklanıyor?}

Snapshot, dalın kendi allocation'ını içermelidir. Misprediction durumunda:
\begin{itemize}
    \item Dalın kendisi geçerlidir (doğru yolun parçası)
    \item Daldan \textit{sonraki} komutlar geçersizdir
\end{itemize}
Bu yüzden snapshot, dalın allocation'ını içeren RAT state'i saklar.
\end{nedenbox}

\paragraph{2. Execute Result Yazma}

Dal execute edildiğinde, sonuç BRAT'a yazılır:

\begin{lstlisting}[caption={Execute result matching}]
// Execute result write interface
.exec_valid_0(exec_branch_valid_i[0]),
.exec_rob_id_0(exec_rob_id_0_i),
.exec_mispredicted_0(exec_mispredicted_i[0]),
.exec_correct_pc_0(exec_correct_pc_0_i),
\end{lstlisting}

BRAT, gelen ROB ID'yi tüm entry'lerle karşılaştırır. Eşleşen entry'nin
\sig{resolved} ve \sig{mispredicted} alanları güncellenir.

\paragraph{3. In-Order Resolution Output}

BRAT, dal sonuçlarını program sırasında çıkarır:

\begin{lstlisting}[caption={In-order resolution çıkışı}]
// BRAT ensures in-order branch resolution outputs
assign branch_resolved_o = {brat_resolved_2, brat_resolved_1, brat_resolved_0};
assign branch_mispredicted_o = {brat_mispredicted_2, brat_mispredicted_1, brat_mispredicted_0};
\end{lstlisting}

\begin{nedenbox}
\textbf{Neden in-order resolution kritik?}

Dallar out-of-order execute edilebilir. Ancak misprediction işleme sırası önemlidir:
\begin{itemize}
    \item Branch A (eski) ve Branch B (genç) aynı anda mispredicted olsun
    \item B, A'nın spekülatif yolunda olabilir
    \item A'nın misprediction'ı düzeltilirse, B zaten geçersiz
    \item B'yi önce işlemek gereksiz çalışma olur
\end{itemize}
BRAT, circular buffer yapısıyla doğal olarak oldest-first sıralama sağlar.
\end{nedenbox}

\paragraph{4. Combinational Bypass}

En düşük recovery latency için, dal execute edildiği çevrimde resolution çıkmalıdır:

\begin{lstlisting}[caption={Combinational bypass}]
// Same-cycle resolution using bypass
always_comb begin
    if (exec_valid_0 && (exec_rob_id_0 == head_branch_phys)) begin
        // Bypass: use incoming execute result directly
        branch_resolved_o_0 = 1'b1;
        branch_mispredicted_o_0 = exec_mispredicted_0;
        correct_pc_o_0 = exec_correct_pc_0;
    end else begin
        // Use stored value
        branch_resolved_o_0 = head_resolved;
        branch_mispredicted_o_0 = head_mispredicted;
        correct_pc_o_0 = head_correct_pc;
    end
end
\end{lstlisting}

\begin{nedenbox}
\textbf{Neden combinational bypass?}

Bypass olmadan akış:
\begin{enumerate}
    \item Cycle N: Execute sonucu gelir, BRAT'a yazılır
    \item Cycle N+1: BRAT'tan okunur, diğer modüllere iletilir
    \item Cycle N+2: Fetch yeni PC'den başlar
\end{enumerate}

Bypass ile:
\begin{enumerate}
    \item Cycle N: Execute sonucu gelir, AYNI ANDA çıkışa iletilir
    \item Cycle N+1: Fetch yeni PC'den başlar
\end{enumerate}

1 çevrim kazanç, yüksek misprediction oranlarında önemli performans farkı yaratır.
\end{nedenbox}

\paragraph{5. Commit Update}

ROB commit olduğunda, BRAT snapshot'ları güncellenir:

\begin{lstlisting}[caption={Commit update mantığı}]
// For each commit, update ALL snapshots that point to this ROB entry
for (int i = 0; i < BRAT_DEPTH; i++) begin
    if (commit_valid && snapshot[i][arch_addr] == rob_idx) begin
        snapshot[i][arch_addr] <= rf_mapping;  // Point to RF instead of ROB
    end
end
\end{lstlisting}

\begin{nedenbox}
\textbf{Neden commit update gerekli?}

Snapshot alındığında, bazı register'lar ROB'a işaret eder. ROB commit olduğunda:
\begin{itemize}
    \item Değer ROB'dan RF'e kopyalanır
    \item ROB entry yeniden kullanılabilir
\end{itemize}

Snapshot güncellenmezse, restore sırasında:
\begin{itemize}
    \item Geçersiz ROB pointer kullanılır
    \item Yanlış veri okunur
\end{itemize}

Bu yüzden commit, tüm snapshot'larda ilgili mapping'i RF'e günceller.
\end{nedenbox}

\paragraph{6. RAT Restore}

Misprediction tespit edildiğinde, RAT snapshot'tan restore edilir:

\begin{lstlisting}[caption={RAT restore}]
always_ff @(posedge clk) begin
    if (brat_restore_en) begin
        for (int i = 0; i < ARCH_REGS; i++) begin
            // Handle same-cycle commit
            if (commit_valid[0] && commit_addr_0 == i &&
                commit_rob_idx_0 == brat_restore_snapshot[i][4:0]) begin
                rat_table[i] <= {1'b0, commit_addr_0};
            end else begin
                rat_table[i] <= brat_restore_snapshot[i];
            end
        end
    end
end
\end{lstlisting}

\subsubsection{RAS Checkpoint/Restore}

BRAT, Return Address Stack için de checkpoint tutar:

\begin{lstlisting}[caption={RAS checkpoint}]
.push_ras_tos_0(push_ras_tos_i),
.ras_restore_valid_o(ras_restore_valid_o),
.ras_restore_tos_o(ras_restore_tos_o)
\end{lstlisting}

Misprediction durumunda RAS pointer da restore edilir, böylece fonksiyon
dönüş tahminleri doğru kalır.

\subsubsection{BRAT Özet Tablosu}

\begin{table}[H]
\centering
\caption{BRAT özellikleri}
\label{tab:brat_summary}
\begin{tabular}{lp{8cm}}
\toprule
\textbf{Özellik} & \textbf{Değer/Açıklama} \\
\midrule
Derinlik & 16 entry (maksimum in-flight branch) \\
Snapshot Boyutu & 32 × 6 bit = 192 bit/entry \\
Toplam Depolama & 16 × (~250 bit) ≈ 4 Kbit \\
Push Genişliği & 3-wide (her cycle 3 branch) \\
Resolution Genişliği & 3-wide (her cycle 3 resolution) \\
Recovery Latency & 0 cycle (combinational bypass) \\
\bottomrule
\end{tabular}
\end{table}


\subsection{RAT Özet Tablosu}

\begin{table}[H]
\centering
\caption{RAT özellikleri}
\label{tab:rat_summary}
\begin{tabular}{lp{8cm}}
\toprule
\textbf{Özellik} & \textbf{Değer/Açıklama} \\
\midrule
Mapping Genişliği & 32 arch → 64 phys \\
Rename Genişliği & 3-wide (her cycle 3 komut) \\
Lookup Genişliği & 6-wide (3×rs1 + 3×rs2) \\
Same-Cycle Forwarding & Destekleniyor \\
Misprediction Recovery & BRAT snapshot restore \\
Commit Update & RF mapping restore \\
\bottomrule
\end{tabular}
\end{table}


\section{Issue Stage Akışı}
\label{sec:issue_flow}

\subsection{Normal Operasyon}

\begin{enumerate}
    \item \textbf{Cycle N - Komut Alımı:}
    \begin{itemize}
        \item Instruction buffer'dan 3 komut okunur
        \item Decode valid sinyalleri kontrol edilir
    \end{itemize}

    \item \textbf{Cycle N - Decode (Kombinasyonel):}
    \begin{itemize}
        \item 3 decoder paralel çalışır
        \item Kontrol sinyalleri ve register adresleri çıkarılır
    \end{itemize}

    \item \textbf{Cycle N - Renaming (Kombinasyonel):}
    \begin{itemize}
        \item RAT lookup: rs1, rs2 için fiziksel register
        \item Allocation: rd için yeni fiziksel register
        \item Same-cycle forwarding: komutlar arası bağımlılık
    \end{itemize}

    \item \textbf{Cycle N - BRAT Push (Kombinasyonel):}
    \begin{itemize}
        \item Dal komutu ise RAT snapshot hazırla
        \item BRAT'a entry ekle
    \end{itemize}

    \item \textbf{Cycle N+1 - Pipeline Register:}
    \begin{itemize}
        \item Tüm bilgiler dispatch stage'e iletilir
        \item RAT tablosu güncellenir
    \end{itemize}
\end{enumerate}

\subsection{Misprediction Durumu}

\begin{enumerate}
    \item Execute stage'den misprediction sinyali gelir
    \item BRAT, oldest mispredicted branch'i tespit eder
    \item RAT, BRAT snapshot'tan restore edilir
    \item Free list pointer reset edilir
    \item Issue stage, \sig{internal\_flush} ile mevcut işlemleri iptal eder
\end{enumerate}

\begin{lstlisting}[caption={Internal flush sinyali}]
logic internal_flush;
assign internal_flush = |branch_mispredicted_o;
\end{lstlisting}

\section{Dispatch Interface}
\label{sec:dispatch_interface}

Issue stage, decode edilmiş ve rename edilmiş komutları dispatch stage'e iletir:

\begin{lstlisting}[caption={Issue-to-Dispatch interface}]
interface issue_to_dispatch_if;
    logic valid;
    logic [DATA_WIDTH-1:0] pc;
    logic [DATA_WIDTH-1:0] immediate;
    logic [25:0] control_word;
    logic [2:0] branch_sel;
    logic branch_prediction;

    // Renamed operands
    logic [PHYS_ADDR_WIDTH-1:0] rs1_phys;
    logic [PHYS_ADDR_WIDTH-1:0] rs2_phys;
    logic [PHYS_ADDR_WIDTH-1:0] rd_phys;  // = ROB ID
    logic [ARCH_ADDR_WIDTH-1:0] rd_arch;

    // LSQ allocation
    logic lsq_alloc_valid;
    logic [2:0] alloc_tag;
endinterface
\end{lstlisting}

\section{Issue Stage Özeti}

\begin{table}[H]
\centering
\caption{Issue stage özellikleri}
\label{tab:issue_summary}
\begin{tabular}{lp{8cm}}
\toprule
\textbf{Özellik} & \textbf{Değer/Açıklama} \\
\midrule
Issue Genişliği & 3 komut/çevrim \\
Decoder Sayısı & 3 (paralel) \\
RAT Boyutu & 32 entry × 6 bit \\
BRAT Derinliği & 16 entry \\
Free List & 32 entry circular buffer \\
LSQ Allocation & Ayrı circular buffer \\
Misprediction Latency & 0 cycle (combinational) \\
\bottomrule
\end{tabular}
\end{table}


% Bölüm 5: Dispatch Stage
% =============================================================================
% BÖLÜM 5: DISPATCH STAGE
% =============================================================================

\chapter{Dispatch Stage}
\label{chap:dispatch}

Dispatch stage, Tomasulo algoritmasının operand resolution ve instruction issue
mekanizmalarını içerir. Bu bölümde, Reservation Station (RS), Reorder Buffer (ROB),
Common Data Bus (CDB) ve Load Store Queue (LSQ) detaylı olarak açıklanmaktadır.

\section{Genel Bakış}
\label{sec:dispatch_genel}

Dispatch stage, issue stage'den gelen rename edilmiş komutları alır ve:
\begin{enumerate}
    \item Fiziksel register file/ROB'dan operand değerlerini okur
    \item Operandlar hazır değilse tag-based bekleme yapar
    \item Tüm operandlar hazır olduğunda functional unit'e gönderir
    \item CDB üzerinden sonuçları yayınlar ve ROB'u günceller
\end{enumerate}

\begin{figure}[H]
\centering
\fbox{
\begin{minipage}{0.95\textwidth}
\centering
\textbf{Dispatch Stage Veri Akışı}\\[1em]
\begin{tabular}{ccccc}
 & \fbox{Register File} & & & \\
\fbox{Issue Stage} $\rightarrow$ & $\downarrow$ & $\rightarrow$ \fbox{RS 0} $\rightarrow$ \fbox{ALU 0} & $\searrow$ & \\
 & \fbox{ROB} & $\rightarrow$ \fbox{RS 1} $\rightarrow$ \fbox{ALU 1} & $\rightarrow$ & \fbox{CDB} \\
 & & $\rightarrow$ \fbox{RS 2} $\rightarrow$ \fbox{ALU 2} & $\nearrow$ & $\downarrow$ \\
 & & & & \fbox{LSQ} \\
\end{tabular}
\\[0.5em]
\textit{CDB sonuçları RS'lere broadcast edilir (operand forwarding)}
\end{minipage}
}
\caption{Dispatch stage blok diyagramı}
\label{fig:dispatch_block}
\end{figure}

\subsection{Dispatch Stage Bileşenleri}

\begin{table}[H]
\centering
\caption{Dispatch stage modülleri}
\label{tab:dispatch_modules}
\begin{tabular}{llp{5cm}}
\toprule
\textbf{Modül} & \textbf{Dosya} & \textbf{Görev} \\
\midrule
Dispatch Stage & \texttt{dispatch\_stage.sv} & Üst seviye koordinasyon \\
Reservation Station & \texttt{reservation\_station.sv} & Operand bekleme ve issue (×3) \\
Reorder Buffer & \texttt{reorder\_buffer.sv} & In-order commit, spekülatif değer \\
Register File & \texttt{multi\_port\_register\_file.sv} & Commit edilmiş değerler \\
LSQ & \texttt{lsq\_simple\_top.sv} & Memory operasyonu yönetimi \\
\bottomrule
\end{tabular}
\end{table}

\section{Fiziksel Register Alanı}
\label{sec:phys_reg_space}

64 fiziksel register iki bölgeye ayrılır:

\begin{table}[H]
\centering
\caption{Fiziksel register alanı}
\label{tab:phys_reg_space}
\begin{tabular}{ccp{6cm}}
\toprule
\textbf{Adres Aralığı} & \textbf{MSB} & \textbf{Kaynak} \\
\midrule
0-31 & 0 & Register File (commit edilmiş) \\
32-63 & 1 & ROB (spekülatif, in-flight) \\
\bottomrule
\end{tabular}
\end{table}

\begin{lstlisting}[caption={Fiziksel adres decode}]
// MSB determines source: RF (0) or ROB (1)
assign inst_0_read_data_a = inst_0_read_addr_a[5] ?
    rob_0_read_data_a : reg_file_read_data_a_0;

// Tag from ROB (may not be ready), or ready if from RF
assign inst_0_read_tag_a = inst_0_read_addr_a[5] ?
    rob_0_read_tag_a : 3'b111;  // RF data always ready
\end{lstlisting}

\begin{nedenbox}
\textbf{Neden iki ayrı kaynak?}

Register File (RF), commit edilmiş değerleri tutar - garantili doğru.
ROB, henüz execute edilmemiş komutların sonuçlarını tutacaktır.

RAT, her register için ``en güncel'' kaynağı gösterir:
\begin{itemize}
    \item Değer commit edilmişse → RF'i gösterir (adres 0-31)
    \item Değer in-flight ise → ROB'u gösterir (adres 32-63)
\end{itemize}

Bu sayede hem spekülatif hem de kesinleşmiş değerler tek sistemde yönetilir.
\end{nedenbox}

\section{Reservation Station}
\label{sec:reservation_station}

Her issue slot için bir Reservation Station (RS) bulunur. RS, operandlar
hazır olana kadar komutu tutar ve CDB'yi izler.

\subsection{RS Yapısı}

\begin{table}[H]
\centering
\caption{Reservation Station entry alanları}
\label{tab:rs_entry}
\begin{tabular}{lcp{5cm}}
\toprule
\textbf{Alan} & \textbf{Boyut} & \textbf{Açıklama} \\
\midrule
\sig{occupied} & 1 bit & Entry dolu mu? \\
\sig{control\_signals} & 11 bit & ALU/memory kontrol \\
\sig{pc} & 32 bit & Komut adresi \\
\sig{rd\_phys\_addr} & 6 bit & Hedef fiziksel register \\
\sig{operand\_a\_data} & 32 bit & Operand A değeri veya tag \\
\sig{operand\_a\_tag} & 3 bit & Operand A producer tag \\
\sig{operand\_b\_data} & 32 bit & Operand B değeri veya tag \\
\sig{operand\_b\_tag} & 3 bit & Operand B producer tag \\
\sig{store\_data} & 32 bit & Store için rs2 değeri \\
\sig{branch\_prediction} & 1 bit & Tahmin edilen yön \\
\bottomrule
\end{tabular}
\end{table}

\subsection{Tag Sistemi}

Tomasulo'nun temel mekanizması olan tag sistemi, producer-consumer ilişkisini izler:

\begin{table}[H]
\centering
\caption{Tag değerleri ve anlamları}
\label{tab:tag_values}
\begin{tabular}{cl}
\toprule
\textbf{Tag} & \textbf{Anlam} \\
\midrule
3'b000 & ALU 0'dan bekliyor \\
3'b001 & ALU 1'den bekliyor \\
3'b010 & ALU 2'den bekliyor \\
3'b011 & LSQ'dan bekliyor (ROB ID ile eşleşme) \\
3'b111 & Hazır (değer geçerli) \\
\bottomrule
\end{tabular}
\end{table}

\begin{nedenbox}
\textbf{Neden tag tabanlı bekleme?}

Klasik yaklaşımda, bağımlı komut producer'ın bitmesini bekler (stall).
Tag sistemiyle:
\begin{itemize}
    \item Producer henüz execute edilmemiş olsa bile, RS dolu olabilir
    \item RS, CDB'yi izleyerek producer sonucunu yakalar
    \item Producer bittiği çevrimde consumer da issue edilebilir
\end{itemize}

Bu, ``pipeline forwarding''in out-of-order versiyonudur.
\end{nedenbox}

\subsection{CDB İzleme ve Operand Resolution}

RS, CDB'yi sürekli izler ve bekleyen operandları çözer:

\begin{lstlisting}[caption={CDB izleme mantığı}]
always_comb begin
    // Check if stored operand A can be resolved from CDB
    operand_a_valid_from_stored = occupied && (
        (stored_operand_a_tag == TAG_READY) ||
        (cdb_if_port.cdb_valid_0 && stored_operand_a_tag == 3'b000) ||
        (cdb_if_port.cdb_valid_1 && stored_operand_a_tag == 3'b001) ||
        (cdb_if_port.cdb_valid_2 && stored_operand_a_tag == 3'b010) ||
        // LSQ matching: tag=011 AND data matches ROB ID
        (cdb_if_port.cdb_valid_3_0 && stored_operand_a_tag == 3'b011 &&
            stored_operand_a_data == cdb_if_port.cdb_dest_reg_3_0)
    );
end
\end{lstlisting}

\begin{nedenbox}
\textbf{Neden LSQ için özel eşleştirme?}

ALU sonuçları tek bir CDB kanalından gelir (ALU başına 1).
Ancak LSQ'dan 3 ayrı sonuç gelebilir. Tag=011 sadece ``LSQ bekliyor''
demektir, hangi LSQ entry'sinden geleceğini belirtmez.

Bu yüzden LSQ için ek ROB ID eşleştirmesi yapılır:
\texttt{stored\_operand\_a\_data == cdb\_dest\_reg\_3\_x}
\end{nedenbox}

\subsection{Operand Seçimi}

Operandlar birden fazla kaynaktan gelebilir:

\begin{lstlisting}[caption={Operand veri seçimi}]
always_comb begin
    // Priority: 1) Stored ready, 2) CDB ALU0, 3) CDB ALU1, 4) CDB ALU2, 5) CDB LSQ
    if (stored_operand_a_tag == TAG_READY) begin
        final_operand_a_data = stored_operand_a_data;
    end else if (cdb_if_port.cdb_valid_0 && stored_operand_a_tag == 3'b000) begin
        final_operand_a_data = cdb_if_port.cdb_data_0;
    end else if (cdb_if_port.cdb_valid_1 && stored_operand_a_tag == 3'b001) begin
        final_operand_a_data = cdb_if_port.cdb_data_1;
    end else if (cdb_if_port.cdb_valid_2 && stored_operand_a_tag == 3'b010) begin
        final_operand_a_data = cdb_if_port.cdb_data_2;
    end else begin
        // LSQ data
        final_operand_a_data = cdb_if_port.cdb_data_3_x;
    end
end
\end{lstlisting}

\subsection{Issue Koşulu}

Komut ancak her iki operand da hazır olduğunda issue edilir:

\begin{lstlisting}[caption={Issue koşulu}]
assign should_issue = (occupied && operand_a_valid && operand_b_valid) ||
    (decode_if.dispatch_valid && operand_a_ready && operand_b_ready);
\end{lstlisting}

İki durum vardır:
\begin{enumerate}
    \item \textbf{Stored issue:} RS'de bekleyen komut, CDB'den operand aldı
    \item \textbf{Direct issue:} Yeni gelen komut, operandları zaten hazır
\end{enumerate}

\subsection{Eager Misprediction Flush}

RS, misprediction durumunda spekülatif komutları temizler:

\begin{lstlisting}[caption={RS flush mantığı}]
always_comb begin
    // Calculate ROB distance from head
    if (stored_rob_idx >= rob_head_ptr_i) begin
        stored_rob_distance = stored_rob_idx - rob_head_ptr_i;
    end else begin
        stored_rob_distance = 32 - rob_head_ptr_i + stored_rob_idx;
    end

    // Flush if: after mispredicted branch in program order
    should_flush_rs = occupied && eager_misprediction_i &&
        (stored_rob_distance > mispredicted_distance_i);
end
\end{lstlisting}

\begin{nedenbox}
\textbf{Neden distance karşılaştırması?}

ROB circular buffer olduğu için basit indeks karşılaştırması yetmez.
Mispredicted branch ROB[25]'te, RS'deki komut ROB[3]'te olabilir.

Distance hesabı:
\begin{itemize}
    \item Head = 20, Branch = 25 → Branch distance = 5
    \item Head = 20, RS = 3 → RS distance = 32 - 20 + 3 = 15
    \item 15 > 5 → RS komutu spekülatif, flush edilmeli
\end{itemize}
\end{nedenbox}

\section{Reorder Buffer (ROB)}
\label{sec:rob}

ROB, out-of-order execution ile in-order commit arasındaki köprüdür.
Spekülatif sonuçları tutar ve program sırasında commit eder.

\subsection{ROB Entry Yapısı}

\begin{lstlisting}[caption={ROB entry yapısı}]
typedef struct packed {
    logic [DATA_WIDTH-1:0] data;       // Result value
    logic [TAG_WIDTH-1:0] tag;         // Producer tag (111 = ready)
    logic [ADDR_WIDTH-1:0] addr;       // Architectural register address
    logic executed;                     // Execution completed?
    logic exception;                    // Misprediction/exception?
    logic is_branch;                    // Branch instruction?
    logic is_store;                     // Store instruction?
} rob_entry_t;
\end{lstlisting}

\subsection{ROB Portları}

\begin{table}[H]
\centering
\caption{ROB port sayıları}
\label{tab:rob_ports}
\begin{tabular}{lcp{5cm}}
\toprule
\textbf{Port Tipi} & \textbf{Sayı} & \textbf{Kullanım} \\
\midrule
Allocation & 3 & Issue stage'den gelen komutlar \\
Read & 6 & RS operand okuması (3×2) \\
CDB Write & 6 & ALU (3) + LSQ (3) sonuçları \\
Commit & 3 & RF'e yazma ve serbest bırakma \\
\bottomrule
\end{tabular}
\end{table}

\subsection{Allocation}

Issue stage'den gelen her komut için ROB entry allocate edilir:

\begin{lstlisting}[caption={ROB allocation}]
always_ff @(posedge clk) begin
    if (alloc_enable_0) begin
        buffer[alloc_idx_0].addr <= alloc_addr_0;
        buffer[alloc_idx_0].tag <= alloc_tag_0;  // Producer ALU tag
        buffer[alloc_idx_0].executed <= 1'b0;
        buffer[alloc_idx_0].exception <= 1'b0;
        buffer[alloc_idx_0].is_store <= alloc_is_store_0;
    end
end
\end{lstlisting}

\subsection{CDB Sonuç Yazma}

Execute tamamlandığında, sonuç ROB'a yazılır:

\begin{lstlisting}[caption={CDB sonuç yazma}]
always_ff @(posedge clk) begin
    if (cdb_valid_0) begin
        buffer[cdb_addr_0].data <= cdb_data_0;
        buffer[cdb_addr_0].tag <= 3'b111;  // Mark as ready
        buffer[cdb_addr_0].executed <= 1'b1;
        buffer[cdb_addr_0].exception <= cdb_exception_0;
        buffer[cdb_addr_0].is_branch <= cdb_is_branch_0;
    end
end
\end{lstlisting}

\subsection{In-Order Commit}

ROB head'den itibaren, sırasıyla en fazla 3 komut commit edilir:

\begin{lstlisting}[caption={Commit mantığı}]
// Commit valid only if: executed AND not exception
assign commit_valid_0 = buffer[head].executed && !flush_pending;
assign commit_valid_1 = commit_valid_0 && buffer[head+1].executed;
assign commit_valid_2 = commit_valid_1 && buffer[head+2].executed;

// Commit data and address
assign commit_data_0 = buffer[head].data;
assign commit_addr_0 = buffer[head].addr;
\end{lstlisting}

\begin{nedenbox}
\textbf{Neden in-order commit kritik?}

Out-of-order execution, komutların farklı sırada bitmesine izin verir.
Ancak:
\begin{itemize}
    \item Exception handling: Sadece ``önceki'' komutlar kesinleşmeli
    \item Misprediction: Spekülatif komutlar geri alınabilmeli
    \item Memory consistency: Store'lar program sırasında görünmeli
\end{itemize}

In-order commit, architectural state'in her zaman tutarlı olmasını sağlar.
\end{nedenbox}

\subsection{Store Permission}

Store komutları, ROB head'e ulaşana kadar memory'ye yazamaz:

\begin{lstlisting}[caption={Store permission}]
assign store_can_issue_0 = buffer[head].is_store && buffer[head].executed;
assign allowed_store_address_0 = head;
\end{lstlisting}

\section{Common Data Bus (CDB)}
\label{sec:cdb}

CDB, execution sonuçlarını tüm bekleme noktalarına yayınlar.

\subsection{CDB Kanalları}

\begin{table}[H]
\centering
\caption{CDB kanalları}
\label{tab:cdb_channels}
\begin{tabular}{lp{6cm}}
\toprule
\textbf{Kanal} & \textbf{Kaynak} \\
\midrule
CDB 0 & ALU 0 sonucu \\
CDB 1 & ALU 1 sonucu \\
CDB 2 & ALU 2 sonucu \\
CDB 3\_0 & LSQ port 0 (load/store) \\
CDB 3\_1 & LSQ port 1 \\
CDB 3\_2 & LSQ port 2 \\
\bottomrule
\end{tabular}
\end{table}

\subsection{CDB Sinyalleri}

Her CDB kanalı şu sinyalleri taşır:

\begin{lstlisting}[caption={CDB interface}]
interface cdb_if;
    // ALU channels
    logic cdb_valid_0, cdb_valid_1, cdb_valid_2;
    logic [DATA_WIDTH-1:0] cdb_data_0, cdb_data_1, cdb_data_2;
    logic [PHYS_ADDR_WIDTH-1:0] cdb_dest_reg_0, cdb_dest_reg_1, cdb_dest_reg_2;
    logic cdb_misprediction_0, cdb_misprediction_1, cdb_misprediction_2;
    logic cdb_is_branch_0, cdb_is_branch_1, cdb_is_branch_2;

    // LSQ channels
    logic cdb_valid_3_0, cdb_valid_3_1, cdb_valid_3_2;
    logic [DATA_WIDTH-1:0] cdb_data_3_0, cdb_data_3_1, cdb_data_3_2;
    logic [PHYS_ADDR_WIDTH-1:0] cdb_dest_reg_3_0, cdb_dest_reg_3_1, cdb_dest_reg_3_2;
endinterface
\end{lstlisting}

\section{Dispatch Stage Özeti}

\begin{table}[H]
\centering
\caption{Dispatch stage özellikleri}
\label{tab:dispatch_summary}
\begin{tabular}{lp{7cm}}
\toprule
\textbf{Özellik} & \textbf{Değer/Açıklama} \\
\midrule
Dispatch Genişliği & 3 komut/çevrim \\
RS Sayısı & 3 (her biri 1 entry) \\
ROB Derinliği & 32 entry \\
RF Boyutu & 32 × 32 bit \\
CDB Kanalları & 6 (3 ALU + 3 LSQ) \\
Commit Genişliği & 3 komut/çevrim \\
Tag Genişliği & 3 bit \\
\bottomrule
\end{tabular}
\end{table}


% Bölüm 6: Execute Stage
% =============================================================================
% BÖLÜM 6: EXECUTE STAGE
% =============================================================================

\chapter{Execute Stage}
\label{chap:execute}

Execute stage, komutların gerçek hesaplamalarının yapıldığı pipeline aşamasıdır.
Bu bölümde, 3 paralel functional unit, ALU operasyonları, dal çözümleme ve
misprediction tespiti detaylı olarak açıklanmaktadır.

\section{Genel Bakış}
\label{sec:execute_genel}

Execute stage, reservation station'lardan gelen komutları alır ve:
\begin{enumerate}
    \item ALU/shifter operasyonlarını gerçekleştirir
    \item Dal koşullarını değerlendirir
    \item Misprediction tespit eder
    \item Sonuçları CDB üzerinden yayınlar
\end{enumerate}

\begin{figure}[H]
\centering
\fbox{
\begin{minipage}{0.9\textwidth}
\centering
\textbf{Execute Stage Veri Akışı}\\[1em]
\begin{tabular}{ccccc}
\fbox{RS 0} & $\rightarrow$ & \fbox{FU 0 (ALU+Branch)} & $\rightarrow$ & \fbox{CDB 0} \\[0.3em]
\fbox{RS 1} & $\rightarrow$ & \fbox{FU 1 (ALU+Branch)} & $\rightarrow$ & \fbox{CDB 1} \\[0.3em]
\fbox{RS 2} & $\rightarrow$ & \fbox{FU 2 (ALU+Branch)} & $\rightarrow$ & \fbox{CDB 2} \\[0.5em]
 & & $\downarrow$ (branch results) & & \\
 & & \fbox{BRAT Update} & & \\
\end{tabular}
\end{minipage}
}
\caption{Execute stage blok diyagramı}
\label{fig:execute_block}
\end{figure}

\subsection{Execute Stage Bileşenleri}

\begin{table}[H]
\centering
\caption{Execute stage modülleri}
\label{tab:execute_modules}
\begin{tabular}{llp{5cm}}
\toprule
\textbf{Modül} & \textbf{Dosya} & \textbf{Görev} \\
\midrule
Execute Stage & \texttt{execute\_stage.sv} & Üst seviye koordinasyon \\
ALU & \texttt{alu.sv} & Aritmetik/mantık operasyonları \\
Arithmetic Unit & \texttt{arithmetic\_unit.sv} & ADD, SUB, SLT, SLTU \\
Logical Unit & \texttt{logical\_unit.sv} & XOR, OR, AND \\
Shifter & \texttt{shifter.sv} & SLL, SRL, SRA \\
Branch Unit & \texttt{functional\_unit.sv} & Dal koşulu değerlendirme \\
\bottomrule
\end{tabular}
\end{table}

\section{Functional Unit Yapısı}
\label{sec:fu_structure}

Her functional unit (FU), tam RV32I ALU ve dal işleme kapasitesine sahiptir:

\begin{lstlisting}[caption={FU sinyal tanımları}]
// Functional unit signals for FU0
logic [DATA_WIDTH-1:0] fu0_data_a, fu0_data_b;
logic [3:0] fu0_func_sel;
logic [DATA_WIDTH-1:0] fu0_result;
logic fu0_carry_out, fu0_overflow, fu0_negative, fu0_zero;
logic fu0_busy;

// Branch control signals
logic fu0_mpc, fu0_jalr;
logic fu0_misprediction;
logic [DATA_WIDTH-1:0] fu0_correct_pc;
\end{lstlisting}

\subsection{Operand Bağlantısı}

\begin{lstlisting}[caption={RS'den FU'ya operand aktarımı}]
// Operand assignment from RS to FU
assign fu0_data_a = rs_to_exec_0.data_a;
assign fu0_data_b = rs_to_exec_0.data_b;

// Function select from control signals [10:7]
assign fu0_func_sel = rs_to_exec_0.control_signals[10:7];
\end{lstlisting}

\section{ALU Operasyonları}
\label{sec:alu_ops}

ALU, aritmetik ve mantıksal operasyonları gerçekleştirir:

\begin{table}[H]
\centering
\caption{ALU fonksiyon seçimi}
\label{tab:alu_func}
\begin{tabular}{clp{5cm}}
\toprule
\textbf{func\_sel} & \textbf{Operasyon} & \textbf{Açıklama} \\
\midrule
3'b000 & ADD & Toplama \\
3'b001 & SUB & Çıkarma \\
3'b010 & SLT & Set Less Than (signed) \\
3'b011 & SLTU & Set Less Than (unsigned) \\
3'b100 & XOR & Bitwise XOR \\
3'b101 & OR & Bitwise OR \\
3'b110 & AND & Bitwise AND \\
3'b111 & Reserved & --- \\
\bottomrule
\end{tabular}
\end{table}

\subsection{ALU Yapısı}

ALU, aritmetik ve mantıksal birimlerden oluşur:

\begin{lstlisting}[caption={ALU iç yapısı}]
module alu #(parameter size = 32)(
    input logic [size-1:0] data_a,
    input logic [size-1:0] data_b,
    input logic [2:0] func_sel,
    output logic [size-1:0] data_result,
    output logic carry_out, overflow, zero, negative
);

    logic [size-1:0] arithmetic_out;
    logic [size-1:0] logical_out;

    arithmetic_unit arithmetic(
        .data_a(data_a), .data_b(data_b),
        .func_sel(func_sel[1:0]),  // ADD, SUB, SLT, SLTU
        .data_result(arithmetic_out),
        ...
    );

    logical_unit logical(
        .data_a(data_a), .data_b(data_b),
        .func_sel(func_sel[1:0]),  // XOR, OR, AND
        .data_result(logical_out)
    );

    // Select based on func_sel[2]
    assign data_result = func_sel[2] ? logical_out : arithmetic_out;
endmodule
\end{lstlisting}

\begin{nedenbox}
\textbf{Neden ayrı arithmetic ve logical unit?}

\begin{itemize}
    \item Paralel hesaplama: Her iki sonuç aynı anda hesaplanır
    \item Basit multiplexer: Sadece MSB ile seçim
    \item Timing: Critical path kısaltılır
    \item Modülerlik: Bağımsız test ve optimizasyon
\end{itemize}
\end{nedenbox}

\section{Dal Çözümleme}
\label{sec:branch_resolution}

Execute stage, dal komutlarının sonuçlarını hesaplar ve tahminle karşılaştırır.

\subsection{Branch Condition Evaluation}

\begin{table}[H]
\centering
\caption{Branch koşulları}
\label{tab:branch_conditions}
\begin{tabular}{clp{4cm}}
\toprule
\textbf{branch\_sel} & \textbf{Komut} & \textbf{Koşul} \\
\midrule
3'b000 & NO\_BRANCH & Dallanma yok \\
3'b001 & BEQ & rs1 == rs2 \\
3'b010 & BNE & rs1 != rs2 \\
3'b011 & BLT & rs1 < rs2 (signed) \\
3'b100 & BGE & rs1 >= rs2 (signed) \\
3'b101 & BLTU/BGEU & Unsigned karşılaştırma \\
3'b110 & JAL & Koşulsuz atlama \\
3'b111 & JALR & Register indirect atlama \\
\bottomrule
\end{tabular}
\end{table}

\subsection{Misprediction Tespiti}

\begin{lstlisting}[caption={Misprediction detection}]
// Branch misprediction: actual outcome != prediction
// mpc = branch condition result (taken/not-taken)
assign fu0_misprediction = fu0_jalr ?
    // JALR: target address mismatch
    (fu0_result != rs_to_exec_0.pc_value_at_prediction) :
    // Branch: direction mismatch
    (fu0_mpc ^ rs_to_exec_0.branch_prediction);
\end{lstlisting}

\begin{nedenbox}
\textbf{Neden JALR için özel kontrol?}

Branch komutları için sadece yön (taken/not-taken) tahmini yapılır.
JALR için hem yön hem de hedef adres tahmini yapılır.

JALR misprediction:
\begin{itemize}
    \item Tahmin edilen hedef: \sig{pc\_value\_at\_prediction}
    \item Gerçek hedef: \sig{fu0\_result} (rs1 + imm)
    \item İkisi farklıysa → misprediction
\end{itemize}
\end{nedenbox}

\subsection{Correct PC Hesaplama}

\begin{lstlisting}[caption={Correct PC calculation}]
// Correct PC for recovery
assign fu0_correct_pc = fu0_jalr ?
    // JALR: use calculated address (align to 4)
    {fu0_result[31:2], 2'b00} :
    // Branch: use current PC (next instruction)
    {rs_to_exec_0.pc[31:2], 2'b00};
\end{lstlisting}

\section{Sonuç Seçimi}
\label{sec:result_select}

Execute stage, farklı komut tipleri için farklı sonuçlar üretir:

\begin{lstlisting}[caption={Result selection}]
// Result selection based on instruction type
// control_signals[5] = save PC (JAL/JALR)
assign fu0_corrected_result = rs_to_exec_0.control_signals[5] ?
    {rs_to_exec_0.pc[31:2], 2'b00} :  // Link address
    fu0_result;                        // ALU result

// Final result: branch returns prediction PC, others return ALU
assign rs_to_exec_0.data_result = rs_to_exec_0.is_branch ?
    rs_to_exec_0.pc_value_at_prediction :
    fu0_corrected_result;
\end{lstlisting}

\begin{table}[H]
\centering
\caption{Komut tipine göre sonuç}
\label{tab:result_by_type}
\begin{tabular}{lp{6cm}}
\toprule
\textbf{Komut Tipi} & \textbf{Sonuç} \\
\midrule
ALU (ADD, SUB, ...) & ALU hesaplama sonucu \\
Load/Store & Adres hesaplama sonucu \\
JAL/JALR & Link adresi (PC + 4) \\
Branch (BEQ, ...) & Tahmin edilen PC \\
\bottomrule
\end{tabular}
\end{table}

\section{Memory Address Calculation}
\label{sec:mem_addr_calc}

Load/Store komutları için adres hesaplaması:

\begin{lstlisting}[caption={Memory address calculation flag}]
// Memory address calculation detection
// control_signals[4] = load, control_signals[3] = memory op
assign rs_to_exec_0.mem_addr_calculation =
    rs_to_exec_0.control_signals[4] ||
    (rs_to_exec_0.control_signals[3] && !rs_to_exec_0.control_signals[6]);
\end{lstlisting}

Bu flag, LSQ'ya adres hesaplamasının tamamlandığını bildirir.

\section{Branch Predictor Update}
\label{sec:predictor_update}

Execute stage, dal sonuçlarını branch predictor'a geri bildirir:

\begin{lstlisting}[caption={Predictor update sinyalleri}]
// Update predictor only for conditional branches
assign update_predictor_0 = rs_to_exec_0.issue_valid &&
    (rs_to_exec_0.branch_sel > 0 && rs_to_exec_0.branch_sel < 6);

// Misprediction output for BRAT
assign misprediction_0 = rs_to_exec_0.issue_valid ? fu0_misprediction : 0;

// Correct PC for recovery
assign correct_pc_0 = fu0_correct_pc;

// PC for predictor table update
assign update_pc_0 = rs_to_exec_0.data_result;

// Physical register (ROB ID) for BRAT matching
assign phys_reg_branch_0 = rs_to_exec_0.rd_phys_addr;
\end{lstlisting}

\section{JALR Handling}
\label{sec:jalr_handling}

JALR, özel işlem gerektirir çünkü hem hedef adres hem de dönüş adresi hesaplanır:

\begin{lstlisting}[caption={JALR detection}]
// JALR detection for special handling
assign is_jalr_0 = rs_to_exec_0.issue_valid && fu0_jalr;
\end{lstlisting}

JALR için:
\begin{itemize}
    \item Hedef adres: rs1 + immediate
    \item Dönüş adresi: PC + 4 (rd'ye yazılır)
    \item Predictor: JALR predictor ayrı güncellenir
\end{itemize}

\section{CDB Broadcast}
\label{sec:cdb_broadcast}

Her FU, sonuçlarını CDB üzerinden yayınlar:

\begin{lstlisting}[caption={CDB output assignment}]
// RS to exec interface carries CDB signals
assign rs_to_exec_0.data_result = fu0_corrected_result;
assign rs_to_exec_0.misprediction = fu0_misprediction;
assign rs_to_exec_0.is_branch = rs_to_exec_0.branch_sel > 0 &&
    rs_to_exec_0.branch_sel < 6;
assign rs_to_exec_0.correct_pc = fu0_correct_pc;
\end{lstlisting}

\section{Execute Stage Özeti}

\begin{table}[H]
\centering
\caption{Execute stage özellikleri}
\label{tab:execute_summary}
\begin{tabular}{lp{7cm}}
\toprule
\textbf{Özellik} & \textbf{Değer/Açıklama} \\
\midrule
FU Sayısı & 3 (paralel) \\
ALU Operasyonları & 7 (ADD, SUB, SLT, SLTU, XOR, OR, AND) \\
Shifter Operasyonları & 3 (SLL, SRL, SRA) \\
Branch Koşulları & 6 (BEQ, BNE, BLT, BGE, BLTU, BGEU) \\
Execute Latency & 1 cycle (combinational ALU) \\
Misprediction Detection & Same-cycle \\
CDB Channels & 3 (FU başına 1) \\
\bottomrule
\end{tabular}
\end{table}


% Bölüm 7: Memory Stage
% =============================================================================
% BÖLÜM 7: MEMORY STAGE - LOAD STORE QUEUE
% =============================================================================

\chapter{Memory Stage ve Load Store Queue}
\label{chap:memory}

Bu bölümde, memory operasyonlarının yönetildiği Load Store Queue (LSQ) yapısı
ve memory subsystem ile etkileşimi detaylı olarak açıklanmaktadır.

\section{Genel Bakış}
\label{sec:memory_genel}

Memory stage, load ve store komutlarının memory'ye erişimini yönetir. Out-of-order
superscalar işlemcilerde memory operasyonları özel dikkat gerektirir:

\begin{itemize}
    \item \textbf{Memory consistency:} Store'lar program sırasında görünmeli
    \item \textbf{Load-store dependency:} Load, önceki store'a bağımlı olabilir
    \item \textbf{Spekülatif execution:} Misprediction durumunda store'lar geri alınmalı
\end{itemize}

\begin{figure}[H]
\centering
\fbox{
\begin{minipage}{0.9\textwidth}
\centering
\textbf{Memory Stage Veri Akışı}\\[1em]
\begin{tabular}{ccccc}
\fbox{Issue Stage} & $\xrightarrow{\text{Allocate}}$ & & & \\
 & & \fbox{LSQ (3 Head)} & $\xleftrightarrow{\text{3 Port}}$ & \fbox{Memory System} \\
\fbox{Execute (Addr)} & $\xrightarrow{\text{CDB}}$ & $\uparrow$ $\downarrow$ & & \\
 & & \fbox{ROB (Commit)} & & \\
 & $\xleftarrow{\text{Load Data + Fwd}}$ & & & \\
\fbox{CDB (3 port)} & & & & \\
\end{tabular}
\end{minipage}
}
\caption{Memory stage blok diyagramı}
\label{fig:memory_block}
\end{figure}

\section{LSQ Tasarım Felsefesi}
\label{sec:lsq_philosophy}

Bu tasarımda, yüksek throughput ve doğru memory ordering bir arada sağlanmaktadır:

\begin{nedenbox}
\textbf{LSQ Temel Özellikleri}

Bu LSQ implementasyonu aşağıdaki gelişmiş özellikleri içerir:
\begin{itemize}
    \item \textbf{3 Paralel Memory Port:} Her çevrimde 3 load/store issue edilebilir
    \item \textbf{3 Bağımsız Head Pointer:} Sliding window ile paralel operasyon
    \item \textbf{Store-to-Load Forwarding:} Age-based forwarding ile memory bypass
    \item \textbf{CDB Snooping:} Address ve data dependency çözümleme
    \item \textbf{Eager Misprediction Flush:} ROB distance tabanlı anında temizleme
    \item \textbf{ROB Koordinasyonlu Store Commit:} Spekülatif store koruması
\end{itemize}
\end{nedenbox}

\section{LSQ Entry Yapısı}
\label{sec:lsq_entry}

\begin{lstlisting}[caption={LSQ entry yapısı}]
typedef struct packed {
    logic                       valid;
    logic                       is_store;
    logic [PHYS_REG_WIDTH-1:0]  phys_reg;    // ROB ID

    // Address
    logic                       addr_valid;
    logic [DATA_WIDTH-1:0]      address;
    logic [TAG_WIDTH-1:0]       addr_tag;

    // Data (for stores)
    logic                       data_valid;
    logic [DATA_WIDTH-1:0]      data;
    logic [TAG_WIDTH-1:0]       data_tag;

    // Operation attributes
    mem_size_t                  size;        // Byte, Half, Word
    logic                       sign_extend;

    // Execution state
    logic                       mem_issued;   // Sent to memory
    logic                       mem_complete; // Memory responded
} lsq_simple_entry_t;
\end{lstlisting}

\begin{table}[H]
\centering
\caption{LSQ entry alanları}
\label{tab:lsq_entry}
\begin{tabular}{lcp{5cm}}
\toprule
\textbf{Alan} & \textbf{Boyut} & \textbf{Açıklama} \\
\midrule
\sig{valid} & 1 bit & Entry geçerli mi? \\
\sig{is\_store} & 1 bit & Store mu load mu? \\
\sig{phys\_reg} & 6 bit & Hedef ROB ID \\
\sig{addr\_valid} & 1 bit & Adres hesaplandı mı? \\
\sig{address} & 32 bit & Memory adresi \\
\sig{data\_valid} & 1 bit & Store verisi hazır mı? \\
\sig{data} & 32 bit & Store verisi \\
\sig{size} & 2 bit & Byte/Half/Word \\
\sig{sign\_extend} & 1 bit & Sign extension? \\
\sig{mem\_issued} & 1 bit & Memory'ye gönderildi mi? \\
\sig{mem\_complete} & 1 bit & Memory yanıt verdi mi? \\
\bottomrule
\end{tabular}
\end{table}

\section{3-Head Pointer Mimarisi}
\label{sec:lsq_3head}

LSQ, 3 bağımsız head pointer ile çalışır. Bu, her çevrimde 3 memory operasyonunun
paralel olarak issue edilmesini sağlar:

\begin{lstlisting}[caption={LSQ pointer yapısı}]
logic [LSQ_ADDR_WIDTH:0] head_ptr;    // Head 0 - oldest tracked
logic [LSQ_ADDR_WIDTH:0] head_ptr_1;  // Head 1
logic [LSQ_ADDR_WIDTH:0] head_ptr_2;  // Head 2
logic [LSQ_ADDR_WIDTH:0] tail_ptr;    // Next free entry

// Age distance calculation (for ordering)
assign distance_0 = (tail_plus_3 - head_ptr);
assign distance_1 = (tail_plus_3 - head_ptr_1);
assign distance_2 = (tail_plus_3 - head_ptr_2);
\end{lstlisting}

\begin{nedenbox}
\textbf{Neden 3 bağımsız head pointer?}

Geleneksel LSQ'larda tek head pointer vardır ve operasyonlar sırayla issue edilir.
Bu tasarımda:
\begin{itemize}
    \item 3 head pointer, 3 farklı entry'yi aynı anda izler
    \item Her head bağımsız olarak memory'ye issue edilebilir
    \item Deallocation sonrası head'ler ``newest + 1'' konumuna kayar
    \item Age-based ordering ile doğru sıralama korunur
\end{itemize}

Bu yaklaşım, 3-way superscalar pipeline ile uyumlu memory throughput sağlar.
\end{nedenbox}

\subsection{Head Pointer Sliding Window}

Deallocate edilen head'ler, en yeni (newest) pointer'ın bir sonrasına atanır:

\begin{lstlisting}[caption={Head pointer update mantığı}]
// Find newest head (closest to tail)
if ((age_dist_0 <= age_dist_1) && (age_dist_0 <= age_dist_2))
    newest_ptr_eff = head_ptr_eff_0;
else if (age_dist_1 <= age_dist_2)
    newest_ptr_eff = head_ptr_eff_1;
else
    newest_ptr_eff = head_ptr_eff_2;

// Refill deallocated slots after newest
if (effective_dealloc_0)
    head_ptr_n = newest_ptr_eff + 1;
if (effective_dealloc_1)
    head_ptr_1_n = newest_ptr_eff + 1 + effective_dealloc_0;
if (effective_dealloc_2)
    head_ptr_2_n = newest_ptr_eff + 1 + effective_dealloc_0 + effective_dealloc_1;
\end{lstlisting}

\section{LSQ Operasyonları}
\label{sec:lsq_ops}

\subsection{Allocation}

Issue stage'den gelen load/store komutları için entry allocate edilir:

\begin{lstlisting}[caption={LSQ allocation}]
always_ff @(posedge clk) begin
    if (alloc_valid_0_i && alloc_ready_o) begin
        lsq_buffer[alloc_0_ptr].valid        <= 1'b1;
        lsq_buffer[alloc_0_ptr].is_store     <= alloc_is_store_0_i;
        lsq_buffer[alloc_0_ptr].phys_reg     <= alloc_phys_reg_0_i;
        lsq_buffer[alloc_0_ptr].addr_valid   <= 1'b0;  // Wait for execute
        lsq_buffer[alloc_0_ptr].data_tag     <= alloc_data_tag_0_i;
        lsq_buffer[alloc_0_ptr].data         <= alloc_data_operand_0_i;
        lsq_buffer[alloc_0_ptr].data_valid   <= (alloc_data_tag_0_i == TAG_READY);
        lsq_buffer[alloc_0_ptr].size         <= mem_size_t'(alloc_size_0_i);
        lsq_buffer[alloc_0_ptr].sign_extend  <= alloc_sign_extend_0_i;
        lsq_buffer[alloc_0_ptr].mem_issued   <= 1'b0;
        lsq_buffer[alloc_0_ptr].mem_complete <= 1'b0;
    end
end
\end{lstlisting}

\subsection{Address Update (CDB'den)}

Execute stage adres hesapladığında, CDB üzerinden LSQ'ya bildirilir:

\begin{lstlisting}[caption={CDB address update}]
// CDB monitoring for address update
always_ff @(posedge clk) begin
    for (int i = 0; i < LSQ_DEPTH; i++) begin
        if (lsq_buffer[i].valid && !lsq_buffer[i].addr_valid) begin
            // Check CDB for address calculation result
            if (cdb_valid_0 && cdb_mem_addr_calc_0 &&
                lsq_buffer[i].phys_reg == cdb_dest_reg_0) begin
                lsq_buffer[i].addr_valid <= 1'b1;
                lsq_buffer[i].address <= cdb_data_0;
            end
            // Similar for CDB 1 and 2...
        end
    end
end
\end{lstlisting}

\subsection{Store Data Update}

Store için rs2 değeri hazır değilse, CDB'den beklenir:

\begin{lstlisting}[caption={Store data CDB update}]
// CDB monitoring for store data
always_ff @(posedge clk) begin
    for (int i = 0; i < LSQ_DEPTH; i++) begin
        if (lsq_buffer[i].valid && lsq_buffer[i].is_store &&
            !lsq_buffer[i].data_valid) begin
            // Match store data tag with CDB
            if (cdb_valid_0 && lsq_buffer[i].data_tag == 3'b000) begin
                lsq_buffer[i].data_valid <= 1'b1;
                lsq_buffer[i].data <= cdb_data_0;
            end
            // Similar for other CDB channels...
        end
    end
end
\end{lstlisting}

\subsection{Memory Issue}

Head'deki operasyon hazır olduğunda memory'ye gönderilir:

\begin{lstlisting}[caption={Memory issue logic}]
// Issue from head when ready
wire head_ready = lsq_buffer[head_idx].valid &&
    lsq_buffer[head_idx].addr_valid &&
    (lsq_buffer[head_idx].is_store ? lsq_buffer[head_idx].data_valid : 1'b1);

// For stores: also need ROB permission
wire store_permitted = !lsq_buffer[head_idx].is_store ||
    (store_can_issue_0 && allowed_store_address_0 == lsq_buffer[head_idx].phys_reg);

assign mem_0_req_valid_o = head_ready && store_permitted &&
    !lsq_buffer[head_idx].mem_issued;
\end{lstlisting}

\begin{nedenbox}
\textbf{Neden store için ROB permission gerekli?}

Store'lar spekülatif execute edilemez - memory'ye yazıldıktan sonra geri alınamaz.
Bu yüzden:
\begin{itemize}
    \item Store, ROB head'e ulaşmalı
    \item Commit kesinleşmeli
    \item Ancak o zaman memory'ye yazılabilir
\end{itemize}

\sig{store\_can\_issue} sinyali, ROB'dan gelir ve ilgili store'un commit
edilebileceğini gösterir.
\end{nedenbox}

\subsection{Memory Response}

Memory yanıt verdiğinde:

\begin{lstlisting}[caption={Memory response handling}]
always_ff @(posedge clk) begin
    if (mem_0_resp_valid_i) begin
        lsq_buffer[head_idx].mem_complete <= 1'b1;

        // For loads: capture data
        if (!lsq_buffer[head_idx].is_store) begin
            load_0_data <= process_load_data(
                mem_0_resp_data_i,
                lsq_buffer[head_idx].address[1:0],
                lsq_buffer[head_idx].size,
                lsq_buffer[head_idx].sign_extend
            );
        end
    end
end
\end{lstlisting}

\subsection{CDB Broadcast (Load)}

Load tamamlandığında, sonuç CDB'ye yayınlanır:

\begin{lstlisting}[caption={Load result CDB broadcast}]
// CDB output for load results
assign cdb_interface.cdb_valid_3_0 = mem_0_resp_valid_i &&
    !lsq_buffer[head_idx].is_store;
assign cdb_interface.cdb_data_3_0 = load_0_data;
assign cdb_interface.cdb_dest_reg_3_0 = {1'b1, lsq_buffer[head_idx].phys_reg[4:0]};
\end{lstlisting}

\section{Store Commit}
\label{sec:store_commit}

Store operasyonları özel commit akışına sahiptir:

\begin{enumerate}
    \item ROB head'e ulaşır
    \item \sig{store\_can\_issue} sinyali aktif olur
    \item LSQ, store'u memory'ye gönderir
    \item Memory yazma tamamlanır
    \item LSQ entry deallocate edilir
\end{enumerate}

\begin{lstlisting}[caption={Store permission from ROB}]
// ROB signals store can be committed
input  logic store_can_issue_0,
input  logic [PHYS_REG_WIDTH-1:0] allowed_store_address_0,

// Only issue store if ROB permits
wire can_issue_store_0 = store_can_issue_0 &&
    (allowed_store_address_0 == lsq_buffer[head_idx].phys_reg[4:0]);
\end{lstlisting}

\section{Eager Misprediction Flush}
\label{sec:lsq_flush}

Misprediction durumunda spekülatif load/store entry'leri temizlenir:

\begin{lstlisting}[caption={LSQ flush logic}]
// Calculate distance of each entry from ROB head
always_comb begin
    for (int i = 0; i < LSQ_DEPTH; i++) begin
        if (lsq_buffer[i].phys_reg[4:0] >= rob_head_ptr_i) begin
            entry_distance[i] = lsq_buffer[i].phys_reg[4:0] - rob_head_ptr_i;
        end else begin
            entry_distance[i] = 32 - rob_head_ptr_i + lsq_buffer[i].phys_reg[4:0];
        end

        // Flush if after mispredicted branch
        should_flush[i] = lsq_buffer[i].valid && eager_misprediction_i &&
            (entry_distance[i] > mispredicted_distance_i);
    end
end

// Apply flush
always_ff @(posedge clk) begin
    for (int i = 0; i < LSQ_DEPTH; i++) begin
        if (should_flush[i]) begin
            lsq_buffer[i].valid <= 1'b0;
        end
    end
end
\end{lstlisting}

\section{Memory Interface}
\label{sec:mem_interface}

LSQ, 3 memory port'u üzerinden memory'ye erişir:

\begin{table}[H]
\centering
\caption{Memory interface sinyalleri}
\label{tab:mem_interface}
\begin{tabular}{lcp{5cm}}
\toprule
\textbf{Sinyal} & \textbf{Yön} & \textbf{Açıklama} \\
\midrule
\sig{mem\_req\_valid\_o} & Out & İstek geçerli \\
\sig{mem\_req\_is\_store\_o} & Out & Store mu? \\
\sig{mem\_req\_addr\_o} & Out & Memory adresi \\
\sig{mem\_req\_data\_o} & Out & Store verisi \\
\sig{mem\_req\_be\_o} & Out & Byte enable \\
\sig{mem\_req\_ready\_i} & In & Memory hazır \\
\sig{mem\_resp\_valid\_i} & In & Yanıt geçerli \\
\sig{mem\_resp\_data\_i} & In & Load verisi \\
\bottomrule
\end{tabular}
\end{table}

\subsection{Byte Enable Hesaplama}

\begin{lstlisting}[caption={Byte enable calculation}]
always_comb begin
    case (lsq_buffer[head_idx].size)
        MEM_BYTE: mem_0_req_be_o = 4'b0001 << address[1:0];
        MEM_HALF: mem_0_req_be_o = 4'b0011 << address[1:0];
        MEM_WORD: mem_0_req_be_o = 4'b1111;
        default:  mem_0_req_be_o = 4'b1111;
    endcase
end
\end{lstlisting}

\subsection{Load Data İşleme}

Load verisi, boyut ve sign extension'a göre işlenir:

\begin{lstlisting}[caption={Load data processing}]
function automatic [DATA_WIDTH-1:0] process_load_data(
    input [DATA_WIDTH-1:0] mem_data,
    input [1:0] byte_offset,
    input mem_size_t size,
    input sign_extend
);
    logic [7:0] byte_val;
    logic [15:0] half_val;

    case (size)
        MEM_BYTE: begin
            byte_val = mem_data >> (byte_offset * 8);
            return sign_extend ?
                {{24{byte_val[7]}}, byte_val} :
                {24'b0, byte_val};
        end
        MEM_HALF: begin
            half_val = mem_data >> (byte_offset * 8);
            return sign_extend ?
                {{16{half_val[15]}}, half_val} :
                {16'b0, half_val};
        end
        MEM_WORD: return mem_data;
        default:  return mem_data;
    endcase
endfunction
\end{lstlisting}

\section{Store-to-Load Forwarding}
\label{sec:stl_forward}

LSQ, memory'ye gitmeden store verisini load'a iletebilir. Bu, memory latency'sini
bypass ederek performansı önemli ölçüde artırır.

\subsection{Forwarding Koşulları}

Forwarding için aşağıdaki koşulların sağlanması gerekir:

\begin{enumerate}
    \item Load, store'dan program sırasında \textbf{sonra} gelmeli (newer)
    \item Store'un adresi ve verisi \textbf{hazır} olmalı
    \item Adresler \textbf{eşleşmeli}
    \item Store boyutu, load boyutuna \textbf{eşit veya büyük} olmalı
\end{enumerate}

\begin{lstlisting}[caption={Forwarding koşul kontrolü}]
// Age comparison: head_0 newer than head_1?
head_0_newer_than_head_1 = (distance_0 < distance_1);

// Address match
head_0_head_1_addr_match = lsq_buffer[head_idx].addr_valid &&
    lsq_buffer[head_idx_1].addr_valid &&
    (lsq_buffer[head_idx].address == lsq_buffer[head_idx_1].address);

// Size comparison: store size >= load size
head_1_size_ge_head_0 = (lsq_buffer[head_idx_1].size >= lsq_buffer[head_idx].size);

// Forwarding decision
if (head_0_newer_than_head_1 && lsq_buffer[head_idx_1].is_store &&
    head_0_head_1_addr_match && head_1_size_ge_head_0) begin
    fwd_head_0 = 1'b1;
    head_0_fwd_source = 2'b01;  // Forward from head_1
end
\end{lstlisting}

\begin{nedenbox}
\textbf{Neden size kontrolü gerekli?}

\begin{itemize}
    \item SW (32-bit store) $\rightarrow$ LB (8-bit load): Forwarding mümkün
    \item SB (8-bit store) $\rightarrow$ LW (32-bit load): Forwarding \textbf{mümkün değil}
\end{itemize}

Store, load'un ihtiyaç duyduğu tüm byte'ları içermelidir. Aksi halde load
memory'den okumalıdır.
\end{nedenbox}

\subsection{Wait Koşulları}

Forwarding yapılamıyorsa ancak potansiyel bir bağımlılık varsa, load beklemeli:

\begin{lstlisting}[caption={Load wait mantığı}]
// Load must wait if:
// 1. Older store's data not ready, OR
// 2. Address match but size insufficient and store not yet issued
head_0_should_wait = !head_1_valids ||
    (head_1_valids && head_0_head_1_addr_match &&
     !head_1_size_ge_head_0 && !lsq_buffer[head_idx_1].mem_issued);
\end{lstlisting}

\subsection{Forwarding Data Path}

Forwarding aktif olduğunda, load verisi store entry'sinden alınır:

\begin{lstlisting}[caption={Forwarding veri yolu}]
// Select data source based on forwarding
assign load_0_src_data = fwd_head_0 ?
    (head_0_fwd_source == 2) ? lsq_buffer[head_idx_2].data :
                               lsq_buffer[head_idx_1].data :
    mem_0_resp_data_i;

// Data organizer applies size/sign extension
data_organizer load_0_data_organizer (
    .data_in(load_0_src_data),
    .Type_sel(mem_0_type_sel),  // {sign_extend, size}
    .data_out(load_0_data)
);
\end{lstlisting}

\section{Single Pipe Mode}
\label{sec:single_pipe}

LSQ, tek port modunda da çalışabilir:

\begin{lstlisting}[caption={Single pipe mode}]
input logic single_pipe_mode_i,

// In single pipe mode, only use port 0
assign effective_dealloc_1 = ... && !single_pipe_mode_i;
assign effective_dealloc_2 = ... && !single_pipe_mode_i;
assign cdb_interface.cdb_valid_3_1 = ... && !single_pipe_mode_i;
assign cdb_interface.cdb_valid_3_2 = ... && !single_pipe_mode_i;
\end{lstlisting}

\begin{nedenbox}
\textbf{Neden single pipe mode?}

\begin{itemize}
    \item Tek portlu memory sistemleri için uyumluluk
    \item Debug için basitleştirilmiş operasyon
    \item Performance comparison (1-pipe vs 3-pipe benchmark)
\end{itemize}
\end{nedenbox}

\section{LSQ Özeti}

\begin{table}[H]
\centering
\caption{LSQ özellikleri}
\label{tab:lsq_summary2}
\begin{tabular}{lp{7cm}}
\toprule
\textbf{Özellik} & \textbf{Değer/Açıklama} \\
\midrule
Buffer Derinliği & 32 entry \\
Allocation Genişliği & 3 entry/çevrim \\
Memory Port Sayısı & 3 paralel port (single pipe mode desteği) \\
Head Pointer Sayısı & 3 bağımsız pointer \\
Store-to-Load Forward & Age-based, size-aware forwarding \\
Flush Mechanism & Eager (ROB distance tabanlı) \\
CDB Integration & 3 load result port (cdb\_valid\_3\_0/1/2) \\
Desteklenen Boyutlar & Byte, Half, Word \\
Sign Extension & Destekleniyor \\
\bottomrule
\end{tabular}
\end{table}


% Bölüm 8: Performans Analizi
% =============================================================================
% BÖLÜM 8: PERFORMANS ANALİZİ
% =============================================================================

\chapter{Performans Analizi}
\label{chap:performans}

Bu bölümde, 3-way superscalar işlemcinin performans karakteristikleri,
darboğazlar ve optimizasyon stratejileri analiz edilmektedir.

\section{Performans Metrikleri}
\label{sec:metrics}

\subsection{Instructions Per Cycle (IPC)}

IPC, işlemci verimliliğinin temel ölçüsüdür:

\begin{equation}
IPC = \frac{\text{Toplam Komut Sayısı}}{\text{Toplam Çevrim Sayısı}}
\end{equation}

\begin{table}[H]
\centering
\caption{Teorik vs gerçek IPC}
\label{tab:ipc}
\begin{tabular}{lcc}
\toprule
\textbf{Konfigürasyon} & \textbf{Teorik Maks} & \textbf{Tipik} \\
\midrule
Scalar (1-way) & 1.0 & 0.7-0.9 \\
3-way Superscalar & 3.0 & 1.5-2.0 \\
\bottomrule
\end{tabular}
\end{table}

\subsection{Speedup}

3-way superscalar'ın scalar'a göre hızlanması:

\begin{equation}
Speedup = \frac{IPC_{superscalar}}{IPC_{scalar}} = \frac{IPC_{3-way}}{IPC_{1-way}}
\end{equation}

Test sonuçlarına göre tipik speedup: \textbf{1.83x - 1.93x}

\begin{nedenbox}
\textbf{Neden teorik 3x'e ulaşılamıyor?}

\begin{itemize}
    \item \textbf{RAW hazards:} Gerçek veri bağımlılıkları parallelliği sınırlar
    \item \textbf{Control hazards:} Branch misprediction'lar pipeline'ı boşaltır
    \item \textbf{Memory latency:} Load/store sıralı çalışır
    \item \textbf{Resource conflicts:} Sınırlı functional unit sayısı
\end{itemize}
\end{nedenbox}

\section{Pipeline Stall Analizi}
\label{sec:stall_analysis}

\subsection{Stall Kaynakları}

\begin{table}[H]
\centering
\caption{Stall kaynakları ve etkileri}
\label{tab:stall_sources}
\begin{tabular}{lp{5cm}c}
\toprule
\textbf{Kaynak} & \textbf{Açıklama} & \textbf{Tipik Etki} \\
\midrule
ROB Full & ROB doldu, yeni komut kabul edilemiyor & 5-10\% \\
RS Full & Reservation station dolu & 3-5\% \\
LSQ Full & Load/store queue dolu & 2-4\% \\
BRAT Full & Maksimum in-flight branch sayısına ulaşıldı & 1-3\% \\
Memory Latency & Memory yanıt bekleniyor & 10-20\% \\
Branch Misprediction & Yanlış yol flush ediliyor & 5-15\% \\
\bottomrule
\end{tabular}
\end{table}

\subsection{Occupancy Analizi}

Pipeline occupancy, kaynakların ne kadar verimli kullanıldığını gösterir:

\begin{lstlisting}[caption={Occupancy hesaplama}]
// ROB Occupancy
ROB_occupancy = (tail_ptr - head_ptr) / ROB_DEPTH;

// RS Occupancy
RS_occupancy = occupied_entries / TOTAL_RS_ENTRIES;

// Average issue width
avg_issue_width = committed_instructions / total_cycles;
\end{lstlisting}

\section{Branch Prediction Performansı}
\label{sec:branch_perf}

\subsection{Misprediction Rate}

\begin{equation}
Misprediction\_Rate = \frac{\text{Mispredicted Branches}}{\text{Total Branches}}
\end{equation}

\begin{table}[H]
\centering
\caption{Predictor karşılaştırması}
\label{tab:predictor_comparison}
\begin{tabular}{lcc}
\toprule
\textbf{Predictor} & \textbf{Mispred Rate} & \textbf{Tablo Boyutu} \\
\midrule
2-bit Bimodal & 8-12\% & 1K entry \\
Gshare & 6-10\% & 4K entry \\
Tournament & 5-8\% & 4K + 4K entry \\
\bottomrule
\end{tabular}
\end{table}

\subsection{Misprediction Penalty}

\begin{equation}
Misprediction\_Penalty = Pipeline\_Depth + Recovery\_Latency
\end{equation}

Bu tasarımda:
\begin{itemize}
    \item Pipeline depth: ~5 stages
    \item Recovery latency: 0-1 cycle (eager recovery)
    \item Total penalty: ~5-6 cycles
\end{itemize}

\subsection{Misprediction Etkisi}

\begin{equation}
IPC_{effective} = IPC_{ideal} \times (1 - Mispred\_Rate \times \frac{Penalty}{Avg\_Branch\_Distance})
\end{equation}

\section{Memory Performansı}
\label{sec:memory_perf}

\subsection{Memory Access Pattern}

\begin{table}[H]
\centering
\caption{Memory access karakteristikleri}
\label{tab:memory_chars}
\begin{tabular}{lc}
\toprule
\textbf{Metrik} & \textbf{Tipik Değer} \\
\midrule
Load oranı & 20-25\% \\
Store oranı & 10-15\% \\
Ortalama memory latency & 1-2 cycle \\
\bottomrule
\end{tabular}
\end{table}

\subsection{LSQ Performans Etkisi}

3-port LSQ ve store-to-load forwarding'in performans etkisi:

\begin{nedenbox}
\textbf{LSQ Performans Özellikleri}

\begin{itemize}
    \item \textbf{3 Paralel Port:} Her çevrimde 3 memory operasyonu
    \item \textbf{Store-to-Load Forwarding:} Memory bypass ile latency azalması
    \item \textbf{Etki:} 3-pipe modunda \%83 performans artışı (1-pipe'a göre)
\end{itemize}

Gömülü sistemlerde bu trade-off kabul edilebilir.
\end{nedenbox}

\section{Kaynak Kullanımı}
\label{sec:resource_util}

\subsection{Kritik Kaynaklar}

\begin{table}[H]
\centering
\caption{Kaynak boyutları ve kullanım}
\label{tab:resource_sizes}
\begin{tabular}{lccc}
\toprule
\textbf{Kaynak} & \textbf{Boyut} & \textbf{Genişlik} & \textbf{Tipik Doluluk} \\
\midrule
ROB & 32 entry & 3 alloc/commit & 60-80\% \\
RS & 3 entry & 3 issue & 40-60\% \\
LSQ & 32 entry & 3 alloc, 3 head & 30-50\% \\
BRAT & 16 entry & 3 push/pop & 20-40\% \\
RAT & 32 entry & 3 lookup & 100\% \\
\bottomrule
\end{tabular}
\end{table}

\section{Benchmark Sonuçları}
\label{sec:benchmarks}

\subsection{Test Programları}

\begin{table}[H]
\centering
\caption{Benchmark karakteristikleri}
\label{tab:benchmarks}
\begin{tabular}{lp{5cm}c}
\toprule
\textbf{Benchmark} & \textbf{Karakteristik} & \textbf{Speedup} \\
\midrule
Dhrystone & Integer, az branch & 1.9x \\
Coremark & Mixed workload & 1.8x \\
Sieve & Loop-intensive & 1.85x \\
Quicksort & Branch-heavy & 1.7x \\
Matmul & Memory-intensive & 1.75x \\
\bottomrule
\end{tabular}
\end{table}

\subsection{Speedup Analizi}

\begin{table}[H]
\centering
\caption{Benchmark speedup sonuçları}
\label{fig:speedup_results}
\begin{tabular}{lc}
\toprule
\textbf{Benchmark} & \textbf{Speedup (x)} \\
\midrule
Dhrystone & 1.90 \\
Coremark & 1.80 \\
Sieve & 1.85 \\
Quicksort & 1.70 \\
Matmul & 1.75 \\
\midrule
\textbf{Average} & \textbf{1.80} \\
\bottomrule
\end{tabular}
\end{table}

\section{Performans Optimizasyon Önerileri}
\label{sec:optimization}

\subsection{Kısa Vadeli İyileştirmeler}

\begin{enumerate}
    \item \textbf{RS Derinliği Artırma:} 1 → 2 entry per RS
    \item \textbf{Daha İyi Predictor:} TAGE predictor
    \item \textbf{Partial Forwarding:} Farklı boyutlu store-load arası kısmi forwarding
\end{enumerate}

\subsection{Uzun Vadeli İyileştirmeler}

\begin{enumerate}
    \item \textbf{Speculative Load Execution:} Address unknown store'ları bypass
    \item \textbf{Daha Geniş Issue:} 4-way veya 6-way
    \item \textbf{SMT:} Simultaneous Multi-Threading
\end{enumerate}

\section{Performans Özeti}

\begin{table}[H]
\centering
\caption{Performans özeti}
\label{tab:perf_summary}
\begin{tabular}{lp{7cm}}
\toprule
\textbf{Metrik} & \textbf{Değer} \\
\midrule
Ortalama Speedup & 1.83x (scalar'a göre) \\
Tipik IPC & 1.5-2.0 \\
Branch Mispred Rate & 5-10\% \\
Mispred Penalty & 5-6 cycle \\
Ortalama Memory Latency & 1-2 cycle \\
\bottomrule
\end{tabular}
\end{table}


% Bölüm 9: Doğrulama Stratejisi
% =============================================================================
% BÖLÜM 9: DOĞRULAMA VE TEST
% =============================================================================

\chapter{Doğrulama ve Test}
\label{chap:dogrulama}

Bu bölümde, 3-way superscalar işlemcinin doğrulama metodolojisi, test stratejileri
ve doğrulama sonuçları açıklanmaktadır.

\section{Doğrulama Metodolojisi}
\label{sec:verification_methodology}

\subsection{Doğrulama Seviyeleri}

\begin{table}[H]
\centering
\caption{Doğrulama seviyeleri}
\label{tab:verification_levels}
\begin{tabular}{lp{6cm}c}
\toprule
\textbf{Seviye} & \textbf{Açıklama} & \textbf{Araç} \\
\midrule
Unit Test & Tek modül doğrulaması & SystemVerilog TB \\
Integration Test & Modül arası etkileşim & SystemVerilog TB \\
System Test & Tam işlemci testi & RISC-V test suite \\
Regression Test & Değişiklik sonrası doğrulama & Otomatik script \\
\bottomrule
\end{tabular}
\end{table}

\subsection{Test Ortamı}

\begin{lstlisting}[caption={Test bench yapısı}]
module tb_superscalar_core;
    logic clk, rst_n;

    // Clock generation
    initial begin
        clk = 0;
        forever #5 clk = ~clk;
    end

    // DUT instantiation
    rv32i_superscalar_core dut (
        .clk(clk),
        .rst_n(rst_n),
        ...
    );

    // Tracer for instruction monitoring
    tracer_interface tracer_0, tracer_1, tracer_2;

    // Test stimulus
    initial begin
        rst_n = 0;
        #100;
        rst_n = 1;

        // Wait for test completion
        wait(test_complete);
        $finish;
    end
endmodule
\end{lstlisting}

\section{Unit Test}
\label{sec:unit_test}

\subsection{RAT Unit Test}

RAT modülü için kritik test senaryoları:

\begin{table}[H]
\centering
\caption{RAT test senaryoları}
\label{tab:rat_tests}
\begin{tabular}{lp{6cm}}
\toprule
\textbf{Test} & \textbf{Açıklama} \\
\midrule
Basic Rename & Tek komut rename ve lookup \\
Same-Cycle Forward & Aynı cycle'da bağımlı komutlar \\
Commit Update & Commit sonrası RAT restore \\
BRAT Restore & Misprediction sonrası snapshot restore \\
Full Allocation & ROB dolu durumu \\
\bottomrule
\end{tabular}
\end{table}

\begin{lstlisting}[caption={RAT test örneği}]
// Test: Same-cycle forwarding
initial begin
    // Issue: ADD x1, x2, x3 (slot 0)
    //        SUB x4, x1, x5 (slot 1) - depends on x1
    decode_valid = 3'b011;
    rd_arch_0 = 5'd1;  // x1
    rs1_arch_1 = 5'd1; // x1 - should forward

    #10;

    // Verify: rs1_phys_1 should equal rd_phys_0
    assert(rs1_phys_1 == rd_phys_0)
        else $error("Same-cycle forward failed");
end
\end{lstlisting}

\subsection{BRAT Unit Test}

\begin{table}[H]
\centering
\caption{BRAT test senaryoları}
\label{tab:brat_tests}
\begin{tabular}{lp{6cm}}
\toprule
\textbf{Test} & \textbf{Açıklama} \\
\midrule
Push/Pop & Branch push ve commit pop \\
Snapshot Capture & RAT snapshot doğruluğu \\
Execute Match & ROB ID eşleştirme \\
Mispred Restore & Snapshot'tan RAT restore \\
Multi-Branch & Birden fazla in-flight branch \\
\bottomrule
\end{tabular}
\end{table}

\subsection{Reservation Station Test}

\begin{table}[H]
\centering
\caption{RS test senaryoları}
\label{tab:rs_tests}
\begin{tabular}{lp{6cm}}
\toprule
\textbf{Test} & \textbf{Açıklama} \\
\midrule
Direct Issue & Operandlar hazır, hemen issue \\
CDB Capture & CDB'den operand yakalama \\
Tag Match & Doğru producer tag eşleştirme \\
Eager Flush & Misprediction flush \\
\bottomrule
\end{tabular}
\end{table}

\section{Integration Test}
\label{sec:integration_test}

\subsection{Pipeline Integration}

Pipeline aşamaları arası entegrasyon testleri:

\begin{enumerate}
    \item \textbf{Fetch → Issue:} Instruction buffer akışı
    \item \textbf{Issue → Dispatch:} Rename ve allocation
    \item \textbf{Dispatch → Execute:} Operand hazırlığı
    \item \textbf{Execute → CDB:} Sonuç yayını
    \item \textbf{CDB → ROB:} Commit akışı
\end{enumerate}

\subsection{Hazard Integration}

\begin{lstlisting}[caption={RAW hazard test}]
// Test: RAW hazard across cycles
// Cycle N:   ADD x1, x2, x3
// Cycle N+1: SUB x4, x1, x5

// Verify:
// 1. SUB waits for ADD result
// 2. CDB capture works correctly
// 3. SUB issues after ADD completes
\end{lstlisting}

\section{System Test}
\label{sec:system_test}

\subsection{RISC-V Compliance Test}

RISC-V Foundation'ın resmi test suite'i kullanılır:

\begin{table}[H]
\centering
\caption{RISC-V test kategorileri}
\label{tab:riscv_tests}
\begin{tabular}{lcc}
\toprule
\textbf{Kategori} & \textbf{Test Sayısı} & \textbf{Durum} \\
\midrule
rv32ui-p (User Integer) & 39 & PASS \\
rv32um-p (Multiply) & 8 & N/A (RV32I) \\
rv32mi-p (Machine) & 6 & PASS \\
\bottomrule
\end{tabular}
\end{table}

\subsection{Benchmark Testleri}

\begin{table}[H]
\centering
\caption{Benchmark test sonuçları}
\label{tab:benchmark_tests}
\begin{tabular}{lccc}
\toprule
\textbf{Benchmark} & \textbf{Cycles} & \textbf{IPC} & \textbf{Durum} \\
\midrule
Dhrystone & 15,234 & 1.82 & PASS \\
Coremark & 28,456 & 1.75 & PASS \\
Sieve & 8,912 & 1.89 & PASS \\
\bottomrule
\end{tabular}
\end{table}

\section{Misprediction Test}
\label{sec:mispred_test}

Branch misprediction ve recovery testleri kritik öneme sahiptir:

\begin{lstlisting}[caption={Misprediction test}]
// Test: Branch misprediction recovery
//
// 1. Issue branch with wrong prediction
// 2. Continue issuing speculative instructions
// 3. Execute branch, detect misprediction
// 4. Verify:
//    - BRAT snapshot restored to RAT
//    - Speculative ROB entries invalidated
//    - Fetch redirected to correct PC
//    - Correct execution resumes

test_misprediction: begin
    // Force wrong prediction
    force tb.dut.branch_prediction = 1'b1;
    // Branch actually not-taken
    ...
    // Verify recovery
    wait(misprediction_detected);
    #10;
    assert(rat_restored) else $error("RAT not restored");
    assert(fetch_pc == correct_pc) else $error("PC not corrected");
end
\end{lstlisting}

\section{Stress Test}
\label{sec:stress_test}

\subsection{Resource Exhaustion}

\begin{table}[H]
\centering
\caption{Resource exhaustion testleri}
\label{tab:stress_tests}
\begin{tabular}{lp{6cm}}
\toprule
\textbf{Test} & \textbf{Senaryo} \\
\midrule
ROB Full & 32+ komut in-flight \\
BRAT Full & 16+ in-flight branch \\
LSQ Full & 8+ in-flight load/store \\
Back-to-back Mispred & Ardışık misprediction \\
\bottomrule
\end{tabular}
\end{table}

\subsection{Corner Case Test}

\begin{itemize}
    \item Commit sırasında misprediction
    \item Aynı cycle'da 3 branch misprediction
    \item ROB wrap-around durumu
    \item Reset sonrası ilk komut
\end{itemize}

\section{Tracer Altyapısı}
\label{sec:tracer}

Instruction tracer, her commit edilen komutu loglar:

\begin{lstlisting}[caption={Tracer output format}]
// Tracer output example
// Cycle | PC       | Instruction | rd | Result
// ------|----------|-------------|----|---------
// 1234  | 00000100 | ADD x1,x2,x3| x1 | 00000005
// 1234  | 00000104 | SUB x4,x1,x5| x4 | 00000003
// 1234  | 00000108 | LW  x6,0(x7)| x6 | 0000ABCD
\end{lstlisting}

\section{Regression Test}
\label{sec:regression}

Otomatik regression test akışı:

\begin{lstlisting}[caption={Regression script}]
#!/bin/tcsh
# Run all tests and check results

foreach test (rv32ui-p-*.hex)
    echo "Running $test..."
    ./sim +program=$test > $test.log

    if ($status != 0) then
        echo "FAIL: $test"
        exit 1
    endif
end

echo "All tests passed!"
\end{lstlisting}

\section{Doğrulama Özeti}

\begin{table}[H]
\centering
\caption{Doğrulama özeti}
\label{tab:verification_summary}
\begin{tabular}{lc}
\toprule
\textbf{Kategori} & \textbf{Durum} \\
\midrule
RISC-V Compliance & PASS (39/39) \\
Unit Tests & PASS (50+ tests) \\
Integration Tests & PASS \\
Benchmark Tests & PASS \\
Stress Tests & PASS \\
\bottomrule
\end{tabular}
\end{table}


% Bölüm 10: Sonuç
% =============================================================================
% BÖLÜM 10: SONUÇ VE GELECEK ÇALIŞMALAR
% =============================================================================

\chapter{Sonuç ve Gelecek Çalışmalar}
\label{chap:sonuc}

\section{Sonuç}
\label{sec:sonuc}

Bu tez çalışmasında, RISC-V RV32I komut seti mimarisini destekleyen 3-way
superscalar bir işlemci tasarlanmış ve gerçeklenmiştir. Tomasulo algoritması
temel alınarak out-of-order execution, register renaming ve spekülatif
branch execution mekanizmaları başarıyla implemente edilmiştir.

\subsection{Başarılan Hedefler}

\begin{enumerate}
    \item \textbf{3-Way Superscalar Mimari:}
    \begin{itemize}
        \item Her çevrimde 3 komut fetch, decode, issue ve commit
        \item Scalar işlemciye göre 1.83x ortalama hızlanma
        \item Tam RV32I komut seti desteği
    \end{itemize}

    \item \textbf{Tomasulo Algoritması Implementasyonu:}
    \begin{itemize}
        \item Register Alias Table (RAT) ile register renaming
        \item Tag-based operand tracking ile dependency resolution
        \item Common Data Bus (CDB) ile result broadcasting
        \item Reorder Buffer (ROB) ile in-order commit
    \end{itemize}

    \item \textbf{Spekülatif Execution:}
    \begin{itemize}
        \item Tournament branch predictor (Gshare + Bimodal)
        \item Return Address Stack (RAS) ile fonksiyon dönüş tahmini
        \item JALR predictor ile indirect jump tahmini
        \item BRAT ile eager misprediction recovery
    \end{itemize}

    \item \textbf{Memory Subsystem:}
    \begin{itemize}
        \item Load Store Queue (LSQ) ile memory operasyonu yönetimi
        \item In-order store commit ile memory consistency
        \item 3-port memory interface
    \end{itemize}
\end{enumerate}

\subsection{Teknik Katkılar}

Bu çalışmanın temel teknik katkıları şunlardır:

\begin{enumerate}
    \item \textbf{BRAT Mekanizması:}
    Branch Resolution Alias Table, misprediction recovery latency'sini
    minimize eden eager recovery mekanizması sağlar. ROB head'e ulaşmadan
    anında RAT restore yapılabilir.

    \item \textbf{Circular Buffer Tabanlı Kaynak Yönetimi:}
    ROB, BRAT ve LSQ için tek bir circular buffer yapısı kullanılarak
    alan ve karmaşıklık optimize edilmiştir.

    \item \textbf{Same-Cycle Forwarding:}
    Aynı çevrimde issue edilen bağımlı komutlar arasında kombinasyonel
    forwarding ile IPC kaybı önlenmiştir.

    \item \textbf{Modüler ve Genişletilebilir Tasarım:}
    SystemVerilog interface'leri ile modüller arası temiz ayrım sağlanmış,
    gelecek genişletmeler için uygun altyapı oluşturulmuştur.
\end{enumerate}

\subsection{Performans Sonuçları}

\begin{table}[H]
\centering
\caption{Performans özeti}
\label{tab:final_perf}
\begin{tabular}{lc}
\toprule
\textbf{Metrik} & \textbf{Değer} \\
\midrule
Issue Genişliği & 3-way \\
Ortalama IPC & 1.5-2.0 \\
Speedup (vs Scalar) & 1.83x \\
ROB Derinliği & 32 entry \\
Branch Mispred Rate & 5-10\% \\
Mispred Penalty & 5-6 cycle \\
\bottomrule
\end{tabular}
\end{table}

\section{Karşılaşılan Zorluklar}
\label{sec:challenges}

Tasarım sürecinde karşılaşılan başlıca zorluklar:

\begin{enumerate}
    \item \textbf{Timing Closure:}
    3-way paralel operasyonlar için kritik yol optimizasyonu gerekti.
    Özellikle RAT lookup ve same-cycle forwarding kombinasyonel
    gecikme ekledi.

    \item \textbf{Misprediction Recovery Karmaşıklığı:}
    BRAT snapshot'larının commit update'leri ile senkronizasyonu,
    köşe durumlarında hatalara yol açtı. Dikkatli tasarım ve
    kapsamlı test ile çözüldü.

    \item \textbf{LSQ-ROB Koordinasyonu:}
    Store commit'in ROB ve LSQ arasında doğru senkronizasyonu,
    özellikle misprediction durumlarında zorlu oldu.

    \item \textbf{Debug Zorluğu:}
    Out-of-order execution, geleneksel debug tekniklerini zorlaştırdı.
    Tracer altyapısı bu sorunu çözmek için geliştirildi.
\end{enumerate}

\section{Gelecek Çalışmalar}
\label{sec:future_work}

\subsection{Kısa Vadeli İyileştirmeler}

\begin{enumerate}
    \item \textbf{Daha Derin Reservation Station:}
    Her RS'yi 1 entry'den 2-4 entry'ye genişletmek, instruction window'u
    artırarak IPC iyileştirmesi sağlayabilir.

    \item \textbf{Partial Store-to-Load Forwarding:}
    Farklı boyutlu store-load arası kısmi byte forwarding desteği eklenebilir.

    \item \textbf{TAGE Branch Predictor:}
    Tournament predictor yerine TAGE kullanmak, misprediction rate'i
    düşürebilir.

    \item \textbf{Daha Geniş Fetch:}
    Fetch genişliğini 4-6 komuta çıkarmak, instruction supply darboğazını
    azaltabilir.
\end{enumerate}

\subsection{Orta Vadeli İyileştirmeler}

\begin{enumerate}
    \item \textbf{RV32M Extension:}
    Multiply/Divide komutları için ayrı functional unit eklemek.

    \item \textbf{Speculative Load Execution:}
    Address bilinmeyen store'ları bypass eden spekülatif load execution.

    \item \textbf{Cache Hierarchy:}
    L1 I-Cache ve D-Cache eklenmesi.

    \item \textbf{Exception Handling:}
    Tam RISC-V exception/interrupt desteği.
\end{enumerate}

\subsection{Uzun Vadeli Hedefler}

\begin{enumerate}
    \item \textbf{SMT (Simultaneous Multi-Threading):}
    Tek çekirdekte birden fazla thread çalıştırma.

    \item \textbf{RV64 Desteği:}
    64-bit veri yolu ve adres alanı.

    \item \textbf{Floating Point:}
    RV32F/RV32D extension desteği.

    \item \textbf{Vector Extension:}
    RV32V ile SIMD operasyonları.

    \item \textbf{Multi-Core:}
    Birden fazla superscalar çekirdek ile cache coherency.
\end{enumerate}

\section{Öğrenilen Dersler}
\label{sec:lessons}

Bu çalışmadan çıkarılan önemli dersler:

\begin{enumerate}
    \item \textbf{Basitlik Önce Gelir:}
    Karmaşık optimizasyonlar yerine çalışan basit tasarımdan başlamak,
    debug sürecini önemli ölçüde kolaylaştırdı.

    \item \textbf{Kapsamlı Test Kritik:}
    Out-of-order execution'da köşe durumları çok fazla. Kapsamlı
    unit test ve regression test olmadan güvenilir tasarım mümkün değil.

    \item \textbf{Tracer Altyapısı Zorunlu:}
    Instruction-level izleme olmadan out-of-order debug neredeyse
    imkansız. Tracer, tasarım sürecinin başında eklenmeliydi.

    \item \textbf{Modüler Tasarım:}
    SystemVerilog interface'leri ile modüller arası temiz ayrım,
    hem geliştirme hem de debug sürecini hızlandırdı.
\end{enumerate}

\section{Son Söz}

Bu tez çalışması, modern superscalar işlemci tasarımının temel
prensiplerini başarıyla uygulamış ve çalışan bir 3-way superscalar
RV32I işlemci ortaya koymuştur. Tomasulo algoritması, spekülatif
execution ve branch prediction mekanizmalarının entegrasyonu,
scalar işlemciye göre önemli performans artışı sağlamıştır.

Tasarım, hem eğitim amaçlı referans işlemci olarak hem de gömülü
sistem uygulamaları için pratik bir çözüm olarak kullanılabilir.
Modüler yapısı, gelecek genişletmeler için sağlam bir temel
oluşturmaktadır.


% =============================================================================
% KAYNAKÇA
% =============================================================================
\bibliographystyle{plain}
% \bibliography{references}

\end{document}
