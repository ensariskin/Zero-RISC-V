% =============================================================================
% BÖLÜM 3: FETCH STAGE
% =============================================================================

\chapter{Fetch Stage}
\label{chap:fetch}

Fetch stage, işlemcinin ilk pipeline aşamasıdır ve bellekten komutların getirilmesinden
sorumludur. Bu bölümde, 5-wide fetch mimarisi, komut tamponu, dal tahmini mekanizmaları
ve program sayacı yönetimi detaylı olarak açıklanmaktadır.

\section{Genel Bakış}
\label{sec:fetch_genel}

Fetch stage, her çevrimde en fazla 5 komut getirerek instruction buffer'ı doldurur.
Issue stage ise bu tampondan 3 komut çeker. Bu asimetrik tasarım, pipeline
verimliliğini artırır.

\begin{figure}[H]
\centering
\fbox{
\begin{minipage}{0.9\textwidth}
\centering
\textbf{Fetch Stage Veri Akışı}\\[1em]
\begin{tabular}{ccccc}
\fbox{Inst Memory} & $\xrightarrow{\text{5 inst}}$ & \fbox{Multi Fetch} & $\xrightarrow{\text{5 inst}}$ & \fbox{Inst Buffer} \\[0.5em]
 & & $\downarrow$ & & $\downarrow$ \\[0.5em]
 & & \scriptsize PC Controller & & $\xrightarrow{\text{3 inst}}$ \fbox{Issue Stage} \\
 & & \scriptsize Branch Predictor & & \\
\end{tabular}
\\[1em]
\textit{Misprediction feedback: Issue Stage $\rightarrow$ Multi Fetch}
\end{minipage}
}
\caption{Fetch stage blok diyagramı}
\label{fig:fetch_block}
\end{figure}

\subsection{Fetch Stage Bileşenleri}

\begin{table}[H]
\centering
\caption{Fetch stage modülleri}
\label{tab:fetch_modules}
\begin{tabular}{llp{7cm}}
\toprule
\textbf{Modül} & \textbf{Dosya} & \textbf{Görev} \\
\midrule
Multi Fetch & \texttt{multi\_fetch.sv} & 5 komut paralel getirme koordinasyonu \\
Instruction Buffer & \texttt{instruction\_buffer\_new.sv} & Fetch-decode ayrıştırma tamponu \\
PC Controller & \texttt{pc\_ctrl\_super.sv} & Program sayacı yönetimi \\
Jump Controller & \texttt{jump\_controller\_super.sv} & Dal/atlama komut tespiti \\
Tournament Predictor & \texttt{tournament\_predictor.sv} & Hibrit dal tahmini \\
GShare Predictor & \texttt{gshare\_predictor\_super.sv} & Global history tabanlı tahmin \\
JALR Predictor & \texttt{jalr\_predictor.sv} & Dolaylı atlama hedef tahmini \\
\bottomrule
\end{tabular}
\end{table}

\section{Multi Fetch Modülü}
\label{sec:multi_fetch}

\module{multi\_fetch} modülü, fetch stage'in ana koordinasyon birimidir. Bellekten
5 komut okur, dal tahminlerini yapar ve instruction buffer'a iletir.

\subsection{Tasarım Amacı}

\begin{nedenbox}
\textbf{Neden 5-wide fetch?}

3-way issue için 5-wide fetch seçilmesinin nedenleri:
\begin{itemize}
    \item \textbf{Buffer Doldurma:} Misprediction sonrası instruction buffer hızla
          yeniden doldurulmalı. 5 > 3 olduğu için buffer birikir.
    \item \textbf{Dal Kesilmesi:} Bir dal ``taken'' tahmin edilirse, sonraki komutlar
          geçersiz olur. 5 komut getirerek, en az 1 geçerli komut garantilenir.
    \item \textbf{Fetch Bandwidth:} Modern bellek sistemleri geniş bant genişliği sunar.
          Bu kapasiteyi kullanmamak israftır.
\end{itemize}
\end{nedenbox}

\subsection{Komut Geçerlilik Mantığı}

Fetch edilen 5 komuttan bazıları geçersiz olabilir. Bir dal komutu ``taken'' tahmin
edilirse, ondan sonraki komutlar program akışında yer almaz:

\begin{lstlisting}[caption={Fetch geçerlilik sinyalleri}]
// Branch prediction invalidation logic
assign block_0 = jump_0 | jalr_0;
assign block_1 = jump_1 | jalr_1;
assign block_2 = jump_2 | jalr_2;
assign block_3 = jump_3 | jalr_3;

// Final fetch valid signals
assign fetch_valid_o[0] = base_valid;
assign fetch_valid_o[1] = base_valid & ~block_0;
assign fetch_valid_o[2] = base_valid & ~block_0 & ~block_1;
assign fetch_valid_o[3] = base_valid & ~block_0 & ~block_1 & ~block_2;
assign fetch_valid_o[4] = base_valid & ~block_0 & ~block_1 & ~block_2 & ~block_3;
\end{lstlisting}

\subsection{Misprediction İşleme}

BRAT'tan gelen misprediction sinyalleri, oldest-first öncelikle işlenir:

\begin{lstlisting}[caption={Eager flush mantığı}]
always_comb begin
    if (misprediction_i_0) begin
        eager_flush = 1'b1;
        eager_flush_target_pc = correct_pc_i_0;
    end else if (misprediction_i_1) begin
        eager_flush = 1'b1;
        eager_flush_target_pc = correct_pc_i_1;
    end else if (misprediction_i_2) begin
        eager_flush = 1'b1;
        eager_flush_target_pc = correct_pc_i_2;
    end else begin
        eager_flush = 1'b0;
        eager_flush_target_pc = {size{1'b0}};
    end
end
\end{lstlisting}

\begin{nedenbox}
\textbf{Neden oldest-first öncelik?}

Aynı anda birden fazla dal çözümlenebilir. En yaşlı misprediction önceliklidir çünkü:
\begin{itemize}
    \item Genç dallar, yaşlı dalın spekülatif yolunda olabilir
    \item Yaşlı misprediction düzeltildiğinde, genç dallar flush edilecek
    \item Genç misprediction'ları işlemek gereksiz çalışma olur
\end{itemize}
\end{nedenbox}

\section{Instruction Buffer}
\label{sec:instruction_buffer}

\module{instruction\_buffer\_new} modülü, fetch ve decode aşamalarını ayrıştıran
bir FIFO tampondur.

\subsection{Tasarım Amacı}

\begin{nedenbox}
\textbf{Neden instruction buffer gerekli?}
\begin{itemize}
    \item \textbf{Hız Uyumsuzluğu:} Fetch 5-wide, issue 3-wide. Buffer bu farkı dengeler.
    \item \textbf{Stall İzolasyonu:} Issue stall olduğunda fetch devam edebilir (buffer dolana kadar).
    \item \textbf{Misprediction Toleransı:} Buffer doluyken, recovery süresi kısalır.
    \item \textbf{Latency Gizleme:} Bellek gecikmesi buffer tarafından emilir.
\end{itemize}
\end{nedenbox}

\subsection{Buffer Yapısı}

Buffer, circular buffer olarak implement edilmiştir:

\begin{table}[H]
\centering
\caption{Instruction buffer parametreleri}
\label{tab:ibuf_params}
\begin{tabular}{lcp{6cm}}
\toprule
\textbf{Parametre} & \textbf{Değer} & \textbf{Açıklama} \\
\midrule
BUFFER\_DEPTH & 16 & Maksimum tamponlanabilir komut sayısı \\
Giriş Genişliği & 5-wide & Her çevrimde yazılabilecek komut \\
Çıkış Genişliği & 3-wide & Her çevrimde okunabilecek komut \\
\bottomrule
\end{tabular}
\end{table}

Her buffer entry'si şu alanları içerir:
\begin{itemize}
    \item \sig{instruction}: 32-bit komut kodu
    \item \sig{pc}: Komutun program sayacı değeri
    \item \sig{imm}: Önceden decode edilmiş immediate değeri
    \item \sig{branch\_prediction}: Dal tahmini sonucu
    \item \sig{pc\_at\_prediction}: Tahmin yapıldığındaki PC
    \item \sig{global\_history}: Dal predictor için global history
    \item \sig{ras\_tos\_checkpoint}: RAS checkpoint pointer
\end{itemize}

\subsection{Backpressure Yönetimi}

Buffer dolduğunda, fetch stage'e backpressure uygulanır:

\begin{lstlisting}[caption={Backpressure mantığı}]
// Conservative: leave space for 5 instructions
assign fetch_ready_o = !flush_i && !buffer_full_o && (space_available >= 5);
\end{lstlisting}

\begin{nedenbox}
\textbf{Neden \texttt{space\_available >= 5}?}

Fetch stage, mevcut çevrimde zaten 5 komut göndermiş olabilir. Bu komutlar henüz
buffer'a yazılmamışken (kombinasyonel gecikme), yeni fetch başlatılmamalı.
Konservatif yaklaşım deadlock'u önler.
\end{nedenbox}

\subsection{Forwarding Mantığı}

Buffer boşken ve fetch valid ise, komutlar doğrudan çıkışa forward edilir:

\begin{lstlisting}[caption={Direct forwarding}]
assign use_fwd_0 = (count == 0) & (num_to_write >= 1) & read_en_0;
assign use_fwd_1 = (count <= read_en_0) & (num_to_write >= (1 + use_fwd_0)) & read_en_1;
assign use_fwd_2 = (count <= read_en_0 + read_en_1) &
                   (num_to_write >= (1 + use_fwd_0 + use_fwd_1)) & read_en_2;
\end{lstlisting}

Bu mekanizma, buffer boşken bile zero-cycle forwarding sağlar, pipeline bubble'ları
minimize eder.

\section{PC Controller}
\label{sec:pc_controller}

\module{pc\_ctrl\_super} modülü, program sayacı yönetiminden sorumludur.

\subsection{PC Güncelleme Senaryoları}

\begin{table}[H]
\centering
\caption{PC güncelleme öncelikleri}
\label{tab:pc_priority}
\begin{tabular}{clp{6cm}}
\toprule
\textbf{Öncelik} & \textbf{Senaryo} & \textbf{Yeni PC Değeri} \\
\midrule
1 & Misprediction & \sig{correct\_pc} (BRAT'tan) \\
2 & JALR (tahminli) & \sig{jalr\_prediction\_target} \\
3 & Branch/JAL (taken) & \sig{PC + imm} \\
4 & Normal akış & \sig{PC + 20} (5 komut) \\
\bottomrule
\end{tabular}
\end{table}

\subsection{Paralel PC Hesaplama}

5 komut için PC değerleri paralel olarak hesaplanır:

\begin{lstlisting}[caption={Paralel PC hesaplama}]
assign current_pc_0 = pc_current_val;
assign current_pc_1 = parallel_mode ? pc_current_val + 32'd4 : current_pc_0;
assign current_pc_2 = parallel_mode ? pc_current_val + 32'd8 : current_pc_0;
assign current_pc_3 = parallel_mode ? pc_current_val + 32'd12 : current_pc_0;
assign current_pc_4 = parallel_mode ? pc_current_val + 32'd16 : current_pc_0;
\end{lstlisting}

\subsection{Misprediction Recovery}

Misprediction durumunda PC, BRAT'tan gelen doğru değere ayarlanır:

\begin{lstlisting}[caption={PC misprediction recovery}]
parametric_mux #(.mem_width(size), .mem_depth(2)) correction_mux(
    .addr(misprediction),
    .data_in({correct_pc, pc_plus}),
    .data_out(pc_new_val));
\end{lstlisting}

\section{Dal Tahmini}
\label{sec:branch_prediction}

İşlemci, üç farklı dal tahmin mekanizması içerir. Bu mekanizmalar, farklı dal
davranış paternlerini hedefler.

% =============================================================================
% DAL TAHMİNİ ALT BÖLÜMLERİ
% =============================================================================

\subsection{Dal Tahmini Genel Bakış}

Dal komutları, program akışında belirsizlik yaratır. Dal sonucu belirlenene kadar
(execute aşaması) işlemci, hangi komutları getireceğini bilemez. Dal tahmini,
bu belirsizliği spekülatif olarak çözerek pipeline verimliliğini artırır.

\begin{nedenbox}
\textbf{Neden dal tahmini kritik?}

Dal komutları tipik programlarda \%15-25 oranında görülür. Tahminsiz bir işlemcide
her dal 3+ çevrim gecikmeye neden olur. \%20 dal oranı ve 3 çevrim ceza ile:
\[
\text{Efektif IPC} = \frac{1}{1 + 0.20 \times 3} = 0.625
\]
Bu, teorik IPC'nin \%37.5 altındadır. Doğru tahmin bu kaybı minimize eder.
\end{nedenbox}

\subsection{Tahmin Mekanizmaları}

İşlemci üç farklı dal tahmin mekanizması içerir:

\begin{table}[H]
\centering
\caption{Dal tahmin mekanizmaları karşılaştırması}
\label{tab:predictors}
\begin{tabular}{lp{3cm}p{3cm}p{4cm}}
\toprule
\textbf{Mekanizma} & \textbf{Güçlü Yön} & \textbf{Zayıf Yön} & \textbf{En İyi Senaryo} \\
\midrule
2-Bit Sayaç & Basit, düşük maliyet & Korelasyon yakalayamaz & Tutarlı dallar (döngüler) \\
GShare & Global korelasyon & Aliasing sorunları & Korele dallar \\
Tournament & Adaptif seçim & Yüksek maliyet & Karışık iş yükleri \\
\bottomrule
\end{tabular}
\end{table}

\subsubsection{2-Bit Sayaç (Bimodal Predictor)}

En basit dal tahmin mekanizmasıdır. Her dal adresi için 2-bit doyurulmuş sayaç tutulur:

\begin{figure}[H]
\centering
\fbox{
\begin{minipage}{0.85\textwidth}
\centering
\textbf{2-Bit Sayaç Durum Makinesi}\\[1em]
\begin{tabular}{cccc}
\fbox{SNT (00)} & $\xrightarrow{\text{taken}}$ & \fbox{WNT (01)} & $\xrightarrow{\text{taken}}$ \\[0.3em]
$\circlearrowleft$ NT & & $\updownarrow$ not taken & \\[0.3em]
& & \fbox{WT (10)} & $\xrightarrow{\text{taken}}$ \\[0.3em]
& & $\updownarrow$ not taken & \\[0.3em]
& & \fbox{ST (11)} & $\circlearrowleft$ T \\
\end{tabular}
\\[1em]
\textit{SNT, WNT: Tahmin = Not Taken \quad|\quad WT, ST: Tahmin = Taken}
\end{minipage}
}
\caption{2-bit sayaç durum makinesi}
\label{fig:2bit_fsm}
\end{figure}

\begin{nedenbox}
\textbf{Neden 2-bit (1-bit değil)?}

1-bit sayaç, tek bir yanlış sonuçta hemen fikir değiştirir. Döngü sonlarında
bu sorunlu olur: döngü 100 kez ``taken'' olduktan sonra 1 kez ``not taken''
olur ve 1-bit sayaç hemen ``not taken'' tahmin etmeye başlar.

2-bit sayaç, iki ardışık yanlış sonuç gerektirir. Bu, döngü sonu gibi
``anomali'' durumlarına karşı dayanıklılık sağlar.
\end{nedenbox}

\subsubsection{GShare Predictor}

GShare, \concept{global history} ile PC'yi XOR'layarak indeks oluşturur.
Bu, farklı dallar arasındaki korelasyonu yakalar.

\begin{lstlisting}[caption={GShare indeks hesaplama}]
// GShare index = PC XOR Global History Register
assign predict_index_0 = current_pc_0[INDEX_WIDTH+1:2] ^ ghm0;
assign predict_index_1 = current_pc_1[INDEX_WIDTH+1:2] ^ ghm1;
...
\end{lstlisting}

\paragraph{Global History Register (GHR)}

GHR, son N dalın sonuçlarını (taken/not-taken) bir shift register'da tutar.
Her dal çözümlendiğinde, sonuç GHR'a shift edilir.

\begin{lstlisting}[caption={GHR güncelleme}]
// Per-slot history advance uses predicted bits
assign global_history_1 = slot_branch_0 ?
    {global_history_0[INDEX_WIDTH-2:0], branch_taken_o_0} : global_history_0;
assign global_history_2 = slot_branch_1 ?
    {global_history_1[INDEX_WIDTH-2:0], branch_taken_o_1} : global_history_1;
\end{lstlisting}

\begin{nedenbox}
\textbf{Neden spekülatif GHR güncellemesi?}

Dal sonucu execute aşamasında belli olur, ancak tahmin fetch aşamasında yapılır.
Eğer GHR güncellemesi commit'e kadar bekleseydi, ardışık dallar için yanlış
history kullanılırdı.

Spekülatif güncelleme, tahmin edilen sonucu GHR'a hemen ekler. Misprediction
durumunda GHR, BRAT'tan restore edilir.
\end{nedenbox}

\subsubsection{Tournament Predictor}

Tournament predictor, GShare ve bimodal predictor'ları birleştirir. Bir
\concept{chooser table}, hangi predictor'ın kullanılacağına karar verir.

\begin{lstlisting}[caption={Tournament chooser durumları}]
typedef enum logic [1:0] {
    STRONG_BIMODAL = 2'b00,
    WEAK_BIMODAL   = 2'b01,
    WEAK_GSHARE    = 2'b10,
    STRONG_GSHARE  = 2'b11
} chooser_state_e;
\end{lstlisting}

\paragraph{Chooser Güncelleme Kuralı}

Chooser, yalnızca iki predictor farklı tahmin yaptığında güncellenir:
\begin{itemize}
    \item Her iki predictor aynı tahmini yaparsa: chooser değişmez
    \item Predictor'lar farklı tahmin yaparsa: doğru olanın yönünde güncelle
\end{itemize}

\begin{nedenbox}
\textbf{Neden her iki predictor'ı da eğitiyoruz?}

Alternatif: Sadece seçilen predictor'ı eğitmek. Ancak bu yaklaşımda, chooser
yanlış predictor'a sabitlenirse, diğer predictor güncellenemez ve ``öğrenemez.''

Her iki predictor'ı eğitmek, chooser değiştiğinde diğer predictor'ın hazır
olmasını sağlar.
\end{nedenbox}

\paragraph{History Packing}

Tournament predictor, hem GShare hem de bimodal bilgisini BRAT'a kaydetmelidir.
Bu bilgi, \sig{global\_history\_*\_o} sinyalinde paketlenir:

\begin{lstlisting}[caption={Tournament history packing}]
// Layout of global_history bus (MSB..LSB):
//   [INDEX_WIDTH+2:3] = GHR_before   (INDEX_WIDTH bits, from gshare)
//   [2]               = gshare_pred
//   [1]               = bimodal_pred
//   [0]               = chooser_sel  (1 => gshare, 0 => bimodal)
\end{lstlisting}

\subsection{JALR Predictor}

JALR (Jump And Link Register) komutları, hedef adresi bir register'dan okur.
Bu nedenle, hedef adres fetch aşamasında bilinmez.

\module{jalr\_predictor} modülü, bir \concept{Branch Target Buffer} (BTB) kullanarak
JALR hedeflerini tahmin eder.

\begin{table}[H]
\centering
\caption{JALR predictor özellikleri}
\label{tab:jalr_predictor}
\begin{tabular}{lp{8cm}}
\toprule
\textbf{Özellik} & \textbf{Değer/Açıklama} \\
\midrule
Tablo Boyutu & 32 entry (parametrik) \\
İndeksleme & PC[INDEX\_WIDTH+1:2] \\
Saklanan Veri & Hedef PC adresi \\
Güncelleme & Execute aşamasından (gerçek hedef) \\
\bottomrule
\end{tabular}
\end{table}

\begin{nedenbox}
\textbf{Neden ayrı JALR predictor?}

Dal predictor'ları yön tahmini yapar (taken/not-taken). JALR için bu yeterli
değildir; hedef adres de tahmin edilmelidir.

JALR'lar genellikle fonksiyon dönüşleri (RET = JALR x0, ra, 0) veya dolaylı
çağrılardır. BTB, son kullanılan hedefi saklar ve çoğu durumda doğru tahmin
sağlar.
\end{nedenbox}

\subsection{Return Address Stack (RAS)}

Fonksiyon dönüşleri (RET) için özel bir yapı olan \concept{Return Address Stack},
CALL/RET çiftlerini takip eder.

\paragraph{RAS Operasyonları}
\begin{itemize}
    \item \textbf{CALL (JAL/JALR with rd=ra):} Dönüş adresini (PC+4) stack'e push et
    \item \textbf{RET (JALR x0, ra, 0):} Stack'ten pop et ve hedef olarak kullan
\end{itemize}

\paragraph{Spekülatif RAS ve Checkpoint}

RAS, spekülatif olarak güncellenir. Misprediction durumunda geri alınabilmesi
için, her dal komutu RAS top-of-stack pointer'ını BRAT'a kaydeder:

\begin{lstlisting}[caption={RAS checkpoint}]
output logic [2:0] ras_tos_checkpoint_o, // RAS TOS at fetch time
input  logic ras_restore_en_i,
input  logic [2:0] ras_restore_tos_i
\end{lstlisting}

\subsection{Predictor Güncelleme Akışı}

\begin{enumerate}
    \item \textbf{Fetch:} Tahmin yapılır, GHR spekülatif güncellenir
    \item \textbf{Issue:} Tahmin bilgisi BRAT'a kaydedilir
    \item \textbf{Execute:} Dal çözümlenir, gerçek sonuç belirlenir
    \item \textbf{BRAT Resolution:} Sonuç in-order olarak çıkar
    \item \textbf{Predictor Update:} Tablo güncellenir, GHR düzeltilir
\end{enumerate}

\begin{lstlisting}[caption={Predictor güncelleme ayrımı}]
// Branch predictor: update when update_valid & !is_jalr
assign branch_update_valid_0 = update_valid_i_0 & ~is_jalr_i_0;

// JALR predictor: update when update_valid & is_jalr
assign jalr_update_valid_0 = update_valid_i_0 & is_jalr_i_0;
\end{lstlisting}


\section{Fetch Stage Özeti}

\begin{table}[H]
\centering
\caption{Fetch stage özellikleri}
\label{tab:fetch_summary}
\begin{tabular}{lp{8cm}}
\toprule
\textbf{Özellik} & \textbf{Değer/Açıklama} \\
\midrule
Fetch Genişliği & 5 komut/çevrim \\
Buffer Derinliği & 16 entry \\
Dal Tahmin Yöntemleri & Tournament, GShare, 2-bit sayaç \\
JALR Tahmini & BTB tabanlı hedef tahmini \\
Misprediction Kaynağı & BRAT (eager recovery) \\
Backpressure & fetch\_ready\_o sinyali ile \\
\bottomrule
\end{tabular}
\end{table}
