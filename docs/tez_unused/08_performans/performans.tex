% =============================================================================
% BÖLÜM 8: PERFORMANS ANALİZİ
% =============================================================================

\chapter{Performans Analizi}
\label{chap:performans}

Bu bölümde, 3-way superscalar işlemcinin performans karakteristikleri,
darboğazlar ve optimizasyon stratejileri analiz edilmektedir.

\section{Performans Metrikleri}
\label{sec:metrics}

\subsection{Instructions Per Cycle (IPC)}

IPC, işlemci verimliliğinin temel ölçüsüdür:

\begin{equation}
IPC = \frac{\text{Toplam Komut Sayısı}}{\text{Toplam Çevrim Sayısı}}
\end{equation}

\begin{table}[H]
\centering
\caption{Teorik vs gerçek IPC}
\label{tab:ipc}
\begin{tabular}{lcc}
\toprule
\textbf{Konfigürasyon} & \textbf{Teorik Maks} & \textbf{Tipik} \\
\midrule
Scalar (1-way) & 1.0 & 0.7-0.9 \\
3-way Superscalar & 3.0 & 1.5-2.0 \\
\bottomrule
\end{tabular}
\end{table}

\subsection{Speedup}

3-way superscalar'ın scalar'a göre hızlanması:

\begin{equation}
Speedup = \frac{IPC_{superscalar}}{IPC_{scalar}} = \frac{IPC_{3-way}}{IPC_{1-way}}
\end{equation}

Test sonuçlarına göre tipik speedup: \textbf{1.83x - 1.93x}

\begin{nedenbox}
\textbf{Neden teorik 3x'e ulaşılamıyor?}

\begin{itemize}
    \item \textbf{RAW hazards:} Gerçek veri bağımlılıkları parallelliği sınırlar
    \item \textbf{Control hazards:} Branch misprediction'lar pipeline'ı boşaltır
    \item \textbf{Memory latency:} Load/store sıralı çalışır
    \item \textbf{Resource conflicts:} Sınırlı functional unit sayısı
\end{itemize}
\end{nedenbox}

\section{Pipeline Stall Analizi}
\label{sec:stall_analysis}

\subsection{Stall Kaynakları}

\begin{table}[H]
\centering
\caption{Stall kaynakları ve etkileri}
\label{tab:stall_sources}
\begin{tabular}{lp{5cm}c}
\toprule
\textbf{Kaynak} & \textbf{Açıklama} & \textbf{Tipik Etki} \\
\midrule
ROB Full & ROB doldu, yeni komut kabul edilemiyor & 5-10\% \\
RS Full & Reservation station dolu & 3-5\% \\
LSQ Full & Load/store queue dolu & 2-4\% \\
BRAT Full & Maksimum in-flight branch sayısına ulaşıldı & 1-3\% \\
Memory Latency & Memory yanıt bekleniyor & 10-20\% \\
Branch Misprediction & Yanlış yol flush ediliyor & 5-15\% \\
\bottomrule
\end{tabular}
\end{table}

\subsection{Occupancy Analizi}

Pipeline occupancy, kaynakların ne kadar verimli kullanıldığını gösterir:

\begin{lstlisting}[caption={Occupancy hesaplama}]
// ROB Occupancy
ROB_occupancy = (tail_ptr - head_ptr) / ROB_DEPTH;

// RS Occupancy
RS_occupancy = occupied_entries / TOTAL_RS_ENTRIES;

// Average issue width
avg_issue_width = committed_instructions / total_cycles;
\end{lstlisting}

\section{Branch Prediction Performansı}
\label{sec:branch_perf}

\subsection{Misprediction Rate}

\begin{equation}
Misprediction\_Rate = \frac{\text{Mispredicted Branches}}{\text{Total Branches}}
\end{equation}

\begin{table}[H]
\centering
\caption{Predictor karşılaştırması}
\label{tab:predictor_comparison}
\begin{tabular}{lcc}
\toprule
\textbf{Predictor} & \textbf{Mispred Rate} & \textbf{Tablo Boyutu} \\
\midrule
2-bit Bimodal & 8-12\% & 1K entry \\
Gshare & 6-10\% & 4K entry \\
Tournament & 5-8\% & 4K + 4K entry \\
\bottomrule
\end{tabular}
\end{table}

\subsection{Misprediction Penalty}

\begin{equation}
Misprediction\_Penalty = Pipeline\_Depth + Recovery\_Latency
\end{equation}

Bu tasarımda:
\begin{itemize}
    \item Pipeline depth: ~5 stages
    \item Recovery latency: 0-1 cycle (eager recovery)
    \item Total penalty: ~5-6 cycles
\end{itemize}

\subsection{Misprediction Etkisi}

\begin{equation}
IPC_{effective} = IPC_{ideal} \times (1 - Mispred\_Rate \times \frac{Penalty}{Avg\_Branch\_Distance})
\end{equation}

\section{Memory Performansı}
\label{sec:memory_perf}

\subsection{Memory Access Pattern}

\begin{table}[H]
\centering
\caption{Memory access karakteristikleri}
\label{tab:memory_chars}
\begin{tabular}{lc}
\toprule
\textbf{Metrik} & \textbf{Tipik Değer} \\
\midrule
Load oranı & 20-25\% \\
Store oranı & 10-15\% \\
Ortalama memory latency & 1-2 cycle \\
\bottomrule
\end{tabular}
\end{table}

\subsection{LSQ Performans Etkisi}

3-port LSQ ve store-to-load forwarding'in performans etkisi:

\begin{nedenbox}
\textbf{LSQ Performans Özellikleri}

\begin{itemize}
    \item \textbf{3 Paralel Port:} Her çevrimde 3 memory operasyonu
    \item \textbf{Store-to-Load Forwarding:} Memory bypass ile latency azalması
    \item \textbf{Etki:} 3-pipe modunda \%83 performans artışı (1-pipe'a göre)
\end{itemize}

Gömülü sistemlerde bu trade-off kabul edilebilir.
\end{nedenbox}

\section{Kaynak Kullanımı}
\label{sec:resource_util}

\subsection{Kritik Kaynaklar}

\begin{table}[H]
\centering
\caption{Kaynak boyutları ve kullanım}
\label{tab:resource_sizes}
\begin{tabular}{lccc}
\toprule
\textbf{Kaynak} & \textbf{Boyut} & \textbf{Genişlik} & \textbf{Tipik Doluluk} \\
\midrule
ROB & 32 entry & 3 alloc/commit & 60-80\% \\
RS & 3 entry & 3 issue & 40-60\% \\
LSQ & 32 entry & 3 alloc, 3 head & 30-50\% \\
BRAT & 16 entry & 3 push/pop & 20-40\% \\
RAT & 32 entry & 3 lookup & 100\% \\
\bottomrule
\end{tabular}
\end{table}

\section{Benchmark Sonuçları}
\label{sec:benchmarks}

\subsection{Test Programları}

\begin{table}[H]
\centering
\caption{Benchmark karakteristikleri}
\label{tab:benchmarks}
\begin{tabular}{lp{5cm}c}
\toprule
\textbf{Benchmark} & \textbf{Karakteristik} & \textbf{Speedup} \\
\midrule
Dhrystone & Integer, az branch & 1.9x \\
Coremark & Mixed workload & 1.8x \\
Sieve & Loop-intensive & 1.85x \\
Quicksort & Branch-heavy & 1.7x \\
Matmul & Memory-intensive & 1.75x \\
\bottomrule
\end{tabular}
\end{table}

\subsection{Speedup Analizi}

\begin{table}[H]
\centering
\caption{Benchmark speedup sonuçları}
\label{fig:speedup_results}
\begin{tabular}{lc}
\toprule
\textbf{Benchmark} & \textbf{Speedup (x)} \\
\midrule
Dhrystone & 1.90 \\
Coremark & 1.80 \\
Sieve & 1.85 \\
Quicksort & 1.70 \\
Matmul & 1.75 \\
\midrule
\textbf{Average} & \textbf{1.80} \\
\bottomrule
\end{tabular}
\end{table}

\section{Performans Optimizasyon Önerileri}
\label{sec:optimization}

\subsection{Kısa Vadeli İyileştirmeler}

\begin{enumerate}
    \item \textbf{RS Derinliği Artırma:} 1 → 2 entry per RS
    \item \textbf{Daha İyi Predictor:} TAGE predictor
    \item \textbf{Partial Forwarding:} Farklı boyutlu store-load arası kısmi forwarding
\end{enumerate}

\subsection{Uzun Vadeli İyileştirmeler}

\begin{enumerate}
    \item \textbf{Speculative Load Execution:} Address unknown store'ları bypass
    \item \textbf{Daha Geniş Issue:} 4-way veya 6-way
    \item \textbf{SMT:} Simultaneous Multi-Threading
\end{enumerate}

\section{Performans Özeti}

\begin{table}[H]
\centering
\caption{Performans özeti}
\label{tab:perf_summary}
\begin{tabular}{lp{7cm}}
\toprule
\textbf{Metrik} & \textbf{Değer} \\
\midrule
Ortalama Speedup & 1.83x (scalar'a göre) \\
Tipik IPC & 1.5-2.0 \\
Branch Mispred Rate & 5-10\% \\
Mispred Penalty & 5-6 cycle \\
Ortalama Memory Latency & 1-2 cycle \\
\bottomrule
\end{tabular}
\end{table}
