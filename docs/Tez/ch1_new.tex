\phantomsection
%%%%%%%%%%%%%%%%%%%%%%%%%%%%%%%%%%%%%%%%%%%%%%%%%%%%%%%%%%%%%%%%%
\chapter{GİRİŞ}\label{ch:giris}
%%%%%%%%%%%%%%%%%%%%%%%%%%%%%%%%%%%%%%%%%%%%%%%%%%%%%%%%%%%%%%%%%

Yarı iletken teknolojisindeki hızlı gelişmeler, transistör boyutlarının on nanometrenin altına inmesini mümkün kılmıştır. Bu minyatürleşme performans ve enerji verimliliği açısından önemli kazanımlar sağlamakla birlikte, çeşitli güvenilirlik sorunlarını da beraberinde getirmektedir. Küçülen yapı boyutları, devrelerin çevresel faktörlere karşı duyarlılığını artırmakta ve yumuşak hata (soft error) olarak tanımlanan geçici arızaların oluşma olasılığını yükseltmektedir \cite{baumann}.

Özellikle uzay, havacılık ve otomotiv gibi kritik uygulama alanlarında kozmik ışınlar ve yüksek enerjili parçacıklar, yarı iletken malzemelerle etkileşerek tek olay bozulması (Single Event Upset - SEU) olarak adlandırılan bit değişikliklerine neden olabilmektedir. Birden fazla bitin aynı anda etkilenmesi durumunda ise çoklu bit bozulması (Multiple Bit Upset - MBU) meydana gelmekte ve sistem üzerindeki olumsuz etkiler önemli ölçüde artmaktadır \cite{rogenmoser_hmr_2023, annink}. Santos ve arkadaşlarının gerçekleştirdiği radyasyon testlerinde, hataların yüzde yetmiş sekizinin belleklerde, yüzde on beşinin ise yazmaç dosyasında meydana geldiği gözlemlenmiştir \cite{santos_2023}. Bu sonuçlar, kritik donanım yapılarının korunması gerektiğini açıkça ortaya koymaktadır.

Bu tür hatalara karşı çeşitli hata toleransı yöntemleri geliştirilmiştir. Uzaysal yedeklilik, zamansal yedeklilik ve bilgi yedekliliği gibi farklı yaklaşımlar mevcut olup, güvenilir sistemlerde bu yöntemlerin birlikte kullanılması gerekmektedir. Her yöntemin farklı maliyet, performans ve koruma özellikleri bulunmakta olup, uygulama gereksinimlerine göre en uygun kombinasyonun seçilmesi kritik öneme sahiptir.

Hata toleranslı tasarımlar, uygulama seviyesine göre iki ana kategoride incelenebilir. Birinci kategori, birden fazla işlemci çekirdeğinin kilitli adım (lock-step) yöntemiyle birlikte çalıştırıldığı çekirdek seviyesi çözümlerdir. İkinci kategori ise tek bir çekirdek içinde boruhattı seviyesinde koruma sağlayan çekirdek içi (intra-core) çözümlerdir. Çekirdek içi çözümler, hataların oluştukları aşamada tespit edilmesini sağlamakta ve yayılmalarını önlemektedir \cite{dorflinger_2022}. Ayrıca bu yaklaşım, tespit gecikmesini minimize ederek daha hızlı toparlanma imkanı sunmaktadır.

Geleneksel yedeklilik yöntemlerinin önemli bir dezavantajı, sistemin sürekli olarak yedekli modda çalışması gerekliliğidir. Bu durum, hata riski düşük olduğu dönemlerde bile kaynak israfına neden olmaktadır. İsteğe Bağlı Modüler Yedeklilik (On-Demand Modular Redundancy - ODMR) yaklaşımı, bu soruna çözüm sunmaktadır \cite{rogenmoser_odrg_2022}. ODMR'da sistem, ihtiyaca göre yedekli mod ile bağımsız çalışma modu arasında geçiş yapabilmektedir. Bu esneklik sayesinde, kritik görevlerde hata koruması sağlanırken normal koşullarda tam performans elde edilebilmektedir.

RISC-V komut seti mimarisi, açık kaynak ve telif ücretsiz yapısı ile hata toleransı araştırmaları için uygun bir platform sunmaktadır. Modüler yapısı, farklı koruma mekanizmalarının esnek biçimde entegre edilmesine olanak tanımaktadır.

%------------------------------------------------------------------------
\section{Tezin Amacı}\label{sec:tezin_amaci}
%------------------------------------------------------------------------

Bu tez çalışmasının temel amacı, çekirdek içi hata tespitinin avantajları ile isteğe bağlı yedekliliğin esnekliğini birleştiren bir RISC-V işlemci tasarlamaktır.

Çekirdek içi (intra-core) çözümler, hataların boruhattı seviyesinde tespit edilmesini ve erken müdahale edilmesini sağlamaktadır. ODMR yaklaşımı ise sistemin yedeklilik seviyesini dinamik olarak ayarlamasına imkan tanımaktadır. Bu tez, bu iki yaklaşımın avantajlarını tek bir mimaride birleştirmeyi hedeflemektedir.

Tasarım, ODMR'a uygun bir altyapı sunmak üzere planlanmıştır. Yedeklilik seviyesi olarak ise uzay, havacılık ve otomotiv sektörlerinde yaygın kullanılan ve hata maskeleme özelliğine sahip Üçlü Modüler Yedeklilik (Triple Modular Redundancy - TMR) tercih edilmiştir \cite{lyons_tmr}. TMR'ın hata maskeleme özelliği, tek bitlik hataların sistem çıkışını etkilemeden düzeltilmesini sağlamaktadır. Bu tercih doğrultusunda işlemci üç yollu süperölçekli ve sıra dışı yürütme özellikli olarak tasarlanmıştır.

Tasarlanan işlemci, çalışma moduna bağlı olarak üç farklı şekilde çalışabilmektedir. Süperölçekli modda üç bağımsız komut akışı paralel olarak yürütülerek yüksek performans sağlanmaktadır. Güvenli modda ise aynı komut üç kanalda eş zamanlı işlenerek hata toleransı elde edilmektedir. Bunlara ek olarak gerektiği durumda işlemci tek hattını kullanarak düşük güç modunda çalışabilmektedir. Modlar arası geçiş, boruhattı seviyesi uygulama sayesinde minimum gecikme ile gerçekleştirilebilmektedir.

Bu tez, daha önce bu tezin ön çalışması olarak yayınlanan çalışmamızda önerilen araştırma yönünün uygulamasını oluşturmaktadır. Bu çalışmada, ODMR yönteminin çip içinde uygulanması ile güvenli modda daha etkili hata düzeltme yapılabileceği, performans modunda ise daha çok uygulamaya destek verilebileceği önerilmiştir \cite{iskin_fault_tolerance_2024}.

%------------------------------------------------------------------------
\section{Tezin Katkıları}\label{sec:katkilar}
%------------------------------------------------------------------------

Bu tez çalışması, çekirdek içi (intra-core) hata tespitinin avantajları ile isteğe bağlı yedekliliğin (ODMR) esnekliğini birleştiren özgün bir işlemci mimarisi sunmaktadır. Çalışmanın temel katkıları şu şekilde özetlenebilir:

\begin{enumerate}
    \item ODMR'a Uygun Üç Yollu Süperölçekli Mimari: Tasarlanan işlemci, üç paralel yürütme kanalına sahip süperölçekli yapıda gerçeklenmiştir. Bu tasarım kararı, ODMR açısından kritik bir avantaj sunmaktadır: normal çalışmada üç kanal farklı komutları paralel işleyerek teorik olarak üç IPC değerine ulaşabilmekte, güvenli modda ise aynı üç kanal tek komutu eş zamanlı işleyerek TMR yapısını oluşturmaktadır. Yedeklilik seviyesi olarak TMR'ın tercih edilmesinin nedeni, uzay, havacılık ve otomotiv sektörlerinde kritik öneme sahip hata maskeleme özelliğidir.
    
    \item Minimal Ek Yük ile Boruhattı Seviyesinde Yedeklilik: Çekirdek seviyesi kilitli adım (lock-step) yöntemlerinden farklı olarak, uzaysal yedeklilik boruhattının içinde uygulanmıştır. İşlemcinin üç yollu süperölçekli olarak tasarlanması sebebiyle halihazırda birçok yapı üç eş olarak bulunacağı için güvenlik için gerekli ek maliyet düşük kalmıştır. TMR için gereken ek alan maliyeti yalnızca oylayıcı devreleriyle sınırlı kalmakta ve minimum düzeyde tutulmaktadır.
    
    \item Düşük Gecikmeli Mod Geçişi: Yedekliliğin boruhattı seviyesinde uygulanması, modlar arası geçiş gecikmesini minimize etmektedir. Süperölçekli moddan güvenli moda geçiş için boruhattının boşaltılması veya çekirdek durumunun senkronizasyonu gerekmemektedir. Yalnızca oylayıcı devrelerinin aktifleştirilmesi ve komut dağıtım mantığının değiştirilmesi yeterlidir.
    
    \item RAT Checkpoint ile Çift Amaçlı Toparlanma Mekanizması: RAT Checkpoint, yanlış dal tahmini durumunda işlemci durumunun tek çevrimde geri alınmasını sağlamak üzere tasarlanmıştır. On altı girişlik bu yapı, güvenli modda radyasyon kaynaklı kritik hatalardan toparlanma için de kullanılabilecek altyapıyı sağlayacak şekilde geliştirilmiştir. Bu sayede TMR ile maskelenemeyen çoklu bit hatalarının iş akışını etkilememesi için hızlı geri sarma mekanizmasına altyapı sunmaktadır.
\end{enumerate}

%------------------------------------------------------------------------
\section{Tezin Kapsamı}\label{sec:tezin_kapsami}
%------------------------------------------------------------------------

Bu tez çalışması aşağıdaki sınırlar dahilinde yürütülmüştür:

Komut Seti: RV32I taban komut seti desteklenmektedir. Çarpma, bölme ve kayan nokta uzantıları kapsam dışındadır.

Mikromimari: İşlemci üç yollu süperölçekli yapıda tasarlanmış olup, sıra dışı yürütme desteği sunmaktadır. Otuz iki mimari yazmaca ek olarak otuz iki yeniden adlandırma yazmacı olmak üzere toplam altmış dört fiziksel yazmaç ve on altı girişlik RAT checkpoint kullanılmaktadır.

Hata Toleransı: Kritik kontrol akışı elemanları TMR ile korunmaktadır. Büyük bellek yapıları için ECC koruması varsayılmakta olup, ECC'nin fiziksel gerçeklemesi kapsam dışındadır.

Doğrulama: Fonksiyonel doğrulama, birçok farklı akademik ve endüstriyel çalışma tarafından da kullanılan Google RISC-V DV çerçevesi ile gerçekleştirilmiş ve Berkeley Spike referans modeli ile doğrulanmıştır \cite{riscv_dv, spike_iss}. Hata enjeksiyonu testleri ile TMR mekanizmasının etkinliği ölçülmüştür.

Fiziksel Gerçekleme: Tasarım, TSMC 16nm FinFET teknolojisine sentezlenmiş ve yerleştirme-yönlendirme aşamalarından geçirilmiştir.

%------------------------------------------------------------------------
\section{Tezin Organizasyonu}\label{sec:organizasyon}
%------------------------------------------------------------------------

Bu tez altı bölümden oluşmaktadır.

Birinci bölüm tezin motivasyonunu, amacını, katkılarını, kapsamını ve organizasyonunu sunmaktadır.

İkinci bölüm temel kavramları ve literatür taramasını içermektedir. Hata toleransı yöntemleri, RISC-V mimarisi ve ilgili çalışmalar bu bölümde detaylı olarak açıklanmaktadır.

Üçüncü bölüm tasarlanan süperölçekli işlemcinin mikro mimarisini anlatmaktadır. Boruhattı aşamaları ayrı alt bölümler halinde ele alınmaktadır.

Dördüncü bölüm isteğe bağlı yedeklilik uygulamasını sunmaktadır. Çalışma modları ve kritik hata toparlanma mekanizması bu bölümde açıklanmaktadır.

Beşinci bölüm doğrulama metodolojisini, performans analizini, hata enjeksiyonu testlerini ve fiziksel tasarım sonuçlarını raporlamaktadır.

Altıncı bölüm sonuçları özetlemekte ve gelecek çalışma önerilerini sunmaktadır.
