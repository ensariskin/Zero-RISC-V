% =============================================================================
% PREAMBLE - RV32I 3-Way Superscalar İşlemci Tez Dokümantasyonu
% =============================================================================

\documentclass[12pt,a4paper,oneside]{report}

% -----------------------------------------------------------------------------
% Temel Paketler
% -----------------------------------------------------------------------------
\usepackage[utf8]{inputenc}
\usepackage[T1]{fontenc}
\usepackage[turkish]{babel}
\usepackage{lmodern}

% -----------------------------------------------------------------------------
% Sayfa Düzeni
% -----------------------------------------------------------------------------
\usepackage[
    top=2.5cm,
    bottom=2.5cm,
    left=3cm,
    right=2.5cm
]{geometry}
\usepackage{setspace}
\onehalfspacing

% -----------------------------------------------------------------------------
% Grafik ve Şekiller
% -----------------------------------------------------------------------------
\usepackage{graphicx}
\usepackage{tikz}
\usetikzlibrary{shapes.geometric, arrows.meta, positioning, calc, fit, backgrounds}

% Fix TikZ + Turkish babel conflict
\makeatletter
\AtBeginDocument{%
  \@ifpackageloaded{babel}{%
    \addto\extrasturkish{\shorthandoff{=}<>}%
  }{}%
}
\makeatother

% -----------------------------------------------------------------------------
% Tablolar
% -----------------------------------------------------------------------------
\usepackage{booktabs}
\usepackage{longtable}
\usepackage{multirow}
\usepackage{array}
\usepackage{tabularx}

% -----------------------------------------------------------------------------
% Kod Listeleme
% -----------------------------------------------------------------------------
\usepackage{listings}
\usepackage{xcolor}

\definecolor{codegreen}{rgb}{0,0.6,0}
\definecolor{codegray}{rgb}{0.5,0.5,0.5}
\definecolor{codepurple}{rgb}{0.58,0,0.82}
\definecolor{backcolour}{rgb}{0.95,0.95,0.92}

\lstdefinestyle{verilogstyle}{
    backgroundcolor=\color{backcolour},
    commentstyle=\color{codegreen},
    keywordstyle=\color{blue}\bfseries,
    numberstyle=\tiny\color{codegray},
    stringstyle=\color{codepurple},
    basicstyle=\ttfamily\footnotesize,
    breakatwhitespace=false,
    breaklines=true,
    captionpos=b,
    keepspaces=true,
    numbers=left,
    numbersep=5pt,
    showspaces=false,
    showstringspaces=false,
    showtabs=false,
    tabsize=2,
    language=Verilog,
    morekeywords={logic, always_ff, always_comb, posedge, negedge, module, endmodule,
                  input, output, reg, wire, parameter, localparam, assign, begin, end,
                  if, else, case, endcase, default, for, generate, endgenerate,
                  typedef, enum, struct, packed, unpacked, interface, endinterface,
                  modport, clocking, endclocking, assert, property, sequence},
    literate={ı}{{\i}}1 {İ}{{\.I}}1 {ğ}{{\u{g}}}1 {Ğ}{{\u{G}}}1
             {ş}{{\c{s}}}1 {Ş}{{\c{S}}}1 {ü}{{\"u}}1 {Ü}{{\"U}}1
             {ö}{{\"o}}1 {Ö}{{\"O}}1 {ç}{{\c{c}}}1 {Ç}{{\c{C}}}1
}

\lstset{style=verilogstyle}

% -----------------------------------------------------------------------------
% Matematik
% -----------------------------------------------------------------------------
\usepackage{amsmath}
\usepackage{amssymb}
\usepackage{amsfonts}

% -----------------------------------------------------------------------------
% Referanslar ve Linkler
% -----------------------------------------------------------------------------
\usepackage{hyperref}
\hypersetup{
    colorlinks=true,
    linkcolor=blue,
    filecolor=magenta,
    urlcolor=cyan,
    citecolor=green,
    pdftitle={RV32I 3-Way Superscalar İşlemci},
    pdfauthor={},
    bookmarks=true
}
\usepackage{cleveref}

% -----------------------------------------------------------------------------
% Başlık ve Alt Başlık Formatları
% -----------------------------------------------------------------------------
\usepackage{titlesec}

\titleformat{\chapter}[display]
{\normalfont\huge\bfseries}{\chaptertitlename\ \thechapter}{20pt}{\Huge}
\titlespacing*{\chapter}{0pt}{-20pt}{40pt}

% -----------------------------------------------------------------------------
% Üst Bilgi ve Alt Bilgi
% -----------------------------------------------------------------------------
\usepackage{fancyhdr}
\pagestyle{fancy}
\fancyhf{}
\fancyhead[L]{\leftmark}
\fancyhead[R]{\thepage}
\renewcommand{\headrulewidth}{0.4pt}

% -----------------------------------------------------------------------------
% Özel Komutlar
% -----------------------------------------------------------------------------
% Sinyal isimleri için
\newcommand{\sig}[1]{\texttt{#1}}

% Modül isimleri için
\newcommand{\module}[1]{\textbf{\texttt{#1}}}

% Kayıt isimleri için
\newcommand{\reg}[1]{\texttt{#1}}

% Bit aralığı için
\newcommand{\bits}[2]{[#1:#2]}

% Önemli kavramlar için
\newcommand{\concept}[1]{\textit{#1}}

% Neden kutusu
\usepackage{tcolorbox}
\tcbuselibrary{skins,breakable}

\newtcolorbox{nedenbox}[1][]{
    colback=blue!5!white,
    colframe=blue!75!black,
    fonttitle=\bfseries,
    title=Neden Bu Tasarım?,
    #1
}

\newtcolorbox{dikkatbox}[1][]{
    colback=orange!5!white,
    colframe=orange!75!black,
    fonttitle=\bfseries,
    title=Dikkat,
    #1
}

\newtcolorbox{ornekbox}[1][]{
    colback=green!5!white,
    colframe=green!75!black,
    fonttitle=\bfseries,
    title=Örnek,
    #1
}

% -----------------------------------------------------------------------------
% Kısaltmalar
% -----------------------------------------------------------------------------
\usepackage{acronym}

% -----------------------------------------------------------------------------
% Float Kontrolü
% -----------------------------------------------------------------------------
\usepackage{float}
\usepackage{placeins}

% -----------------------------------------------------------------------------
% Dipnot
% -----------------------------------------------------------------------------
\usepackage{footnote}

% -----------------------------------------------------------------------------
% Algoritma
% -----------------------------------------------------------------------------
\usepackage{algorithm}
\usepackage{algpseudocode}
\floatname{algorithm}{Algoritma}
\renewcommand{\algorithmicrequire}{\textbf{Girdi:}}
\renewcommand{\algorithmicensure}{\textbf{Çıktı:}}
