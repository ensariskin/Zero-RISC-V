% =============================================================================
% BÖLÜM 10: SONUÇ VE GELECEK ÇALIŞMALAR
% =============================================================================

\chapter{Sonuç ve Gelecek Çalışmalar}
\label{chap:sonuc}

\section{Sonuç}
\label{sec:sonuc}

Bu tez çalışmasında, RISC-V RV32I komut seti mimarisini destekleyen 3-way
superscalar bir işlemci tasarlanmış ve gerçeklenmiştir. Tomasulo algoritması
temel alınarak out-of-order execution, register renaming ve spekülatif
branch execution mekanizmaları başarıyla implemente edilmiştir.

\subsection{Başarılan Hedefler}

\begin{enumerate}
    \item \textbf{3-Way Superscalar Mimari:}
    \begin{itemize}
        \item Her çevrimde 3 komut fetch, decode, issue ve commit
        \item Scalar işlemciye göre 1.83x ortalama hızlanma
        \item Tam RV32I komut seti desteği
    \end{itemize}

    \item \textbf{Tomasulo Algoritması Implementasyonu:}
    \begin{itemize}
        \item Register Alias Table (RAT) ile register renaming
        \item Tag-based operand tracking ile dependency resolution
        \item Common Data Bus (CDB) ile result broadcasting
        \item Reorder Buffer (ROB) ile in-order commit
    \end{itemize}

    \item \textbf{Spekülatif Execution:}
    \begin{itemize}
        \item Tournament branch predictor (Gshare + Bimodal)
        \item Return Address Stack (RAS) ile fonksiyon dönüş tahmini
        \item JALR predictor ile indirect jump tahmini
        \item BRAT ile eager misprediction recovery
    \end{itemize}

    \item \textbf{Memory Subsystem:}
    \begin{itemize}
        \item Load Store Queue (LSQ) ile memory operasyonu yönetimi
        \item In-order store commit ile memory consistency
        \item 3-port memory interface
    \end{itemize}
\end{enumerate}

\subsection{Teknik Katkılar}

Bu çalışmanın temel teknik katkıları şunlardır:

\begin{enumerate}
    \item \textbf{BRAT Mekanizması:}
    Branch Resolution Alias Table, misprediction recovery latency'sini
    minimize eden eager recovery mekanizması sağlar. ROB head'e ulaşmadan
    anında RAT restore yapılabilir.

    \item \textbf{Circular Buffer Tabanlı Kaynak Yönetimi:}
    ROB, BRAT ve LSQ için tek bir circular buffer yapısı kullanılarak
    alan ve karmaşıklık optimize edilmiştir.

    \item \textbf{Same-Cycle Forwarding:}
    Aynı çevrimde issue edilen bağımlı komutlar arasında kombinasyonel
    forwarding ile IPC kaybı önlenmiştir.

    \item \textbf{Modüler ve Genişletilebilir Tasarım:}
    SystemVerilog interface'leri ile modüller arası temiz ayrım sağlanmış,
    gelecek genişletmeler için uygun altyapı oluşturulmuştur.
\end{enumerate}

\subsection{Performans Sonuçları}

\begin{table}[H]
\centering
\caption{Performans özeti}
\label{tab:final_perf}
\begin{tabular}{lc}
\toprule
\textbf{Metrik} & \textbf{Değer} \\
\midrule
Issue Genişliği & 3-way \\
Ortalama IPC & 1.5-2.0 \\
Speedup (vs Scalar) & 1.83x \\
ROB Derinliği & 32 entry \\
Branch Mispred Rate & 5-10\% \\
Mispred Penalty & 5-6 cycle \\
\bottomrule
\end{tabular}
\end{table}

\section{Karşılaşılan Zorluklar}
\label{sec:challenges}

Tasarım sürecinde karşılaşılan başlıca zorluklar:

\begin{enumerate}
    \item \textbf{Timing Closure:}
    3-way paralel operasyonlar için kritik yol optimizasyonu gerekti.
    Özellikle RAT lookup ve same-cycle forwarding kombinasyonel
    gecikme ekledi.

    \item \textbf{Misprediction Recovery Karmaşıklığı:}
    BRAT snapshot'larının commit update'leri ile senkronizasyonu,
    köşe durumlarında hatalara yol açtı. Dikkatli tasarım ve
    kapsamlı test ile çözüldü.

    \item \textbf{LSQ-ROB Koordinasyonu:}
    Store commit'in ROB ve LSQ arasında doğru senkronizasyonu,
    özellikle misprediction durumlarında zorlu oldu.

    \item \textbf{Debug Zorluğu:}
    Out-of-order execution, geleneksel debug tekniklerini zorlaştırdı.
    Tracer altyapısı bu sorunu çözmek için geliştirildi.
\end{enumerate}

\section{Gelecek Çalışmalar}
\label{sec:future_work}

\subsection{Kısa Vadeli İyileştirmeler}

\begin{enumerate}
    \item \textbf{Daha Derin Reservation Station:}
    Her RS'yi 1 entry'den 2-4 entry'ye genişletmek, instruction window'u
    artırarak IPC iyileştirmesi sağlayabilir.

    \item \textbf{Partial Store-to-Load Forwarding:}
    Farklı boyutlu store-load arası kısmi byte forwarding desteği eklenebilir.

    \item \textbf{TAGE Branch Predictor:}
    Tournament predictor yerine TAGE kullanmak, misprediction rate'i
    düşürebilir.

    \item \textbf{Daha Geniş Fetch:}
    Fetch genişliğini 4-6 komuta çıkarmak, instruction supply darboğazını
    azaltabilir.
\end{enumerate}

\subsection{Orta Vadeli İyileştirmeler}

\begin{enumerate}
    \item \textbf{RV32M Extension:}
    Multiply/Divide komutları için ayrı functional unit eklemek.

    \item \textbf{Speculative Load Execution:}
    Address bilinmeyen store'ları bypass eden spekülatif load execution.

    \item \textbf{Cache Hierarchy:}
    L1 I-Cache ve D-Cache eklenmesi.

    \item \textbf{Exception Handling:}
    Tam RISC-V exception/interrupt desteği.
\end{enumerate}

\subsection{Uzun Vadeli Hedefler}

\begin{enumerate}
    \item \textbf{SMT (Simultaneous Multi-Threading):}
    Tek çekirdekte birden fazla thread çalıştırma.

    \item \textbf{RV64 Desteği:}
    64-bit veri yolu ve adres alanı.

    \item \textbf{Floating Point:}
    RV32F/RV32D extension desteği.

    \item \textbf{Vector Extension:}
    RV32V ile SIMD operasyonları.

    \item \textbf{Multi-Core:}
    Birden fazla superscalar çekirdek ile cache coherency.
\end{enumerate}

\section{Öğrenilen Dersler}
\label{sec:lessons}

Bu çalışmadan çıkarılan önemli dersler:

\begin{enumerate}
    \item \textbf{Basitlik Önce Gelir:}
    Karmaşık optimizasyonlar yerine çalışan basit tasarımdan başlamak,
    debug sürecini önemli ölçüde kolaylaştırdı.

    \item \textbf{Kapsamlı Test Kritik:}
    Out-of-order execution'da köşe durumları çok fazla. Kapsamlı
    unit test ve regression test olmadan güvenilir tasarım mümkün değil.

    \item \textbf{Tracer Altyapısı Zorunlu:}
    Instruction-level izleme olmadan out-of-order debug neredeyse
    imkansız. Tracer, tasarım sürecinin başında eklenmeliydi.

    \item \textbf{Modüler Tasarım:}
    SystemVerilog interface'leri ile modüller arası temiz ayrım,
    hem geliştirme hem de debug sürecini hızlandırdı.
\end{enumerate}

\section{Son Söz}

Bu tez çalışması, modern superscalar işlemci tasarımının temel
prensiplerini başarıyla uygulamış ve çalışan bir 3-way superscalar
RV32I işlemci ortaya koymuştur. Tomasulo algoritması, spekülatif
execution ve branch prediction mekanizmalarının entegrasyonu,
scalar işlemciye göre önemli performans artışı sağlamıştır.

Tasarım, hem eğitim amaçlı referans işlemci olarak hem de gömülü
sistem uygulamaları için pratik bir çözüm olarak kullanılabilir.
Modüler yapısı, gelecek genişletmeler için sağlam bir temel
oluşturmaktadır.
