% =============================================================================
% BÖLÜM 1: GİRİŞ
% =============================================================================

\chapter{Giriş}
\label{chap:giris}

\section{Motivasyon}
\label{sec:motivasyon}

Modern bilgisayar sistemlerinin performans gereksinimleri sürekli artmaktadır. Tek çevrimde
tek komut işleyen geleneksel skalar işlemciler, bu artan talepleri karşılamakta yetersiz
kalmaktadır. İşlemci frekansını artırmanın fiziksel sınırlarına (güç tüketimi, ısı dağılımı)
ulaşılmasıyla birlikte, performans artışı için farklı yaklaşımlar gerekli hale gelmiştir.

\concept{Superscalar} işlemciler, tek bir çevrimde birden fazla komutun paralel olarak
işlenmesine olanak tanıyarak bu soruna çözüm sunar. Bu yaklaşım, \concept{Komut Düzeyinde
Paralellik} (Instruction-Level Parallelism - ILP) kavramını kullanarak, programdaki
bağımsız komutları eşzamanlı yürütür.

Bu tez çalışmasında, RISC-V RV32I komut seti mimarisini destekleyen, 3 yollu (3-way)
superscalar bir işlemci tasarımı ve gerçeklemesi sunulmaktadır. Tasarım, Tomasulo
algoritmasını temel alarak sıra dışı (out-of-order) yürütme yeteneği sunarken,
spekülatif dal tahmini ile performansı maksimize etmeyi hedeflemektedir.

\section{Tezin Kapsamı}
\label{sec:kapsam}

Bu çalışma aşağıdaki konuları kapsamaktadır:

\begin{itemize}
    \item \textbf{RV32I Komut Seti Desteği:} RISC-V temel tamsayı komut setinin tam
          gerçeklemesi (37 komut)

    \item \textbf{3-Way Superscalar Mimari:} Her çevrimde en fazla 3 komutun paralel
          olarak decode, issue, dispatch ve commit edilmesi

    \item \textbf{Sıra Dışı Yürütme:} Tomasulo algoritması tabanlı dinamik zamanlama
          ile veri bağımlılıklarının donanım seviyesinde çözümlenmesi

    \item \textbf{Spekülatif Dal Tahmini:} Tournament, GShare ve 2-bit sayaç tabanlı
          dal tahmin mekanizmaları ile JALR hedef tahmini

    \item \textbf{Eager Misprediction Recovery:} Branch Resolution Alias Table (BRAT)
          ile sıfır gecikmeli yanlış tahmin düzeltmesi

    \item \textbf{Load/Store Queue:} Bellek operasyonlarının sıralı commit ile
          sıra dışı yürütülmesi
\end{itemize}

\section{Tasarım Hedefleri}
\label{sec:hedefler}

İşlemci tasarımında aşağıdaki hedefler gözetilmiştir:

\begin{enumerate}
    \item \textbf{Yüksek IPC (Instructions Per Cycle):} Teorik maksimum 3.0 IPC hedefine
          yaklaşmak için pipeline verimliliğinin maksimize edilmesi

    \item \textbf{Düşük Dal Cezası:} Yanlış tahmin durumunda minimum çevrim kaybı
          için eager recovery mekanizması

    \item \textbf{Ölçeklenebilirlik:} Parametrik tasarım ile farklı konfigürasyonlara
          kolayca uyarlanabilirlik

    \item \textbf{Doğrulanabilirlik:} Modüler yapı ve kapsamlı assertion'lar ile
          fonksiyonel doğrulama kolaylığı

    \item \textbf{Sentezlenebilirlik:} FPGA ve ASIC hedefleri için uygun RTL tasarımı
\end{enumerate}

\section{Tez Organizasyonu}
\label{sec:organizasyon}

Bu tez aşağıdaki şekilde organize edilmiştir:

\begin{description}
    \item[Bölüm 2 - Mimari Genel Bakış:] İşlemcinin genel mimarisi, pipeline yapısı
         ve temel tasarım prensipleri

    \item[Bölüm 3 - Fetch Stage:] Komut getirme aşaması, çoklu komut fetch, dal
         tahmini ve program sayacı yönetimi

    \item[Bölüm 4 - Issue Stage:] Komut decode, register renaming, kaynak tahsisi
         ve BRAT mekanizması

    \item[Bölüm 5 - Dispatch Stage:] Reorder Buffer, Reservation Station ve
         Register File yapıları

    \item[Bölüm 6 - Execute Stage:] ALU, shifter ve dal çözümleme birimleri

    \item[Bölüm 7 - Memory Stage:] Load/Store Queue ve bellek erişim yönetimi

    \item[Bölüm 8 - Performans Analizi:] Benchmark sonuçları ve karşılaştırmalı
         analiz

    \item[Bölüm 9 - Doğrulama Stratejisi:] Test metodolojisi ve doğrulama
         yaklaşımları

    \item[Bölüm 10 - Sonuç:] Çalışmanın özeti ve gelecek çalışmalar
\end{description}

\section{Katkılar}
\label{sec:katkilar}

Bu tez çalışmasının başlıca katkıları şunlardır:

\begin{enumerate}
    \item \textbf{BRAT Mekanizması:} Spekülatif dallanma için düşük gecikmeli
          recovery sağlayan yeni bir RAT snapshot yönetim yapısı

    \item \textbf{3-Port Circular Buffer:} ROB ve LSQ kaynak tahsisi için verimli
          bir paralel allocation/deallocation mekanizması

    \item \textbf{Combinational Bypass:} Dal çözümleme sonuçlarının aynı çevrimde
          tüm pipeline aşamalarına iletilmesi

    \item \textbf{Tournament Predictor:} Yerel ve global dal geçmişini birleştiren
          hibrit tahmin mekanizması

    \item \textbf{Açık Kaynak Gerçekleme:} Eğitim ve araştırma amaçlı kullanılabilecek
          tam dokümante edilmiş SystemVerilog kodu
\end{enumerate}
