% =============================================================================
% RAT - REGISTER ALIAS TABLE
% =============================================================================

\section{Register Alias Table (RAT)}
\label{sec:rat}

Bu bölümde, Tomasulo algoritmasının temel bileşeni olan Register Alias Table (RAT)
yapısı detaylı olarak açıklanmaktadır.

\subsection{Tasarım Amacı}

\concept{Register renaming}, WAW (Write-After-Write) ve WAR (Write-After-Read)
hazard'larını ortadan kaldırmak için kullanılır. RAT, mimari register'ları
(x0-x31) fiziksel register'lara (0-63) eşler.

\begin{nedenbox}
\textbf{Neden register renaming gerekli?}

Aşağıdaki kod parçasını düşünün:
\begin{lstlisting}
ADD x1, x2, x3    // I1: x1'e yaz
SUB x4, x1, x5    // I2: x1'i oku (RAW - gerçek bağımlılık)
MUL x1, x6, x7    // I3: x1'e yaz (WAW - I1 ile)
AND x8, x1, x9    // I4: x1'i oku (RAW - I3 ile)
\end{lstlisting}

Renaming olmadan:
\begin{itemize}
    \item I3, I1 bitene kadar beklemeli (WAW)
    \item I4, I3 bitene kadar beklemeli
\end{itemize}

Renaming ile:
\begin{itemize}
    \item I1 → p32, I3 → p33 (farklı fiziksel register)
    \item I1 ve I3 paralel yürütülebilir
    \item Sadece gerçek RAW bağımlılıkları kalır
\end{itemize}
\end{nedenbox}

\subsection{RAT Yapısı}

\begin{table}[H]
\centering
\caption{RAT parametreleri}
\label{tab:rat_params}
\begin{tabular}{lcp{6cm}}
\toprule
\textbf{Parametre} & \textbf{Değer} & \textbf{Açıklama} \\
\midrule
ARCH\_REGS & 32 & Mimari register sayısı (x0-x31) \\
PHYS\_REGS & 64 & Fiziksel register sayısı \\
PHYS\_ADDR\_WIDTH & 6 bit & Fiziksel register adresi genişliği \\
\bottomrule
\end{tabular}
\end{table}

\paragraph{Fiziksel Register Alanı}

64 fiziksel register iki bölgeye ayrılır:
\begin{itemize}
    \item \textbf{0-31:} Register File (RF) - commit edilmiş değerler
    \item \textbf{32-63:} Reorder Buffer (ROB) - in-flight değerler
\end{itemize}

\begin{nedenbox}
\textbf{Neden 64 fiziksel register?}

32 mimari register + 32 ROB entry = 64 fiziksel register.
\begin{itemize}
    \item Her in-flight komut için 1 ROB entry gerekli
    \item 32 ROB entry, 32 komut paralel yürütme kapasitesi sağlar
    \item 3-way superscalar için bu yeterli buffer derinliği
\end{itemize}
\end{nedenbox}

\subsection{RAT Operasyonları}

\subsubsection{Kaynak Register Lookup}

Kaynak register'lar (rs1, rs2) için mevcut mapping okunur:

\begin{lstlisting}[caption={Kaynak register lookup}]
// Direct RAT lookup
assign rs1_phys_0 = rat_table[rs1_arch_0];
assign rs2_phys_0 = rat_table[rs2_arch_0];

// Same-cycle forwarding for dependent instructions
assign rs1_phys_1 = rs1_arch_1_equal_rd_arch_0 ? rd_phys_0 : rat_table[rs1_arch_1];
assign rs2_phys_1 = rs2_arch_1_equal_rd_arch_0 ? rd_phys_0 : rat_table[rs2_arch_1];

assign rs1_phys_2 = rs1_arch_2_equal_rd_arch_1 ? rd_phys_1 :
                    rs1_arch_2_equal_rd_arch_0 ? rd_phys_0 : rat_table[rs1_arch_2];
\end{lstlisting}

\begin{nedenbox}
\textbf{Neden same-cycle forwarding?}

Aynı çevrimde issue edilen 3 komut arasında bağımlılık olabilir:
\begin{lstlisting}
ADD x1, x2, x3    // Inst 0: x1'e yaz
SUB x4, x1, x5    // Inst 1: x1'i oku (Inst 0'a bağımlı)
\end{lstlisting}

RAT tablosu henüz güncellenmedi. Forwarding olmadan Inst 1, eski x1 mapping'ini
görür. Same-cycle forwarding, Inst 0'ın yeni rd\_phys değerini Inst 1'e iletir.
\end{nedenbox}

\subsubsection{Hedef Register Allocation}

Hedef register (rd) için yeni fiziksel register allocate edilir:

\begin{lstlisting}[caption={Hedef register allocation}]
always_comb begin
    // Instruction 0
    if (need_alloc_0 && found_first) begin
        allocated_phys_reg[0] = first_free;
        allocation_success[0] = 1'b1;
    end

    // Instruction 1
    if (need_alloc_1 && found_second) begin
        allocated_phys_reg[1] = second_free;
        allocation_success[1] = 1'b1;
    end

    // Instruction 2
    if (need_alloc_2 && found_third) begin
        allocated_phys_reg[2] = third_free;
        allocation_success[2] = 1'b1;
    end
end
\end{lstlisting}

\subsubsection{RAT Güncelleme}

Başarılı allocation sonrası RAT tablosu güncellenir:

\begin{lstlisting}[caption={RAT güncelleme}]
always_ff @(posedge clk) begin
    // Rename: Update RAT for new allocations
    if (need_alloc_0 && rd_arch_0 != 0) begin
        rat_table[rd_arch_0] <= allocated_phys_reg[0];
    end
    if (need_alloc_1 && rd_arch_1 != 0) begin
        rat_table[rd_arch_1] <= allocated_phys_reg[1];
    end
    if (need_alloc_2 && rd_arch_2 != 0) begin
        rat_table[rd_arch_2] <= allocated_phys_reg[2];
    end
end
\end{lstlisting}

\begin{nedenbox}
\textbf{Neden \texttt{rd\_arch != 0} kontrolü?}

RISC-V'de x0 register'ı sabit sıfırdır ve yazılamaz. x0'a yapılan yazmalar
görmezden gelinir. RAT'ta x0 her zaman fiziksel register 0'a map edilir.
\end{nedenbox}

\subsubsection{Commit İşleme}

ROB commit olduğunda, değer RF'e yazılır ve RAT güncellenir:

\begin{lstlisting}[caption={Commit işleme}]
// Commit: Restore architectural register to RF mapping
if (commit_valid[0] && commit_addr_0 != 0) begin
    if (commit_rob_idx_0 == rat_table[commit_addr_0][4:0]) begin
        rat_table[commit_addr_0] <= {1'b0, commit_addr_0};
    end
end
\end{lstlisting}

\begin{nedenbox}
\textbf{Neden ROB indeksi karşılaştırması?}

Aynı mimari register için birden fazla in-flight yazma olabilir:
\begin{lstlisting}
ADD x1, x2, x3    // ROB[5]: x1'e yaz
MUL x1, x4, x5    // ROB[8]: x1'e yaz
\end{lstlisting}

ADD commit olduğunda, x1 hâlâ MUL'un sonucuna (ROB[8]) bağlıdır.
RAT'ı RF'e döndürmek yanlış olur. Karşılaştırma, sadece ``en son yazma''
commit olduğunda RF mapping'e dönmeyi sağlar.
\end{nedenbox}

\subsection{3-Way Paralel Renaming}

3 komut aynı anda rename edilir. Bu, karmaşık bağımlılık kontrolü gerektirir:

\begin{lstlisting}[caption={3-way bağımlılık kontrolü}]
// Instruction 1 depends on Instruction 0?
assign rs1_arch_1_equal_rd_arch_0 = (rs1_arch_1 == rd_arch_0) &&
    (rd_arch_0 != 5'h0) && decode_valid[0] && rd_write_enable_0;

// Instruction 2 depends on Instruction 0 or 1?
assign rs1_arch_2_equal_rd_arch_0 = (rs1_arch_2 == rd_arch_0) &&
    (rd_arch_0 != 5'h0) && decode_valid[0] && rd_write_enable_0;
assign rs1_arch_2_equal_rd_arch_1 = (rs1_arch_2 == rd_arch_1) &&
    (rd_arch_1 != 5'h0) && decode_valid[1] && rd_write_enable_1;
\end{lstlisting}

\subsection{Kaynak Yönetimi}

RAT, ROB ve LSQ kaynak tahsisini de yönetir. Bu mekanizmalar aşağıdaki alt
bölümlerde detaylı olarak açıklanmaktadır.

% =============================================================================
% CIRCULAR BUFFER - ROB VE LSQ ALLOCATION
% =============================================================================

\subsection{3-Port Circular Buffer ile Kaynak Allocation}
\label{sec:circular_buffer}

Bu bölümde, ROB ve LSQ kaynak tahsisi için kullanılan 3-port circular buffer
yapısı detaylı olarak açıklanmaktadır.

\subsubsection{Problem: 3-Way Paralel Allocation}

3-way superscalar mimaride, her çevrimde en fazla 3 komut issue edilir. Her komut
potansiyel olarak şu kaynakları gerektirir:
\begin{itemize}
    \item 1 ROB entry (fiziksel register = ROB indeksi)
    \item 1 LSQ entry (load/store komutları için)
\end{itemize}

Geleneksel \concept{free list} yaklaşımında, 3 bağımsız boş kaynak bulmak karmaşık
priority encoder mantığı gerektirir. N entry'li bir free list için:
\begin{itemize}
    \item İlk boş entry'yi bul: O(N) tarama veya priority encoder
    \item İkinci boş entry'yi bul: O(N) tarama, ilkini hariç tut
    \item Üçüncü boş entry'yi bul: O(N) tarama, ilk ikisini hariç tut
\end{itemize}

Bu yaklaşım hem alan hem de zamanlama açısından maliyetlidir.

\begin{nedenbox}
\textbf{Neden geleneksel free list yetersiz?}
\begin{itemize}
    \item \textbf{Donanım Maliyeti:} 3 bağımsız priority encoder, her biri 32-64 bit
    \item \textbf{Critical Path:} Birinci sonuç ikinciye, ikinci üçüncüye bağımlı
    \item \textbf{Karmaşıklık:} Misprediction recovery için tüm allocation'ları track etmek gerekir
\end{itemize}
\end{nedenbox}

\subsubsection{Çözüm: Index-as-Value Circular Buffer}

Bu tasarımda, circular buffer'ın her entry'sinin değeri kendi indeksine eşittir:

\begin{lstlisting}[caption={Index-as-value circular buffer yapısı}]
// Buffer initialization
for (int i = 0; i < BUFFER_DEPTH; i++) begin
    buffer[i] = i;  // Entry[0]=0, Entry[1]=1, ..., Entry[31]=31
end
\end{lstlisting}

\paragraph{Temel Fikir}

Buffer, gerçek veri depolamaz. Sadece hangi indekslerin ``kullanılabilir'' olduğunu
yönetir:
\begin{itemize}
    \item \sig{read\_ptr}: Bir sonraki allocation'ın yapılacağı pozisyon
    \item \sig{write\_ptr}: Deallocation yapıldığında kullanılacak pozisyon
    \item \sig{count}: Mevcut kullanılabilir entry sayısı
\end{itemize}

\paragraph{Allocation (Okuma)}

Allocation, \sig{read\_ptr}'dan okuma ile yapılır:

\begin{lstlisting}[caption={3-port paralel allocation}]
// 3 parallel reads
assign read_data_0 = read_ptr;           // Allocated ID = read_ptr
assign read_data_1 = read_ptr + 1;       // Next ID
assign read_data_2 = read_ptr + 2;       // Next+1 ID

// Advance pointer by number of successful allocations
always_ff @(posedge clk) begin
    if (read_en_0 || read_en_1 || read_en_2)
        read_ptr <= read_ptr + read_count;
end
\end{lstlisting}

\paragraph{Deallocation (Yazma)}

Deallocation, \sig{write\_ptr}'a yazma ile yapılır (değer zaten sabit):

\begin{lstlisting}[caption={3-port paralel deallocation}]
// 3 parallel writes (from commit)
always_ff @(posedge clk) begin
    if (write_en_0 || write_en_1 || write_en_2)
        write_ptr <= write_ptr + write_count;
end
\end{lstlisting}

\begin{nedenbox}
\textbf{Neden bu tasarım üstün?}
\begin{itemize}
    \item \textbf{O(1) Complexity:} Priority encoder yok, sadece pointer aritmetiği
    \item \textbf{Paralel Erişim:} 3 allocation aynı anda, bağımsız olarak
    \item \textbf{Sıfır Depolama:} Gerçek veri saklanmaz, sadece pointer'lar
    \item \textbf{Basit Recovery:} Pointer reset ile tüm allocation'lar geri alınır
\end{itemize}
\end{nedenbox}

\subsubsection{Free List Yönetimi (ROB Allocation)}

ROB allocation için 32 entry'lik circular buffer kullanılır:

\begin{lstlisting}[caption={Free address buffer instantiation}]
circular_buffer_3port free_address_buffer(
    .clk(clk),
    .rst_n(reset),
    .redo_last_alloc(|branch_mispredicted_o),
    .read_en_0(need_alloc_0),
    .read_en_1(need_alloc_1),
    .read_en_2(need_alloc_2),
    .read_data_0(first_free),
    .read_data_1(second_free),
    .read_data_2(third_free),
    .read_valid_0(found_first),
    .read_valid_1(found_second),
    .read_valid_2(found_third),
    .write_en_0(commit_valid[0]),
    .write_en_1(commit_valid[1]),
    .write_en_2(commit_valid[2]),
    .set_read_ptr_en(free_addr_set_en),
    .set_read_ptr_value(free_addr_set_value)
);
\end{lstlisting}

\paragraph{ROB Allocation neden RAT'ta?}

Bu tasarımda, \concept{fiziksel register = ROB indeksi} eşitliği kullanılır:
\begin{itemize}
    \item Mimari register'lar (x0-x31): Fiziksel register 0-31 (RF'te)
    \item ROB entry'leri (0-31): Fiziksel register 32-63 (ROB'da)
\end{itemize}

RAT zaten register renaming yapıyor. Yeni hedef için ROB ID allocation doğal
olarak renaming sürecinin parçasıdır.

\subsubsection{LSQ Index Allocation}

Load/store komutları için ayrı bir circular buffer kullanılır:

\begin{lstlisting}[caption={LSQ address buffer instantiation}]
circular_buffer_3port #(.BUFFER_DEPTH(32)) lsq_address_buffer(
    .clk(clk),
    .rst_n(reset),
    .redo_last_alloc(|branch_mispredicted_o),
    .read_en_0(need_lsq_alloc_0),
    .read_en_1(need_lsq_alloc_1),
    .read_en_2(need_lsq_alloc_2),
    .write_en_0(lsq_commit_0),
    .write_en_1(lsq_commit_1),
    .write_en_2(lsq_commit_2),
    .set_read_ptr_en(lsq_flush_valid_i),
    .set_read_ptr_value(first_invalid_lsq_idx_i)
);
\end{lstlisting}

\begin{nedenbox}
\textbf{Neden ayrı LSQ buffer?}

LSQ allocation sadece load/store komutları için gerekli. Ayrı buffer tutmak:
\begin{itemize}
    \item ROB ve LSQ yaşam döngülerini bağımsız yönetir
    \item Her yapı kendi hızında dolup boşalabilir
    \item Misprediction recovery ayrı ayrı yapılabilir
\end{itemize}
\end{nedenbox}

\subsubsection{Misprediction Recovery}

Misprediction durumunda, yanlış yolda yapılan allocation'lar geri alınmalıdır.
İki mekanizma kullanılır:

\paragraph{1. Pointer Reset (\sig{set\_read\_ptr\_en})}

Misprediction tespit edildiğinde, \sig{read\_ptr} mispredicting instruction'ın
allocation noktasına reset edilir:

\begin{lstlisting}[caption={Misprediction pointer reset}]
always_comb begin
    if (brat_resolved_0 && brat_mispredicted_0) begin
        free_addr_set_en = 1'b1;
        free_addr_set_value = brat_resolved_phys_0 + 1;
    end else if (brat_resolved_1 && brat_mispredicted_1) begin
        free_addr_set_en = 1'b1;
        free_addr_set_value = brat_resolved_phys_1 + 1;
    end else ...
end
\end{lstlisting}

\begin{nedenbox}
\textbf{Neden \texttt{+1}?}

Mispredicting branch'in kendi allocation'ı geçerlidir. Sadece ondan sonraki
allocation'lar geri alınmalı. Bu yüzden yeni \sig{read\_ptr} = branch'in
fiziksel register'ı + 1.
\end{nedenbox}

\paragraph{2. Redo Last Allocation (\sig{redo\_last\_alloc})}

Misprediction aynı çevrimde tespit edilirse, o çevrimdeki allocation'lar henüz
commit edilmemiştir. Bu sinyal, son allocation'ı geri alır:

\begin{lstlisting}[caption={Redo last allocation}]
always_ff @(posedge clk) begin
    if (redo_last_alloc) begin
        read_ptr <= read_ptr - last_alloc_count;
    end
end
\end{lstlisting}

\subsubsection{Buffer Doluluk Kontrolü}

Allocation'a hazır olup olmadığını belirleyen sinyaller:

\begin{lstlisting}[caption={Rename ready sinyalleri}]
// ROB allocation ready
assign rename_ready = (free_count >= 3) ? 3'b111 :
                      (free_count == 2) ? 3'b011 :
                      (free_count == 1) ? 3'b001 : 3'b000;

// LSQ allocation ready
assign lsq_alloc_ready = (lsq_free_count >= 3) ? 3'b111 :
                         (lsq_free_count == 2) ? 3'b011 :
                         (lsq_free_count == 1) ? 3'b001 : 3'b000;
\end{lstlisting}

Bu sinyaller, issue stage'e kaç komutun kabul edilebileceğini bildirir.

\subsubsection{Circular Buffer Özet Tablosu}

\begin{table}[H]
\centering
\caption{Circular buffer özellikleri}
\label{tab:circular_buffer}
\begin{tabular}{lcc}
\toprule
\textbf{Özellik} & \textbf{Free List (ROB)} & \textbf{LSQ Buffer} \\
\midrule
Derinlik & 32 entry & 32 entry \\
Port Sayısı & 3 read, 3 write & 3 read, 3 write \\
Allocation & Issue aşamasında & Load/store issue'da \\
Deallocation & ROB commit'te & LSQ commit'te \\
Reset Kaynağı & BRAT misprediction & BRAT misprediction \\
\bottomrule
\end{tabular}
\end{table}


% =============================================================================
% BRAT - BRANCH RESOLUTION ALIAS TABLE
% =============================================================================

\subsection{Branch Resolution Alias Table (BRAT)}
\label{sec:brat}

Bu bölümde, spekülatif dal tahmini recovery'si için kullanılan BRAT mekanizması
detaylı olarak açıklanmaktadır.

\subsubsection{Problem: Geleneksel Misprediction Recovery}

Geleneksel Tomasulo tabanlı işlemcilerde, misprediction recovery şu şekilde çalışır:

\begin{enumerate}
    \item Dal komutu execute edilir, misprediction tespit edilir
    \item Dal komutu ROB başına ulaşana kadar beklenir
    \item ROB başında recovery başlar: RAT sıfırlanır, pipeline flush edilir
    \item Doğru yoldan fetch yeniden başlar
\end{enumerate}

\begin{figure}[H]
\centering
\fbox{
\begin{minipage}{0.85\textwidth}
\centering
\textbf{Geleneksel Recovery}\\[0.5em]
ROB: \fbox{0 (Head)} \fbox{1} \fbox{\textcolor{red}{2 (Mispred)}} \fbox{3} \fbox{4} \fbox{5} \fbox{6} \fbox{7}\\[0.5em]
\textit{Bekleme: Mispredicting branch (2) ROB head'e ulaşmalı $\rightarrow$ 2 commit bekle}
\end{minipage}
}
\caption{Geleneksel recovery: Dal ROB head'e ulaşmalı}
\label{fig:traditional_recovery}
\end{figure}

\begin{nedenbox}
\textbf{Neden geleneksel yaklaşım yavaş?}

Mispredicting dal ROB'un ortasındaysa, önündeki tüm komutların commit olması
beklenir. 32 entry'lik ROB'da, dal 16. pozisyondaysa, 15 commit beklenir.
Her commit 1 çevrimde 3 komut işlese bile, bu 5+ çevrim gecikme demektir.
Spekülatif dallar için bu gecikme kabul edilemez.
\end{nedenbox}

\subsubsection{Çözüm: BRAT ile Eager Recovery}

BRAT, her dal komutu için RAT'ın anlık görüntüsünü (snapshot) saklar. Misprediction
tespit edildiğinde, ROB başı beklenmeden anında geri yükleme yapılır.

\begin{figure}[H]
\centering
\fbox{
\begin{minipage}{0.85\textwidth}
\centering
\textbf{BRAT: Snapshot Tabanlı Anında Recovery}\\[1em]
\begin{tabular}{ccc}
\fbox{Branch 0 Snapshot} & \fbox{Branch 1 Snapshot} & \fbox{Branch 2 Snapshot} \\
\scriptsize (Head) & & \scriptsize (Tail) \\
\end{tabular}
\\[0.5em]
$\uparrow$ Execute Result (Match?) \\[0.5em]
$\downarrow$ Mispred $\rightarrow$ RAT Restore \\[0.5em]
\textit{ROB head beklenmeden anında recovery}
\end{minipage}
}
\caption{BRAT: Snapshot tabanlı anında recovery}
\label{fig:brat_recovery}
\end{figure}

\subsubsection{BRAT Entry Yapısı}

Her BRAT entry'si şu alanları saklar:

\begin{table}[H]
\centering
\caption{BRAT entry alanları}
\label{tab:brat_entry}
\begin{tabular}{llp{6cm}}
\toprule
\textbf{Alan} & \textbf{Boyut} & \textbf{Açıklama} \\
\midrule
\sig{branch\_phys} & 6 bit & Branch'in fiziksel register ID'si (ROB ID) \\
\sig{rat\_snapshot} & 32×6 bit & Tüm RAT mapping'inin kopyası \\
\sig{resolved} & 1 bit & Branch execute edildi mi? \\
\sig{mispredicted} & 1 bit & Tahmin yanlış mıydı? \\
\sig{correct\_pc} & 32 bit & Doğru hedef PC \\
\sig{is\_jalr} & 1 bit & Branch mi JALR mı? \\
\sig{pc\_at\_prediction} & 32 bit & Predictor update için orijinal PC \\
\sig{global\_history} & 8+ bit & Branch predictor history \\
\sig{ras\_tos} & 3 bit & RAS checkpoint pointer \\
\bottomrule
\end{tabular}
\end{table}

\begin{nedenbox}
\textbf{Neden bu kadar veri saklanıyor?}

Recovery sadece RAT restore değildir. Misprediction sonrası:
\begin{itemize}
    \item RAT restore edilmeli (snapshot)
    \item Fetch doğru PC'ye yönlendirilmeli (correct\_pc)
    \item Branch predictor güncellenmeli (pc\_at\_prediction, global\_history)
    \item RAS restore edilmeli (ras\_tos)
    \item JALR predictor güncellenmeli (is\_jalr)
\end{itemize}
Tüm bilgileri tek yerde tutmak, tek çevrimde recovery sağlar.
\end{nedenbox}

\subsubsection{BRAT Operasyonları}

\paragraph{1. Push (Dal Issue Edildiğinde)}

Yeni dal komutu issue edildiğinde, BRAT'a entry eklenir:

\begin{lstlisting}[caption={BRAT push mantığı}]
always_comb begin
    brat_push_en[0] = decode_valid[0] && branch_0 && !brat_full && !brat_restore_en;
    brat_push_en[1] = decode_valid[1] && branch_1 && !brat_full && !brat_restore_en;
    brat_push_en[2] = decode_valid[2] && branch_2 && !brat_full && !brat_restore_en;

    // Push snapshots - Store RAT state AFTER the branch instruction
    brat_push_snapshot_0 = rat_after_inst0;
    brat_push_snapshot_1 = rat_after_inst1;
    brat_push_snapshot_2 = rat_after_inst2;
end
\end{lstlisting}

\begin{nedenbox}
\textbf{Neden ``daldan sonraki'' RAT state saklanıyor?}

Snapshot, dalın kendi allocation'ını içermelidir. Misprediction durumunda:
\begin{itemize}
    \item Dalın kendisi geçerlidir (doğru yolun parçası)
    \item Daldan \textit{sonraki} komutlar geçersizdir
\end{itemize}
Bu yüzden snapshot, dalın allocation'ını içeren RAT state'i saklar.
\end{nedenbox}

\paragraph{2. Execute Result Yazma}

Dal execute edildiğinde, sonuç BRAT'a yazılır:

\begin{lstlisting}[caption={Execute result matching}]
// Execute result write interface
.exec_valid_0(exec_branch_valid_i[0]),
.exec_rob_id_0(exec_rob_id_0_i),
.exec_mispredicted_0(exec_mispredicted_i[0]),
.exec_correct_pc_0(exec_correct_pc_0_i),
\end{lstlisting}

BRAT, gelen ROB ID'yi tüm entry'lerle karşılaştırır. Eşleşen entry'nin
\sig{resolved} ve \sig{mispredicted} alanları güncellenir.

\paragraph{3. In-Order Resolution Output}

BRAT, dal sonuçlarını program sırasında çıkarır:

\begin{lstlisting}[caption={In-order resolution çıkışı}]
// BRAT ensures in-order branch resolution outputs
assign branch_resolved_o = {brat_resolved_2, brat_resolved_1, brat_resolved_0};
assign branch_mispredicted_o = {brat_mispredicted_2, brat_mispredicted_1, brat_mispredicted_0};
\end{lstlisting}

\begin{nedenbox}
\textbf{Neden in-order resolution kritik?}

Dallar out-of-order execute edilebilir. Ancak misprediction işleme sırası önemlidir:
\begin{itemize}
    \item Branch A (eski) ve Branch B (genç) aynı anda mispredicted olsun
    \item B, A'nın spekülatif yolunda olabilir
    \item A'nın misprediction'ı düzeltilirse, B zaten geçersiz
    \item B'yi önce işlemek gereksiz çalışma olur
\end{itemize}
BRAT, circular buffer yapısıyla doğal olarak oldest-first sıralama sağlar.
\end{nedenbox}

\paragraph{4. Combinational Bypass}

En düşük recovery latency için, dal execute edildiği çevrimde resolution çıkmalıdır:

\begin{lstlisting}[caption={Combinational bypass}]
// Same-cycle resolution using bypass
always_comb begin
    if (exec_valid_0 && (exec_rob_id_0 == head_branch_phys)) begin
        // Bypass: use incoming execute result directly
        branch_resolved_o_0 = 1'b1;
        branch_mispredicted_o_0 = exec_mispredicted_0;
        correct_pc_o_0 = exec_correct_pc_0;
    end else begin
        // Use stored value
        branch_resolved_o_0 = head_resolved;
        branch_mispredicted_o_0 = head_mispredicted;
        correct_pc_o_0 = head_correct_pc;
    end
end
\end{lstlisting}

\begin{nedenbox}
\textbf{Neden combinational bypass?}

Bypass olmadan akış:
\begin{enumerate}
    \item Cycle N: Execute sonucu gelir, BRAT'a yazılır
    \item Cycle N+1: BRAT'tan okunur, diğer modüllere iletilir
    \item Cycle N+2: Fetch yeni PC'den başlar
\end{enumerate}

Bypass ile:
\begin{enumerate}
    \item Cycle N: Execute sonucu gelir, AYNI ANDA çıkışa iletilir
    \item Cycle N+1: Fetch yeni PC'den başlar
\end{enumerate}

1 çevrim kazanç, yüksek misprediction oranlarında önemli performans farkı yaratır.
\end{nedenbox}

\paragraph{5. Commit Update}

ROB commit olduğunda, BRAT snapshot'ları güncellenir:

\begin{lstlisting}[caption={Commit update mantığı}]
// For each commit, update ALL snapshots that point to this ROB entry
for (int i = 0; i < BRAT_DEPTH; i++) begin
    if (commit_valid && snapshot[i][arch_addr] == rob_idx) begin
        snapshot[i][arch_addr] <= rf_mapping;  // Point to RF instead of ROB
    end
end
\end{lstlisting}

\begin{nedenbox}
\textbf{Neden commit update gerekli?}

Snapshot alındığında, bazı register'lar ROB'a işaret eder. ROB commit olduğunda:
\begin{itemize}
    \item Değer ROB'dan RF'e kopyalanır
    \item ROB entry yeniden kullanılabilir
\end{itemize}

Snapshot güncellenmezse, restore sırasında:
\begin{itemize}
    \item Geçersiz ROB pointer kullanılır
    \item Yanlış veri okunur
\end{itemize}

Bu yüzden commit, tüm snapshot'larda ilgili mapping'i RF'e günceller.
\end{nedenbox}

\paragraph{6. RAT Restore}

Misprediction tespit edildiğinde, RAT snapshot'tan restore edilir:

\begin{lstlisting}[caption={RAT restore}]
always_ff @(posedge clk) begin
    if (brat_restore_en) begin
        for (int i = 0; i < ARCH_REGS; i++) begin
            // Handle same-cycle commit
            if (commit_valid[0] && commit_addr_0 == i &&
                commit_rob_idx_0 == brat_restore_snapshot[i][4:0]) begin
                rat_table[i] <= {1'b0, commit_addr_0};
            end else begin
                rat_table[i] <= brat_restore_snapshot[i];
            end
        end
    end
end
\end{lstlisting}

\subsubsection{RAS Checkpoint/Restore}

BRAT, Return Address Stack için de checkpoint tutar:

\begin{lstlisting}[caption={RAS checkpoint}]
.push_ras_tos_0(push_ras_tos_i),
.ras_restore_valid_o(ras_restore_valid_o),
.ras_restore_tos_o(ras_restore_tos_o)
\end{lstlisting}

Misprediction durumunda RAS pointer da restore edilir, böylece fonksiyon
dönüş tahminleri doğru kalır.

\subsubsection{BRAT Özet Tablosu}

\begin{table}[H]
\centering
\caption{BRAT özellikleri}
\label{tab:brat_summary}
\begin{tabular}{lp{8cm}}
\toprule
\textbf{Özellik} & \textbf{Değer/Açıklama} \\
\midrule
Derinlik & 16 entry (maksimum in-flight branch) \\
Snapshot Boyutu & 32 × 6 bit = 192 bit/entry \\
Toplam Depolama & 16 × (~250 bit) ≈ 4 Kbit \\
Push Genişliği & 3-wide (her cycle 3 branch) \\
Resolution Genişliği & 3-wide (her cycle 3 resolution) \\
Recovery Latency & 0 cycle (combinational bypass) \\
\bottomrule
\end{tabular}
\end{table}


\subsection{RAT Özet Tablosu}

\begin{table}[H]
\centering
\caption{RAT özellikleri}
\label{tab:rat_summary}
\begin{tabular}{lp{8cm}}
\toprule
\textbf{Özellik} & \textbf{Değer/Açıklama} \\
\midrule
Mapping Genişliği & 32 arch → 64 phys \\
Rename Genişliği & 3-wide (her cycle 3 komut) \\
Lookup Genişliği & 6-wide (3×rs1 + 3×rs2) \\
Same-Cycle Forwarding & Destekleniyor \\
Misprediction Recovery & BRAT snapshot restore \\
Commit Update & RF mapping restore \\
\bottomrule
\end{tabular}
\end{table}
