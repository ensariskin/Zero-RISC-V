\phantomsection
%%%%%%%%%%%%%%%%%%%%%%%%%%%%%%%%%%%%%%%%%%%%%%%%%%%%%%%%%%%%%%%%%
\chapter{TEMEL KAVRAMLAR VE LİTERATÜR}\label{ch:temel_kavramlar}
%%%%%%%%%%%%%%%%%%%%%%%%%%%%%%%%%%%%%%%%%%%%%%%%%%%%%%%%%%%%%%%%%

\section{RISC-V Mimarisi}\label{sec:riscv}

% TODO: Açık kaynak ISA konsepti, tarihçe
% TODO: RV32I taban komut seti, temel komut formatları (R, I, S, B, U, J)
% TODO: Machine mode odaklı kısa bilgi

%------------------------------------------------------------------------

\section{İşlemci Mikro Mimarisi Kavramları}\label{sec:mikromimarisi}

\subsection{Boruhattı ve Tehlikeler}\label{subsec:pipeline_hazards}

% TODO: Klasik 5-aşamalı pipeline (Fetch, Decode, Execute, Memory, Writeback)
% TODO: Data, Control, Structural hazards
% TODO: Forwarding, stalling, branch prediction

\subsection{Süperölçekli Yürütme}\label{subsec:superscalar}

% TODO: Instruction Level Parallelism (ILP)
% TODO: Çoklu komut getirme ve yürütme

\subsection{Sıra Dışı Yürütme}\label{subsec:ooo}

% TODO: Dinamik zamanlama
% TODO: Register renaming, WAW/WAR eliminasyonu
% TODO: Reorder buffer

%------------------------------------------------------------------------

\section{Hata Toleransı Kavramları}\label{sec:fault_tolerance}

\subsection{Hata Türleri}\label{subsec:fault_types}

% TODO: Geçici hatalar (Transient faults)
% TODO: SEU, MBU kavramları
% TODO: Kalıcı hatalar (Permanent faults)

\subsection{Yedeklilik Teknikleri}\label{subsec:redundancy}

% TODO: Uzaysal yedeklilik (DMR, TMR)
% TODO: Zamansal yedeklilik
% TODO: Bilgi yedekliliği (ECC)

%------------------------------------------------------------------------

\section{İlgili Çalışmalar}\label{sec:related_work}

\subsection{Süperölçekli İşlemci Tasarımları}\label{subsec:superscalar_designs}

% TODO: BOOM, Rocket, NaxRiscv
% TODO: Diğer önemli süperölçekli tasarımlar

\subsection{Hata Toleranslı İşlemci Tasarımları}\label{subsec:ft_processors}

% TODO: DCLS, TMR tabanlı çalışmalar
% TODO: DuckCore, Trikarenos gibi hibrit yaklaşımlar

